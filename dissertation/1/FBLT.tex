\documentclass[12pt,english]{report}

\makeatletter
\newif\if@restonecol

\let\algorithm\relax
\let\endalgorithm\relax
%\usepackage[tight,footnotesize]{subfigure}
\usepackage{subfigure}
\usepackage {paralist}
\usepackage{comment}
\usepackage{array}
\usepackage{amsmath}
\usepackage{graphicx}
\usepackage{amssymb}
\usepackage{cite}
\usepackage{url}
\usepackage[ruled]{algorithm2e}
\usepackage{babel}
\usepackage{listings}
%\usepackage[pdftex, breaklinks,colorlinks=false,pdfborder={0 0 0}]{hyperref}
%\usepackage[all]{hypcap}
\makeatother
\makeatletter
\providecommand{\tabularnewline}{\\}


\newtheorem{clm}{Claim}
\newtheorem{proof}{Proof}
\newtheorem{mydef}{Definition}


\makeatletter

%%%%%%%%%%%%%%%%%%%%%%%%%%%%%% LyX specific LaTeX commands.
\pdfpageheight\paperheight
\pdfpagewidth\paperwidth

%% Because html converters don't know tabularnewline
\providecommand{\tabularnewline}{\\}
%% A simple dot to overcome graphicx limitations
\newcommand{\lyxdot}{.}


%%%%%%%%%%%%%%%%%%%%%%%%%%%%%% User specified LaTeX commands.



\makeatother



\setlength{\textwidth}{6.5in}
\setlength{\textheight}{8.5in}
\setlength{\evensidemargin}{0in}
\setlength{\oddsidemargin}{0in}
\setlength{\topmargin}{0in}

\setlength{\parindent}{0pt}
\setlength{\parskip}{0.1in}

% Uncomment for double-spaced document.
% \renewcommand{\baselinestretch}{2}

% \usepackage{epsf}

\begin{document}

\thispagestyle{empty}
\pagenumbering{roman}

% TITLE
{\Large 
FLBT
}

\section{Motivation}
\textbf{NEEDS TO BE WRITTEN}

\section{FBLT}

\begin{algorithm}[h]
\footnotesize{
\LinesNumbered
\KwData{
$s_i^k$: interfered transaction\;
$s_j^l$: interfering transactions\;
$\delta_i^k(L)$: the maximum number of times $s_i^k$ can be aborted during an interval $L$\;
$\eta_i^k$: number of times $s_i^k$ has already been aborted during an interval $L$\;
$m\_$set: contains at most m non-preemptive transactions\;
$m\_prio$: priority of any transaction in $m\_$set. $m\_prio$ is higher than any priority of any real-time task\;
$r(s_i^k)$: time point at which $s_i^k$ joined $m\_$set\;
}
\KwResult{atomic sections that will abort}
\eIf{\label{fblt_first}$p_j^l > p_i^k$}
{
\eIf{$\eta_i^k(L) \le \delta_i^k(L)$}
{
Increment $\eta_i^k$ by 1\;
Abort $s_i^k$\;
}
{
\If{$s_i^k \not\in m\_$set}
{
Add $s_i^k$ to $m\_$set\;
Record $r(s_i^k)$\;
Increase priority of $s_i^k$ to $m\_prio$\;
}
\eIf{$s_j^l \in m\_$set AND $r(s_j^l)<r(s_i^k)$}
{
Abort $s_i^k$\;
}
{
Abort $s_j^l$\;
}
}
}
{
Swap $s_i^k$ and $s_j^l$\;
Go to Step~\ref{clm:fblt_ecm}
}
}
\caption{FBLT}\label{fblt-algorithm}
\end{algorithm}

\subsection{Illustrative Example}
NEEDS TO BE WRITTEN

\section{Properties}


\section{Retry Cost Under FBLT}

We now derive an upper bound on the retry cost of any job $\tau_{i}^{x}$
under FBLT during an interval $L\le T_{i}$. Since all tasks are sporadic
(i.e., each task $\tau_{i}$ has a minimum period $T_{i}$), $T_{i}$
is the maximum study interval for each task $\tau_{i}$.

\begin{clm}

The total retry cost for any job $\tau_{i}^{x}$ due to: 1) conflicts
between its transactions and transactions of other jobs under FBLT
during an interval $L\le T_{i}$. 2) release of higher priority jobs,
is upper bounded by:

\begin{equation}
RC_{to}(L)\le\sum_{\forall s_{i}^{k}\in s_{i}}\left(\delta_{i}^{k}(L)len(s_{i}^{k})+\sum_{s_{iz}^k\in \chi_i^k} len(s_{iz}^{k})\right)+RC_{re}(L)\label{eq:fblt_rc}
\end{equation} 
where $\chi_i^k$ is the set of at most $m-1$ maximum length transactions sharing objects with $s_i^k$. Each transaction $s_{iz}^k \in \chi_i^k$ belongs to a distinct task $\tau_j, j \ne i$. $RC_{re}(L)$ is the retry cost resulting
from release of higher piority jobs which preempt $\tau_{i}^{x}$.
$RC_{re}(L)$ is calculated by (6.8) in \cite{shambake_phd_proposal}
for G-EDF, and (6.10) in \cite{shambake_phd_proposal} for G-RMA.

\end{clm}

\begin{proof}

By definition of FBLT, $s_{i}^{k}\in\tau_{i}^{x}$ can be aborted
at maximum $\delta_{i}^{k}(L)$ times during interval $L$ before
$s_{i}^{k}$ joins $m\_$set. Before joining $m\_$set, $s_{i}^{k}$
can be aborted due to higher priority transactions, or transactions
in the $m\_$set. Original priority (before step \label{fblt:increase_priority})
of transactions in $m\_$set can be of higher or lower priority than
$p_{i}^{x}$. Thus, the maximum time $s_{i}^{k}$ is aborted before
joining $m\_$set occurs if $s_{i}^{k}$ is aborted for $\delta_{i}^{k}$.
The worst case scenario for $s_{i}^{k}$ after joining $m\_$set occurs
if $s_{i}^{k}$ is preceded by $m-1$ maximum lenght conflicting transactions.
Hence, $s_{i}^{k}$ has to wait for the previous $m-1$ transactions
to commit first. Priority of $s_{i}^{k}$ after joining $m\_$set
is higher than any real-time task. So, $s_{i}^{k}$ is not aborted
by any task. If $s_{i}^{k}$ has not joined $m\_$set yet, and a higher
piority job $\tau_{j}^{y}$ is released while $s_{i}^{k}$ is running,
then $s_{i}^{k}$ may be aborted if $\tau_{j}^{y}$ has conflicting
transactions with $s_{i}^{k}$. $\tau_{j}^{y}$ causes only one abort
in $\tau_{i}^{x}$ because $\tau_{j}^{y}$ preempts $\tau_{i}^{x}$
only once. If $s_{i}^{k}$ has already joined $m\_$set, then $s_{i}^{k}$
cannot be aborted by release of higher priority jobs. So, the maximum
number of abort times to transactions in $\tau_{i}^{x}$ due to release
of higher prirority jobs is less or equal to number of interfering
higher priority jobs to $\tau_{i}^{x}$. Claim follows.

\end{proof}

\begin{clm}

The blocking time for a job $\tau_{i}^{x}$ due to lower priority
jobs during an interval $L\le T_{i}$ is upper bounded by: 
\begin{equation}
D(\tau_{i}^{x})=min\left(max_{1}^{m}(s_{j_{max},\forall\tau_{j}^{l},\, p_{j}^{l}<p_{i}^{x}})\right)\label{eq:fblt_delay}
\end{equation}
where $s_{j_{max}}$ is the maximum length transaction in any job
$\tau_{j}^{l}$ with original priority lower than $p_{i}^{x}$. The
right hand side of (\ref{eq:fblt_delay}) is the minimum of the $m$
maximum transactional lengths in all jobs with lower priority than
$\tau_{i}^{x}$.

\end{clm}

\begin{proof}

$\tau_{i}^{x}$ is blocked when it is initially released and all processors
are busy with lower priority jobs with non-preemptive transactions.
Although $\tau_{i}^{x}$ can be preempted by higher priority jobs,
$\tau_{i}^{x}$ cannot be blocked after it is released. If $\tau_{i}^{x}$
is preempted by a higher priority job $\tau_{j}^{y}$, then $\tau_{j}^{y}$
finishes execution, the underlying scheduler will not choose a lower
priority job than $\tau_{i}^{x}$ before $\tau_{i}^{x}$. So, after
$\tau_{i}^{x}$ is released, there is no chance for any transaction
$s_{u}^{v}$ belonging to a lower priority job than $\tau_{i}^{x}$
to run before $\tau_{i}^{x}$. Thus, $s_{u}^{v}$ cannot join $m\_$set
before $\tau_{i}^{x}$ finishes. Consequently, the worst case blocking
time for $\tau_{i}^{x}$ occurs when the maximum length $m$ transactions
in lower priority jobs than $\tau_{i}^{x}$ are executing non-preemptively.
After the minimum length transaction in the $m\_$set finishes, the
underlying scheduler will choose $\tau_{i}^{x}$ or a higher priority
job to run. Claim follows.

\end{proof}

\begin{clm}

Response time of any job $\tau_{i}^{x}$ during an interval $L\le T_{i}$
under FBLT is upper bounded by 
\begin{equation}
R_{i}^{up}=c_{i}+RC_{to}(L)+D(\tau_{i}^{x})+\left\lfloor \frac{1}{m}\sum_{\forall j\ne i}W_{ij}(R_{i}^{up})\right\rfloor \label{eq:fblt_res_time}
\end{equation}
where $RC_{to}(L)$ is calculated by (\ref{eq:fblt_rc}), $D(\tau_{i}^{x})$
is calculated by by (\ref{eq:fblt_delay}), and $W_{ij}(R_{i}^{up})$
is calculated by (11) in \cite{stmconcurrencycontrol:emsoft11} for
G-EDF, and (17) in \cite{stmconcurrencycontrol:emsoft11} for G-RMA.
(11) and (17) in \cite{stmconcurrencycontrol:emsoft11} inflates $c_{j}$
of any job of $\tau_{j}\ne\tau_{i},\, p_{j}>p_{i}$ by retry cost
of transactions in $\tau_{j}$.

\end{clm}

\begin{proof}

Response time of any job $\tau_{i}^{x}$ is a direct result of FBLT
bahviour. Response time of any job $\tau_{i}^{x}$ is the sum of its
worst case execution time $c_{i}$, plus retry cost of transactions
in $\tau_{i}^{x}$ ($RC(L)$), plus blocking time of $\tau_{i}^{x}$
($D(\tau_{i}^{x})$), and the workload interference of higher priority
jobs. Workload interfernence of higher priority jobs scheduled by
G-EDF is calculated by (11) in \cite{stmconcurrencycontrol:emsoft11},
and by (17) in \cite{stmconcurrencycontrol:emsoft11} for G-RMA. 

\end{proof}


\section{FBLT vs. Competitors}

Let $RC_{A}(T_{i})$ denote the retry cost of any $\tau_{i}^{x}$
using the synchronization method $A$ during $T_{i}$. Let $RC_{B}(T_{i})$
denote the retry cost of any $\tau_{i}^{x}$ using synchronization
method $B$ during $T_{i}$. Then, schedulability of $A$ is comparable
to $B$ if 
\begin{eqnarray}
\sum_{\forall\tau_{i}}\frac{c_{i}+RC_{A}(T_{i})}{T_{i}} & \le & \sum_{\forall\tau_{i}}\frac{c_{i}+RC_{B}(T_{i})}{T_{i}}\nonumber \\
\sum_{\forall\tau_{i}}\frac{RC_{A}(T_{i})}{T_{i}} & \le & \sum_{\forall\tau_{i}}\frac{RC_{B}(T_{i})}{T_{i}}\label{eq:utilization comparison}
\end{eqnarray}



\subsection{FBLT vs. ECM}

\begin{clm}\label{clm:fblt_ecm}

Schedulability of FBLT is equal or better to ECM's when maximum number
of abort times of any transaction $s_i^k$ in any job of $\tau_{i}$ is less
or equal to number of conflicting transactions to $s_i^k$ in all other jobs with higher priority than priority of current job of $\tau_{i}$.

\end{clm}

\begin{proof}

By substituing $RC_{A}(T_{i})$ and $RC_{B}(T_{i})$ in (\ref{eq:utilization comparison})
with (\ref{eq:fblt_rc}) and (6.7) in \cite{shambake_phd_proposal}
respectively. 

\begin{eqnarray}
 & \sum_{\forall\tau_{i}}\frac{\sum_{\forall s_{i}^{k}\in s_{i}}\left(\delta_{i}^{k}(T_{i})len(s_{i}^{k})+\sum_{s_{iz}^k\in \chi_i^k} len(s_{iz}^{k})\right)+RC_{re}(T_{i})}{T_{i}}\label{eq:fblt_edf_comparison_1}\\
\le & \sum_{\forall\tau_{i}}\frac{\left(\sum_{\forall\tau_{j}\in\gamma_{i}^{ex}}\sum_{\theta\in\theta_{i}^{ex}}\left(\left\lceil \frac{T_{i}}{T_{j}}\right\rceil \sum_{\forall\bar{s_{j}^{h}}(\theta)}len\left(\bar{s_{j}^{h}}(\theta)+s_{max}^{j}(\theta)\right)\right)\right)+RC_{re}(T_{i})}{T_{i}}\nonumber 
\end{eqnarray}


Let $\theta_{i}^{ex}=\theta_{i}+\theta_{i}^{*}$ where $\theta_{i}^{*}$
is the set of objects not accessed directly by $\tau_{i}$ but can
enforce transactions in $\tau_{i}$ to retry due to transitive retry.
Let $\gamma_{i}^{ex}=\gamma_{i}+\gamma_{i}^{*}$where $\gamma_{i}^{*}$
is the set of tasks that access objects in $\theta_{i}^{*}$. $\bar{s_{j}^{h}}(\theta)$
can access multiple objects, so $s_{max}^{j}(\theta)$ is the maximum
length transaction conflicting with $\bar{s_{j}^{h}}(\theta)$. FBLT
needs to bound transitive retry. So, $\theta_{i}^{ex}$ will be avoided
in (\ref{eq:fblt_edf_comparison_1}). Thus, transactions under ECM behave
as if there were no transitive retry. Consequently, (\ref{eq:fblt_edf_comparison_1})
will be 
\begin{eqnarray}
 & \sum_{\forall\tau_{i}}\frac{\sum_{\forall s_{i}^{k}\in s_{i}}\left(\delta_{i}^{k}(T_{i})len(s_{i}^{k})+\sum_{s_{iz}^k\in \chi_i^k} len(s_{iz}^{k})\right)}{T_{i}}\label{eq:fblt_edf_comparison_2}\\
\le & \sum_{\forall\tau_{i}}\frac{\left(\sum_{\forall\tau_{j}\in\gamma_{i}}\sum_{\theta\in\theta_{i}}\left(\left\lceil \frac{T_{i}}{T_{j}}\right\rceil \sum_{\forall\bar{s_{j}^{h}}(\theta)}len\left(\bar{s_{j}^{h}}(\theta)+s_{max}^{j}(\theta)\right)\right)\right)}{T_{i}}\nonumber 
\end{eqnarray}
As $\bar{s_{j}^{h}}(\theta)$ is included only once for all objects
accessed by it. $s_{max}^{j}(\theta)$ is also included once for each
$\bar{s_{j}^{h}}(\theta)$. Consequently, (\ref{eq:fblt_edf_comparison_2})
becomes 
\begin{eqnarray}
 & \sum_{\forall\tau_{i}}\frac{\sum_{\forall s_{i}^{k}\in s_{i}}\left(\delta_{i}^{k}(T_{i})len(s_{i}^{k})+\sum_{s_{iz}^k\in \chi_i^k} len(s_{iz}^{k})\right)}{T_{i}}\label{eq:fblt_edf_comparison_3}\\
\le & \sum_{\forall\tau_{i}}\frac{\left(\sum_{\forall\tau_{j}\in\gamma_{i}}\sum_{\bar{s_{j}^{h}}(\theta),\,\theta\in\theta_{i}}\left(\left\lceil \frac{T_{i}}{T_{j}}\right\rceil len\left(\bar{s_{j}^{h}}(\theta)+s_{max}^{j}(\theta)\right)\right)\right)}{T_{i}}\nonumber 
\end{eqnarray}
Although different $s_{i}^{k}$ can have common conflicting transactions
$\bar{s_{j}^{h}}$, but no more than one $s_{i}^{k}$ can be preceeded
by the same $\bar{s_{j}^{h}}$ in the $m\_$set. This happens because
transactions in the $m\_$set are non-preemptive. Original priority
of transactions preeceding $s_{i}^{k}$ in the $m\_$set can be of
lower or higher priority than original priority of $s_{i}^{k}$. Under
G-EDF, $\tau_{j}\ne\tau_{i}$ can have at least one job of higher
priority than $\tau_{i}^{x}$, then $\left\lceil \frac{T_{i}}{T_{j}}\right\rceil \ge1$.
So, each one of the $s_{iz}^{k}$ in the left hand side of (\ref{eq:fblt_edf_comparison_3})
is included in one of the $\bar{s_{j}^{h}}(\theta)$ in the right hand
side of (\ref{eq:fblt_edf_comparison_3}). $\therefore$ (\ref{eq:fblt_edf_comparison_3})
holds if 
\begin{eqnarray}
 & \sum_{\forall\tau_{i}}\frac{\sum_{\forall s_{i}^{k}\in s_{i}}\delta_{i}^{k}(T_{i})len(s_{i}^{k})}{T_{i}}\label{eq:fblt_edf_comparison_4}\\
\le & \sum_{\forall\tau_{i}}\frac{\left(\sum_{\forall\tau_{j}\in\gamma_{i}}\sum_{\bar{s_{j}^{h}}(\theta),\,\theta\in\theta_{i}}\left(\left\lceil \frac{T_{i}}{T_{j}}\right\rceil len\left(s_{max}^{j}(\theta)\right)\right)\right)}{T_{i}}\nonumber 
\end{eqnarray}
For each $s_{i}^{k}\in s_{i}$, there are a set of zero or more $\bar{s_{j}^{h}}(\theta)\in\tau_{j},\,\forall\tau_{j}\ne\tau_{i}$
that are conflicting with $s_{i}^{k}$. Assuming this set of conflicting
transactions with $s_{i}^{k}$ is denoted as $\eta_{i}^{k}=\left\{ \bar{s_{j}^{h}}(\theta)\in\tau_{j}:\left(\theta\in\theta_{i}\right)\wedge\left(\forall\tau_{j}\ne\tau_{i}\right)\wedge\left(\bar{s_{j}^{h}}(\theta)\not\in\eta_{i}^{l},\, l\ne k\right)\right\} $.
The last condition $\bar{s_{j}^{h}}(\theta)\not\in\eta_{i}^{l},\, l\ne k$
in defintion of $\eta_{i}^{k}$ ensures that common transactions $\bar{s_{j}^{h}}$
that can conflict with more than one transaction $s_{i}^{k}\in\tau_{i}$
are split among different $\eta_{i}^{k},\, k=1,..,|s_{i}|$. This
condition is necessary because in ECM, no two or more transactions
of $\tau_{i}^{x}$ can be aborted by the same transaction of $\tau_{j}^{h}$
where $p_{j}^{h}>p_{i}^{x}$. By substitution of $\eta_{i}^{k}$ in
(\ref{eq:fblt_edf_comparison_4}), we get 
\begin{eqnarray}
 & \sum_{\forall\tau_{i}}\frac{\sum_{\forall s_{i}^{k}\in s_{i}}\delta_{i}^{k}(T_{i})len(s_{i}^{k})}{T_{i}}\label{eq:fblt_edf_comparison_5}\\
\le & \sum_{\forall\tau_{i}}\frac{\left(\sum_{\forall k=1}^{|s_{i}|}\sum_{\bar{s_{j}^{h}}(\theta)\in\eta_{i}^{k}}\left(\left\lceil \frac{T_{i}}{T_{j}}\right\rceil len\left(s_{max}^{j}(\theta)\right)\right)\right)}{T_{i}}\nonumber 
\end{eqnarray}
$\because\, len\left(s_{max}^{j}(\theta)\right)\ge len(s_{i}^{k})$,
$\therefore$ (\ref{eq:fblt_edf_comparison_5}) holds if for each $s_{i}^{k}\in\tau_{i}$
\begin{equation*}
\delta_{i}^{k}\le\frac{\sum_{\bar{s_{j}^{h}}(\theta)\in\eta_{i}^{k}}\left(\left\lceil \frac{T_{i}}{T_{j}}\right\rceil len\left(s_{i}^{k}\right)\right)}{len(s_{i}^{k})}=\sum_{\bar{s_{j}^{h}}(\theta)\in\eta_{i}^{k}}\left\lceil \frac{T_{i}}{T_{j}}\right\rceil
\end{equation*}

So, $\sum_{\bar{s_{j}^{h}}(\theta)\in\eta_{i}^{k}}\left\lceil \frac{T_{i}}{T_{j}}\right\rceil$
is the maximum number of conflicting transactions with $s_i^k$ in all jobs with higher priority than priority of current job of $\tau_{i}$. Claim follows.

\end{proof}

\subsection{FBLT vs. RCM}

\begin{clm}\label{clm:fblt_rcm}

Schedulability of FBLT is equal or better to RCM's when maximum number
of abort times of each transaction $s_{i}^{k}$ in any job of $\tau_{i}$
is less or equal to number of conflicting transactions with $s_{i}^{k}$
in all jobs with higher priority than $\tau_{i}$ minus sum of maximum
length $m-1$ transactions conflicting with $s_{i}^{k}$.

\end{clm}

\begin{proof}

By substituing $RC_{A}(T_{i})$ and $RC_{B}(T_{i})$ in (\ref{eq:utilization comparison})
with (\ref{eq:fblt_rc}) and (6.9) in \cite{shambake_phd_proposal}
respectively. 

\begin{eqnarray}
 & \sum_{\forall\tau_{i}}\frac{\sum_{\forall s_{i}^{k}\in s_{i}}\left(\delta_{i}^{k}(T_{i})len(s_{i}^{k})+\sum_{s_{iz}^k\in \chi_i^k} len(s_{iz}^{k})\right)+RC_{re}(T_{i})}{T_{i}}\label{eq:fblt_rcm_comparison_1}\\
\le & \sum_{\forall\tau_{i}}\frac{\left(\sum_{\forall\tau_{j}^{*}\in\gamma_{i}^{ex}}\sum_{\forall\theta\in\theta_{i}^{ex}}\left(\left\lceil \frac{T_{i}}{T_{j}}\right\rceil +1\right)\sum_{\forall\bar{s_{j}^{h}(\theta)}}len\left(\bar{s_{j}^{h}(\theta)}+s_{max}^{j}(\theta)\right)\right)+RC_{re}(T_{i})}{T_{i}}\nonumber 
\end{eqnarray}
where $\tau_{j}^{*}=\left\{ \tau_{j}:\left(\tau_{j}\ne\tau_{i}\right)\wedge\left(p_{j}>p_{i}\right)\right\} $.
Let $\theta_{i}^{ex}=\theta_{i}+\theta_{i}^{*}$ where $\theta_{i}^{*}$
is the set of objects not accessed directly by $\tau_{i}$ but can
enforce transactions in $\tau_{i}$ to retry due to transitive retry.
Let $\gamma_{i}^{ex}=\gamma_{i}+\gamma_{i}^{*}$where $\gamma_{i}^{*}$
is the set of tasks that access objects in $\theta_{i}^{*}$. $\bar{s_{j}^{h}}(\theta)$
can access multiple objects, so $s_{max}^{j}(\theta)$ is the maximum
length transaction conflicting with $\bar{s_{j}^{h}}(\theta)$. FBLT
needs to bound transitive retry. So, $\theta_{i}^{ex}$ will be avoided
in (\ref{eq:fblt_rcm_comparison_1}). Thus, transactions under RCM behave
as if there were no transitive retry. Consequently, (\ref{eq:fblt_rcm_comparison_1})
will be 
\begin{eqnarray}
 & \sum_{\forall\tau_{i}}\frac{\sum_{\forall s_{i}^{k}\in s_{i}}\left(\delta_{i}^{k}(T_{i})len(s_{i}^{k})+\sum_{s_{iz}^k\in \chi_i^k} len(s_{iz}^{k})\right)}{T_{i}}\label{eq:fblt_rcm_comparison_2}\\
\le & \sum_{\forall\tau_{i}}\frac{\sum_{\forall\tau_{j}^{*}\in\gamma_{i}}\sum_{\forall\theta\in\theta_{i}}\left(\left\lceil \frac{T_{i}}{T_{j}}\right\rceil +1\right)\sum_{\forall\bar{s_{j}^{h}(\theta)}}len\left(\bar{s_{j}^{h}(\theta)}+s_{max}^{j}(\theta)\right)}{T_{i}}\nonumber 
\end{eqnarray}
As $\bar{s_{j}^{h}}(\theta)$ is included only once for all objects
accessed by it, $s_{max}^{j}(\theta)$ is also included once for each
$\bar{s_{j}^{h}}(\theta)$. Consequently, (\ref{eq:fblt_rcm_comparison_2})
becomes 
\begin{eqnarray}
 & \sum_{\forall\tau_{i}}\frac{\sum_{\forall s_{i}^{k}\in s_{i}}\left(\delta_{i}^{k}(T_{i})len(s_{i}^{k})+\sum_{s_{iz}^k\in \chi_i^k} len(s_{iz}^{k})\right)}{T_{i}}\label{eq:fblt_rcm_comparison_3}\\
\le & 
\sum_{\forall\tau_{i}}\frac{\left(\sum_{\forall\tau_{j}^{*}\in\gamma_{i}}\sum_{\bar{s_{j}^{h}}(\theta),\,\theta\in\theta_{i}}\left(\left\lceil \frac{T_{i}}{T_{j}}\right\rceil +1\right)len\left(\bar{s_{j}^{h}}(\theta)+s_{max}^{j}(\theta)\right)\right)}{T_{i}}\nonumber 
\end{eqnarray}

Although different $s_{i}^{k}$ can have common conflicting transactions
$\bar{s_{j}^{h}}$, but no more than one $s_{i}^{k}$ can be preceeded
by the same $\bar{s_{j}^{h}}$ in the $m\_$set. This happens because
transactions in the $m\_$set are non-preemptive. Original priority
of transactions preeceding $s_{i}^{k}$ in the $m\_$set can be of
lower or higher priority than original priority of $s_{i}^{k}$. Under
G-RMA, $p_{j}>p_{i}$ means that $T_{j}\le T_{i}$, then $\left\lceil \frac{T_{i}}{T_{j}}\right\rceil \ge1$.
For each $s_{i}^{k}\in s_{i}$, there are a set of zero or more $\bar{s_{j}^{h}}(\theta)\in\tau_{j}^{*}$
that are conflicting with $s_{i}^{k}$. Assuming this set of conflicting
transactions with $s_{i}^{k}$ is denoted as $\eta_{i}^{k}=\left\{ \bar{s_{j}^{h}}(\theta)\in\tau_{j}^{*}:\left(\theta\in\theta_{i}\right)\wedge\left(\bar{s_{j}^{h}}(\theta)\not\in\eta_{i}^{l},\, l\ne k\right)\right\} $.
The last condition $\bar{s_{j}^{h}}(\theta)\not\in\eta_{i}^{l},\, l\ne k$
in defintion of $\eta_{i}^{k}$ ensures that common transactions $\bar{s_{j}^{h}}$
that can conflict with more than one transaction $s_{i}^{k}\in\tau_{i}$
are split among different $\eta_{i}^{k},\, k=1,..,|s_{i}|$. This
condition is necessary because in RCM, no two or more transactions
of $\tau_{i}^{x}$ can be aborted by the same transaction of $\tau_{j}^{h}$
where $p_{j}^{h}>p_{i}^{x}$. By substitution of $\eta_{i}^{k}$ in
(\ref{eq:fblt_rcm_comparison_3}), we get 
\begin{eqnarray}
 & \sum_{\forall\tau_{i}}\frac{\sum_{\forall s_{i}^{k}\in s_{i}}\left(\delta_{i}^{k}(T_{i})len(s_{i}^{k})+\sum_{s_{iz}^k\in \chi_i^k} len(s_{iz}^{k})\right)}{T_{i}}\label{eq:fblt_rcm_comparison_4}\\
\le & \sum_{\forall\tau_{i}}\frac{\left(\sum_{\forall k=1}^{|s_{i}|}\sum_{\bar{s_{j}^{h}}(\theta)\in\eta_{i}^{k}}\left(\left(\left\lceil \frac{T_{i}}{T_{j}}\right\rceil +1\right)len\left(\bar{s_{j}^{h}}(\theta)+s_{max}^{j}(\theta)\right)\right)\right)}{T_{i}}\nonumber 
\end{eqnarray}


$\bar{s_{j}^{h}}$ belongs to higher priority jobs than $\tau_{i}$
and $s_{max}^{j}$ belongs to higher priority jobs than $\tau_{i}$
or $\tau_{i}$ itself. Transactions in $m\_$set can belong to jobs
with original priority higher or lower than $\tau_{i}$. So, (\ref{eq:fblt_rcm_comparison_3})
holds if for each $s_{i}^{k}\in\tau_{i}$ 
\begin{equation}
\delta_{i}^{k}(T_{i})len(s_{i}^{k})\le\left(\sum_{\bar{s_{j}^{h}}(\theta)\in\eta_{i}^{k}}\left(\left(\left\lceil \frac{T_{i}}{T_{j}}\right\rceil +1\right)len\left(\bar{s_{j}^{h}}(\theta)+s_{max}^{j}(\theta)\right)\right)\right)-\sum_{s_{iz}^k\in \chi_i^k} len(s_{iz}^{k})\label{eq:fblt_rcm_comparison_5}
\end{equation}
\begin{equation}
\therefore\,\delta_{i}^{k}(T_{i})\le\left(\sum_{\bar{s_{j}^{h}}(\theta)\in\eta_{i}^{k}}\left(\left(\left\lceil \frac{T_{i}}{T_{j}}\right\rceil +1\right)len\left(\frac{\bar{s_{j}^{h}}(\theta)+s_{max}^{j}(\theta)}{s_{i}^{k}}\right)\right)\right)-\sum_{s_{iz}^k \in \chi_i^k} len\left(\frac{s_{iz}^{k}}{s_{i}^{k}}\right)\label{eq:fblt_rcm_comparison_14}
\end{equation}
$\because\, len\left(\frac{s_{iz}^{k}}{s_{i}^{k}}\right)\le len(s_{iz}^{k})$,
and $len(s_{max}^{j}(\theta))>len(s_{i}^{k})$ $\therefore$ (\ref{eq:fblt_rcm_comparison_14})
holds if 
\begin{equation*}
\therefore\,\delta_{i}^{k}(T_{i})\le\left(\sum_{\bar{s_{j}^{h}}(\theta)\in\eta_{i}^{k}}\left(\left\lceil \frac{T_{i}}{T_{j}}\right\rceil +1\right)\right)-\sum_{s_{iz}^k \in \chi_i^k} len\left(s_{iz}^{k}\right)
\end{equation*}
$\sum_{\bar{s_{j}^{h}}(\theta)\in\eta_{i}^{k}}\left(\left\lceil \frac{T_{i}}{T_{j}}\right\rceil +1\right)$
represents number of conflicting transactions with $s_{i}^{k}$ in
all jobs with higher priority than $\tau_{i}$. $\sum_{s_{iz}^k \in \chi_i^k} len\left(s_{iz}^{k}\right)$
is sum of maximum $m-1$ transactional length transactions conflicting
with $s_{i}^{k}$. Claim follows.

\end{proof}

\subsection{FBLT vs. G-EDF/LCM}

\begin{clm}\label{clm:fblt_lcm_edf}

Schedulability of FBLT is equal or better to G-EDF/LCM's when maximum
number of abort times of each transaction $s_{i}^{k}$ in any job
of $\tau_{i}$ is less or equal to sum of total number each transaction
$s_{j}^{h}$ can conflict with $s_{i}^{k}$ multiplied by maximum
$\alpha$ with which $s_{j}^{h}$ can conflict with maximum length
transaction sharing objects with $s_{i}^{k}$ and $s_{j}^{h}$.

\end{clm}

\begin{proof}

By substituting $RC_{A}(T_{i})$ and $RC_{B}(T_{i})$ in (\ref{eq:utilization comparison})
with (\ref{eq:fblt_rc}) and (6.7) in \cite{shambake_phd_proposal}
respectively. 

\begin{eqnarray}
 & \sum_{\forall\tau_{i}}\frac{\sum_{\forall s_{i}^{k}\in s_{i}}\left(\delta_{i}^{k}(T_{i})len(s_{i}^{k})+\sum_{\forall s_{iz}^{k}\in\chi_{i}^{k}}len(s_{iz}^{k})\right)+RC_{re}(T_{i})}{T_{i}}\label{eq:fblt_lcm_edf_comparison_1}\\
\le & \sum_{\forall\tau_{i}}\frac{\left(\sum_{\forall\tau_{j}\in\gamma_{i}^{ex}}\sum_{\theta\in\theta_{i}^{ex}}\left(\left\lceil \frac{T_{i}}{T_{j}}\right\rceil \sum_{\forall\bar{s_{j}^{h}(\theta)}}len\left(\bar{s_{j}^{h}(\theta)}+\bar{\alpha_{max}^{jh}}s_{max}^{j}(\theta)\right)\right)\right)}{T_{i}}\nonumber \\
+ & \sum_{\forall\tau_{i}}\frac{\left(\sum_{\forall s_{i}^{k}}\left(1-\alpha_{max}^{ik}\right)len\left(s_{max}^{i}\right)\right)+RC_{re}(T_{i})}{T_{i}}\nonumber 
\end{eqnarray}
Let $\theta_{i}^{ex}=\theta_{i}+\theta_{i}^{*}$ where $\theta_{i}^{*}$
is the set of objects not accessed directly by $\tau_{i}$ but can
enforce transactions in $\tau_{i}$ to retry due to transitive retry.
Let $\gamma_{i}^{ex}=\gamma_{i}+\gamma_{i}^{*}$where $\gamma_{i}^{*}$
is the set of tasks that access objects in $\theta_{i}^{*}$. $\bar{s_{j}^{h}}(\theta)$
can access multiple objects, so $s_{max}^{j}(\theta)$ is the maximum
length transaction conflicting with $\bar{s_{j}^{h}}(\theta)$. FBLT
needs to bound transitive retry. So, $\theta_{i}^{ex}$ will be avoided
in (\ref{eq:fblt_lcm_edf_comparison_1}). Thus, transactions under G-EDF/LCM
behave as if there were no transitive retry. Consequently, (\ref{eq:fblt_lcm_edf_comparison_1})
will be 
\begin{eqnarray}
 & \sum_{\forall\tau_{i}}\frac{\sum_{\forall s_{i}^{k}\in s_{i}}\left(\delta_{i}^{k}(T_{i})len(s_{i}^{k})+\sum_{\forall s_{iz}^{k}\in\chi_{i}^{k}}len(s_{iz}^{k})\right)}{T_{i}}\label{eq:fblt_lcm_edf_comparison_2}\\
\le & \sum_{\forall\tau_{i}}\frac{\left(\sum_{\forall\tau_{j}\in\gamma_{i}}\sum_{\theta\in\theta_{i}}\left(\left\lceil \frac{T_{i}}{T_{j}}\right\rceil \sum_{\forall\bar{s_{j}^{h}(\theta)}}len\left(\bar{s_{j}^{h}(\theta)}+\bar{\alpha_{max}^{jh}}s_{max}^{j}(\theta)\right)\right)\right)}{T_{i}}\nonumber \\
+ & \sum_{\forall\tau_{i}}\frac{\left(\sum_{\forall s_{i}^{k}}\left(1-\alpha_{max}^{ik}\right)len\left(s_{max}^{i}\right)\right)}{T_{i}}\nonumber 
\end{eqnarray}
As $\bar{s_{j}^{h}}(\theta)$ is included only once for all objects
accessed by it. $s_{max}^{j}(\theta)$ is also included once for each
$\bar{s_{j}^{h}}(\theta)$. Consequently, (\ref{eq:fblt_lcm_edf_comparison_2})
becomes 
\begin{eqnarray}
 & \sum_{\forall\tau_{i}}\frac{\sum_{\forall s_{i}^{k}\in s_{i}}\left(\delta_{i}^{k}(T_{i})len(s_{i}^{k})+\sum_{\forall s_{iz}^{k}\in\chi_{i}^{k}}len(s_{iz}^{k})\right)}{T_{i}}\label{eq:fblt_lcm_edf_comparison_3}\\
\le & \sum_{\forall\tau_{i}}\frac{\left(\sum_{\forall\tau_{j}\in\gamma_{i}}\sum_{\forall\bar{s_{j}^{h}(\theta)},\theta\in\theta_{i}}\left(\left\lceil \frac{T_{i}}{T_{j}}\right\rceil len\left(\bar{s_{j}^{h}(\theta)}+\bar{\alpha_{max}^{jh}}s_{max}^{j}(\theta)\right)\right)\right)}{T_{i}}\nonumber \\
+ & \sum_{\forall\tau_{i}}\frac{\left(\sum_{\forall s_{i}^{k}}\left(1-\alpha_{max}^{ik}\right)len\left(s_{max}^{i}\right)\right)}{T_{i}}\nonumber 
\end{eqnarray}
Although different $s_{i}^{k}$ can have common conflicting transactions
$\bar{s_{j}^{h}}$, but no more than one $s_{i}^{k}$ can be preceded
by the same $\bar{s_{j}^{h}}$ in the $m\_$set. This happens because
transactions in the $m\_$set are non-preemptive. Original priority
of transactions preceding $s_{i}^{k}$ in the $m\_$set can be of
lower or higher priority than original priority of $s_{i}^{k}$. Under
G-EDF/LCM, $\tau_{j}\ne\tau_{i}$ can have at least one job of higher
priority than current job of $\tau_{i}$, then $\left\lceil \frac{T_{i}}{T_{j}}\right\rceil \ge1$.
So, each one of the $s_{iz}^{k}$ in the left hand side of \eqref{eq:fblt_lcm_edf_comparison_3}
is included in one of the $\bar{s_{j}^{h}}(\theta)$ in the right
hand side of \eqref{eq:fblt_lcm_edf_comparison_3}. $\therefore$
\eqref{eq:fblt_lcm_edf_comparison_3} holds if 
\begin{eqnarray}
 & \sum_{\forall\tau_{i}}\frac{\sum_{\forall s_{i}^{k}\in s_{i}}\delta_{i}^{k}(T_{i})len(s_{i}^{k})}{T_{i}}\label{eq:fblt_lcm_edf_comparison_4}\\
\le & \sum_{\forall\tau_{i}}\frac{\left(\sum_{\forall\tau_{j}\in\gamma_{i}}\sum_{\forall\bar{s_{j}^{h}(\theta)},\theta\in\theta_{i}}\left(\left\lceil \frac{T_{i}}{T_{j}}\right\rceil len\left(\bar{\alpha_{max}^{jh}}s_{max}^{j}(\theta)\right)\right)\right)}{T_{i}}\nonumber \\
+ & \sum_{\forall\tau_{i}}\frac{\sum_{\forall s_{i}^{k}}\left(1-\alpha_{max}^{ik}\right)len\left(s_{max}^{i}\right)}{T_{i}}\nonumber 
\end{eqnarray}
For each $s_{i}^{k}\in s_{i}$, there are a set of zero or more $\bar{s_{j}^{h}}(\theta)\in\tau_{j},\,\forall\tau_{j}\ne\tau_{i}$
that are conflicting with $s_{i}^{k}$. Assuming this set of conflicting
transactions with $s_{i}^{k}$ is denoted as $\eta_{i}^{k}=\left\{ \bar{s_{j}^{h}}(\theta)\in\tau_{j}:\left(\theta\in\theta_{i}\right)\wedge\left(\forall\tau_{j}\ne\tau_{i}\right)\wedge\left(\bar{s_{j}^{h}}(\theta)\not\in\eta_{i}^{l},\, l\ne k\right)\right\} $.
The last condition $\bar{s_{j}^{h}}(\theta)\not\in\eta_{i}^{l},\, l\ne k$
in definition of $\eta_{i}^{k}$ ensures that common transactions
$\bar{s_{j}^{h}}$ that can conflict with more than one transaction
$s_{i}^{k}\in\tau_{i}$ are split among different $\eta_{i}^{k},\, k=1,..,|s_{i}|$.
This condition is necessary because in G-EDF/LCM, no two or more transactions
of $\tau_{i}^{x}$ can be aborted by the same transaction of $\tau_{j}^{h}$
where $p_{j}^{h}>p_{i}^{x}$. By substitution of $\eta_{i}^{k}$ in
(\ref{eq:fblt_lcm_edf_comparison_4}), we get 
\begin{eqnarray}
 & \sum_{\forall\tau_{i}}\frac{\sum_{\forall s_{i}^{k}\in s_{i}}\delta_{i}^{k}(T_{i})len(s_{i}^{k})}{T_{i}}\label{eq:fblt_lcm_edf_comparison_5}\\
\le & \sum_{\forall\tau_{i}}\frac{\sum_{k=1}^{|s_{i}|}\sum_{\forall\bar{s_{j}^{h}}(\theta)\in\eta_{i}^{k}}\left(\left\lceil \frac{T_{i}}{T_{j}}\right\rceil len\left(\bar{\alpha_{max}^{jh}}s_{max}^{j}(\theta)\right)\right)}{T_{i}}\nonumber \\
+ & \sum_{\forall\tau_{i}}\frac{\left(\sum_{\forall s_{i}^{k}}\left(1-\alpha_{max}^{ik}\right)len\left(s_{max}^{i}\right)\right)}{T_{i}}\nonumber 
\end{eqnarray}
$\bar{s_{j}^{h}}$ belongs to higher priority jobs than $\tau_{i}$
and $s_{max}^{j}$ belongs to higher priority jobs than $\tau_{i}$
or $\tau_{i}$ itself. Transactions in $m\_$set can belong to jobs
with original priority higher or lower than $\tau_{i}$. So, (\ref{eq:fblt_lcm_edf_comparison_3})
holds if for each $s_{i}^{k}\in\tau_{i}$ 
\[
\delta_{i}^{k}(T_{i})len(s_{i}^{k})\le\left(\sum_{\forall\bar{s_{j}^{h}}(\theta)\in\eta_{i}^{k}}\left(\left\lceil \frac{T_{i}}{T_{j}}\right\rceil \right)len\left(\bar{\alpha_{max}^{jh}}s_{max}^{j}(\theta)\right)\right)+\left(1-\alpha_{max}^{ik}\right)len\left(s_{max}^{i}\right)
\]
\begin{equation}
\therefore\,\delta_{i}^{k}(T_{i})\le\left(\sum_{\forall\bar{s_{j}^{h}}(\theta)\in\eta_{i}^{k}}\left(\left\lceil \frac{T_{i}}{T_{j}}\right\rceil \right)len\left(\frac{\bar{\alpha_{max}^{jh}}s_{max}^{j}(\theta)}{s_{i}^{k}}\right)\right)+\left(1-\alpha_{max}^{ik}\right)len\left(\frac{s_{max}^{i}}{s_{i}^{k}}\right)\label{eq:fblt_lcm_edf_comparison_6}
\end{equation}
$\because\, len\left(\frac{s_{max}^{j}(\theta)}{s_{i}^{k}}\right)\ge1$,
$\therefore$ (\ref{eq:fblt_lcm_edf_comparison_6}) holds if 
\begin{equation*}
\therefore\,\delta_{i}^{k}(T_{i})\le\left(\sum_{\bar{s_{j}^{h}}(\theta)\in\eta_{i}^{k}}\left(\left\lceil \frac{T_{i}}{T_{j}}\right\rceil \bar{\alpha_{max}^{jh}}\right)\right)
\end{equation*}
Claim follows.

\end{proof}

\subsection{FBLT vs. G-RMA/LCM}

\begin{clm}\label{clm:fblt_lcm_rma}

Schedulability of FBLT is equal or better to G-RMA/LCM's when maximum
number of abort times of each transaction $s_{i}^{k}$ in any job
of $\tau_{i}$ is less or equal to sum of maximum length $m-1$ transactions
conflicting with $s_{i}^{k}$ subtracted from sum of total number
each transaction $s_{j}^{h}$ can conflict with $s_{i}^{k}$ multiplied by maximum $\alpha$ with which $s_{j}^{h}$ can conflict with maximum length transaction sharing objects with $s_{i}^{k}$
and $s_{j}^{h}$.

\end{clm}

\begin{proof}

By substituing $RC_{A}(T_{i})$ and $RC_{B}(T_{i})$ in (\ref{eq:utilization comparison})
with (\ref{eq:fblt_rc}) and (6.9) in \cite{shambake_phd_proposal}
respectively. 

\begin{eqnarray}
 & \sum_{\forall\tau_{i}}\frac{\sum_{\forall s_{i}^{k}\in s_{i}}\left(\delta_{i}^{k}(T_{i})len(s_{i}^{k})+\sum_{s_{iz}^{k}\in\chi_{i}^{k}}len(s_{iz}^{k})\right)+RC_{re}(T_{i})}{T_{i}}\label{eq:fblt_lcm_rma_comparison_1}\\
\le & \sum_{\forall\tau_{i}}\frac{\sum_{\forall\tau_{j}^{*}\in\gamma_{i}^{ex}}\sum_{\forall\theta\in\theta_{i}^{ex}}\left(\left\lceil \frac{T_{i}}{T_{j}}\right\rceil +1\right)\sum_{\forall\bar{s_{j}^{h}(\theta)}}len\left(\bar{s_{j}^{h}(\theta)}+\bar{\alpha_{max}^{jh}}s_{max}^{j}(\theta)\right)}{T_{i}}\nonumber \\
+ & \sum_{\forall\tau_{i}}\frac{\sum_{\forall s_{i}^{k}}\left(1-\alpha_{max}^{ik}\right)len(s_{max}^{i})+RC_{re}(T_{i})}{T_{i}}\nonumber 
\end{eqnarray}
where $\tau_{j}^{*}=\left\{ \tau_{j}:p_{j}>p_{i}\right\} $. Let $\theta_{i}^{ex}=\theta_{i}+\theta_{i}^{*}$
where $\theta_{i}^{*}$ is the set of objects not accessed directly
by $\tau_{i}$ but can enforce transactions in $\tau_{i}$ to retry
due to transitive retry. Let $\gamma_{i}^{ex}=\gamma_{i}+\gamma_{i}^{*}$where
$\gamma_{i}^{*}$ is the set of tasks that access objects in $\theta_{i}^{*}$.
$\bar{s_{j}^{h}}(\theta)$ can access multiple objects, so $s_{max}^{j}(\theta)$
is the maximum length transaction conflicting with $\bar{s_{j}^{h}}(\theta)$.
FBLT needs to bound transitive retry. So, $\theta_{i}^{ex}$ will
be avoided in (\ref{eq:fblt_lcm_rma_comparison_1}). Thus, transactions
under G-RMA/LCM behave as if there were no transitive retry. Consequently,
(\ref{eq:fblt_lcm_rma_comparison_1}) will be 
\begin{eqnarray}
 & \sum_{\forall\tau_{i}}\frac{\sum_{\forall s_{i}^{k}\in s_{i}}\left(\delta_{i}^{k}(T_{i})len(s_{i}^{k})+\sum_{s_{iz}^{k}\in\chi_{i}^{k}}len(s_{iz}^{k})\right)}{T_{i}}\label{eq:fblt_lcm_rma_comparison_2}\\
\le & \sum_{\forall\tau_{i}}\frac{\sum_{\forall\tau_{j}^{*}\in\gamma_{i}}\sum_{\forall\theta\in\theta_{i}}\left(\left\lceil \frac{T_{i}}{T_{j}}\right\rceil +1\right)\sum_{\forall\bar{s_{j}^{h}(\theta)}}len\left(\bar{s_{j}^{h}(\theta)}+\bar{\alpha_{max}^{jh}}s_{max}^{j}(\theta)\right)}{T_{i}}\nonumber \\
+ & \sum_{\forall\tau_{i}}\frac{\sum_{\forall s_{i}^{k}}\left(1-\alpha_{max}^{ik}\right)len(s_{max}^{i})}{T_{i}}\nonumber 
\end{eqnarray}
As $\bar{s_{j}^{h}}(\theta)$ is included only once for all objects
accessed by it. $s_{max}^{j}(\theta)$ is also included once for each
$\bar{s_{j}^{h}}(\theta)$. Consequently, (\ref{eq:fblt_lcm_rma_comparison_2})
becomes 
\begin{eqnarray}
 & \sum_{\forall\tau_{i}}\frac{\sum_{\forall s_{i}^{k}\in s_{i}}\left(\delta_{i}^{k}(T_{i})len(s_{i}^{k})+\sum_{s_{iz}^{k}\in\chi_{i}^{k}}len(s_{iz}^{k})\right)}{T_{i}}\label{eq:fblt_lcm_rma_comparison_3}\\
\le & \sum_{\forall\tau_{i}}\frac{\left(\sum_{\forall\tau_{j}^{*}\in\gamma_{i}}\sum_{\bar{s_{j}^{h}}(\theta),\,\theta\in\theta_{i}}\left(\left\lceil \frac{T_{i}}{T_{j}}\right\rceil +1\right)len\left(\bar{s_{j}^{h}}(\theta)+\bar{\alpha_{max}^{jh}}s_{max}^{j}(\theta)\right)\right)}{T_{i}}\nonumber \\
+ & \sum_{\forall\tau_{i}}\frac{\sum_{\forall s_{i}^{k}}\left(1-\alpha_{max}^{ik}\right)len(s_{max}^{i})}{T_{i}}\nonumber 
\end{eqnarray}
Although different $s_{i}^{k}$ can have common conflicting transactions
$\bar{s_{j}^{h}}$, but no more than one $s_{i}^{k}$ can be preceded
by the same $\bar{s_{j}^{h}}$ in the $m\_$set. This happens because
transactions in the $m\_$set are non-preemptive. Original priority
of transactions preceding $s_{i}^{k}$ in the $m\_$set can be of
lower or higher priority than original priority of $s_{i}^{k}$. Under
G-RMA, $p_{j}>p_{i}$ means that $T_{j}\le T_{i}$, then $\left\lceil \frac{T_{i}}{T_{j}}\right\rceil \ge1$.
For each $s_{i}^{k}\in s_{i}$, there are a set of zero or more $\bar{s_{j}^{h}}(\theta)\in\tau_{j}^{*}$
that are conflicting with $s_{i}^{k}$. Assuming this set of conflicting
transactions with $s_{i}^{k}$ is denoted as $\eta_{i}^{k}=\left\{ \bar{s_{j}^{h}}(\theta)\in\tau_{j}^{*}:\left(\theta\in\theta_{i}\right)\wedge\left(\bar{s_{j}^{h}}(\theta)\not\in\eta_{i}^{l},\, l\ne k\right)\right\} $.
The last condition $\bar{s_{j}^{h}}(\theta)\not\in\eta_{i}^{l},\, l\ne k$
in definition of $\eta_{i}^{k}$ ensures that common transactions
$\bar{s_{j}^{h}}$ that can conflict with more than one transaction
$s_{i}^{k}\in\tau_{i}$ are split among different $\eta_{i}^{k},\, k=1,..,|s_{i}|$.
This condition is necessary because in G-RMA/LCM, no two or more transactions
of $\tau_{i}^{x}$ can be aborted by the same transaction of $\tau_{j}^{h}$
where $p_{j}^{h}>p_{i}^{x}$. By substitution of $\eta_{i}^{k}$ in
(\ref{eq:fblt_lcm_rma_comparison_3}), we get 
\begin{eqnarray}
 & \sum_{\forall\tau_{i}}\frac{\sum_{\forall s_{i}^{k}\in s_{i}}\left(\delta_{i}^{k}(T_{i})len(s_{i}^{k})+\sum_{s_{iz}^{k}\in\chi_{i}^{k}}len(s_{iz}^{k})\right)}{T_{i}}\label{eq:fblt_lcm_rma_comparison_4}\\
\le & \sum_{\forall\tau_{i}}\frac{\left(\sum_{\forall k=1}^{|s_{i}|}\sum_{\bar{s_{j}^{h}}(\theta)\in\eta_{i}^{k}}\left(\left\lceil \frac{T_{i}}{T_{j}}\right\rceil +1\right)len\left(\bar{s_{j}^{h}}(\theta)+\bar{\alpha_{max}^{jh}}s_{max}^{j}(\theta)\right)\right)}{T_{i}}\nonumber \\
+ & \sum_{\forall\tau_{i}}\frac{\sum_{\forall s_{i}^{k}}\left(1-\alpha_{max}^{ik}\right)len(s_{max}^{i})}{T_{i}}\nonumber 
\end{eqnarray}


$\bar{s_{j}^{h}}$ belongs to higher priority jobs than $\tau_{i}$
and $s_{max}^{j}$ belongs to higher priority jobs than $\tau_{i}$
or $\tau_{i}$ itself. Transactions in $m\_$set can belong to jobs
with original priority higher or lower than $\tau_{i}$. So, (\ref{eq:fblt_lcm_rma_comparison_4})
holds if for each $s_{i}^{k}\in\tau_{i}$ 
\begin{eqnarray*}
\delta_{i}^{k}(T_{i})len(s_{i}^{k}) & \le & \left(\sum_{\bar{s_{j}^{h}}(\theta)\in\eta_{i}^{k}}\left(\left(\left\lceil \frac{T_{i}}{T_{j}}\right\rceil +1\right)len\left(\bar{s_{j}^{h}}(\theta)+\bar{\alpha_{max}^{jh}}s_{max}^{j}(\theta)\right)\right)\right)\\
 & + & \left(1-\alpha_{max}^{ik}\right)len(s_{max}^{i})-\sum_{s_{iz}^{k}\in\chi_{i}^{k}}len(s_{iz}^{k})
\end{eqnarray*}
\begin{eqnarray}
\therefore\,\delta_{i}^{k}(T_{i}) & \le & \left(\sum_{\bar{s_{j}^{h}}(\theta)\in\eta_{i}^{k}}\left(\left(\left\lceil \frac{T_{i}}{T_{j}}\right\rceil +1\right)len\left(\frac{\bar{s_{j}^{h}}(\theta)+\bar{\alpha_{max}^{jh}}s_{max}^{j}(\theta)}{s_{i}^{k}}\right)\right)\right)\nonumber \\
 & + & \left(1-\alpha_{max}^{ik}\right)len(\frac{s_{max}^{i}}{s_{i}^{k}})-\sum_{s_{iz}^{k}\in\chi_{i}^{k}}len\left(\frac{s_{iz}^{k}}{s_{i}^{k}}\right)\label{eq:fblt_lcm_rma_comparison_5}
\end{eqnarray}
$\because\, len\left(\frac{s_{iz}^{k}}{s_{i}^{k}}\right)\le len(s_{iz}^{k})$,
$\bar{\alpha_{max}^{jh}}\le1$ and $s_{max}^{i}$ can be greater,
less or equal to $s_{i}^{k}\,\therefore$ (\ref{eq:fblt_lcm_rma_comparison_5})
holds if 
\begin{equation}
\therefore\,\delta_{i}^{k}(T_{i})\le\left(\sum_{\bar{s_{j}^{h}}(\theta)\in\eta_{i}^{k}}\left(\left\lceil \frac{T_{i}}{T_{j}}\right\rceil +1\right)\bar{\alpha_{max}^{jh}}len\left(\frac{\bar{s_{j}^{h}}(\theta)+s_{max}^{j}(\theta)}{s_{i}^{k}}\right)\right)-\sum_{s_{iz}^{k}\in\chi_{i}^{k}}len\left(s_{iz}^{k}\right)\label{eq:fblt_lcm_rma_comparison_6}
\end{equation}
$\because\, len(\frac{s_{max}^{j}(\theta)}{s_{i}^{k}})\ge1,\,\therefore$
(\ref{eq:fblt_lcm_rma_comparison_6}) holds if 
\begin{equation*}
\therefore\,\delta_{i}^{k}(T_{i})\le\left(\sum_{\bar{s_{j}^{h}}(\theta)\in\eta_{i}^{k}}\left(\left\lceil \frac{T_{i}}{T_{j}}\right\rceil +1\right)\bar{\alpha_{max}^{jh}}\right)-\sum_{s_{iz}^{k}\in\chi_{i}^{k}}len\left(s_{iz}^{k}\right)
\end{equation*}
Claim follows.

\end{proof}

\subsection{FBLT vs. PNF}

\begin{clm}\label{clm:fblt_pnf_edf}

Schedulability of FBLT is equal or better to PNF's with G-EDF and
G-RMA if for any $\tau_{i}^{x}$ the following conditions are satisfied:
\begin{enumerate}
\item Total sum of transactional lengths of all transactions conflicting
with the maximum length transaction in $\tau_{i}^{x}$, $s_{i_{max}}$,
is no less than maximum retry cost of $\tau_{i}^{x}$ due to release
of higher priority jobs.
\item Maximum abort number of $s_{i_{max}}$ is less or equal to difference
between total sum of transactional lengths of all transactions conflicting
with $s_{i_{max}}$ and maximum retry cost of $\tau_{i}^{x}$ due
to release of higher priority jobs divided by length of $s_{i_{max}}$.
\item Maximum abort number of any $s_{i}^{k}$, other than $s_{i_{max}}$,
should be less or equal to total sum of transactional lengths of all
conflicting transactions with $s_{i}^{k}$ divided by length of $s_{i}^{k}$.
\end{enumerate}
\end{clm}

\begin{proof}

By substituting $RC_{A}(T_{i})$ and $RC_{B}(T_{i})$ in (\ref{eq:utilization comparison})
with (\ref{eq:fblt_rc}) and (6.1) in \cite{shambake_phd_proposal}
respectively. 

\begin{eqnarray}
 & \sum_{\forall\tau_{i}}\frac{\sum_{\forall s_{i}^{k}\in s_{i}}\left(\delta_{i}^{k}(T_{i})len(s_{i}^{k})+\sum_{s_{iz}^{k}\in\chi_{i}^{k}}len(s_{iz}^{k})\right)+RC_{re}(T_{i})}{T_{i}}\label{eq:fblt_pnf_comparison_1}\\
\le & \sum_{\forall\tau_{i}}\frac{\sum_{\forall\tau_{j}\in\gamma_{i}}\sum_{\theta\in\theta_{i}}\left(\left(\left\lceil \frac{T_{i}}{T_{j}}\right\rceil +1\right)\sum_{\forall\bar{s_{j}^{h}}(\theta)}len\left(\bar{s_{j}^{h}}(\theta)\right)\right)}{T_{i}}\nonumber 
\end{eqnarray}
$\bar{s_{j}^{h}}(\theta)$ can access multiple objects, so $s_{max}^{j}(\theta)$
is the maximum length transaction conflicting with $\bar{s_{j}^{h}}(\theta)$.
As $\bar{s_{j}^{h}}(\theta)$ is included only once for all objects
accessed by it. $s_{max}^{j}(\theta)$ is also included once for each
$\bar{s_{j}^{h}}(\theta)$. Consequently, \ref{eq:fblt_pnf_comparison_1}
becomes 
\begin{eqnarray}
 & \sum_{\forall\tau_{i}}\frac{\sum_{\forall s_{i}^{k}\in s_{i}}\left(\delta_{i}^{k}(T_{i})len(s_{i}^{k})+\sum_{s_{iz}^{k}\in\chi_{i}^{k}}len(s_{iz}^{k})\right)+RC_{re}(T_{i})}{T_{i}}\label{eq:fblt_pnf_comparison_2}\\
\le & \sum_{\forall\tau_{i}}\frac{\sum_{\forall\tau_{j}\in\gamma_{i}}\sum_{\bar{s_{j}^{h}}(\theta),\,\theta\in\theta_{i}}\left(\left(\left\lceil \frac{T_{i}}{T_{j}}\right\rceil +1\right)len\left(\bar{s_{j}^{h}}(\theta)\right)\right)}{T_{i}}\nonumber 
\end{eqnarray}
$RC_{re}(T_{i})$ is given by (6.8) in \cite{shambake_phd_proposal}
in case of G-EDF, and (6.10) in \cite{shambake_phd_proposal} in case
of G-RMA. Substituting $RC_{re}(T_{i})=\sum_{\forall\tau_{j}\in\gamma_{i}}\left\lceil \frac{T_{i}}{T_{j}}\right\rceil s_{i_{max}}$,
covers $RC_{re}(T_{i})$ given by (6.8) and (6.10) in \cite{shambake_phd_proposal}
and maintains validity of \ref{eq:fblt_pnf_comparison_2}. If $\tau_{j}$
has no shared objects with $\tau_{i}$, then release of any higher
priority job $\tau_{j}^{y}$ will not abort any transaction in any
job of $\tau_{i}$. $\therefore$ \ref{eq:fblt_pnf_comparison_2}
holds if 
\begin{eqnarray}
 & \sum_{\forall\tau_{i}}\frac{\sum_{\forall s_{i}^{k}\in s_{i}}\left(\delta_{i}^{k}(T_{i})len(s_{i}^{k})+\sum_{s_{iz}^{k}\in\chi_{i}^{k}}len(s_{iz}^{k})\right)+\left(\sum_{\forall\tau_{j}\in\gamma_{i}}\left\lceil \frac{T_{i}}{T_{j}}\right\rceil s_{i_{max}}\right)}{T_{i}}\label{eq:fblt_pnf_comparison_9}\\
\le & \sum_{\forall\tau_{i}}\frac{\left(\sum_{\forall\tau_{j}\in\gamma_{i}}\sum_{\bar{s_{j}^{h}}(\theta),\,\theta\in\theta_{i}}\left(\left(\left\lceil \frac{T_{i}}{T_{j}}\right\rceil +1\right)len\left(\bar{s_{j}^{h}}(\theta)\right)\right)\right)}{T_{i}}\nonumber 
\end{eqnarray}


Although different $s_{i}^{k}$ can have common conflicting transactions
$\bar{s_{j}^{h}}$, but no more than one $s_{i}^{k}$ can be preceded
by the same $\bar{s_{j}^{h}}$ in the $m\_$set. This happens because
transactions in the $m\_$set are non-preemptive. Original priority
of transactions preceding $s_{i}^{k}$ in the $m\_$set can be of
lower or higher priority than original priority of $s_{i}^{k}$. Under
PNF, $\tau_{j}^{y}$ can have a priority higher or lower priority
than $\tau_{i}^{x}$, still transactions in $\tau_{j}^{y}$ can abort
transactions in $\tau_{i}^{x}$. So, each one of the $s_{iz}^{k}$
in the left hand side of \ref{eq:fblt_pnf_comparison_9} is included
in one of the $\bar{s_{j}^{h}}(\theta)$ in the right hand side of
\ref{eq:fblt_pnf_comparison_9}. $\therefore$ \ref{eq:fblt_pnf_comparison_9}
holds if 
\begin{eqnarray}
 & \sum_{\forall\tau_{i}}\frac{\sum_{\forall s_{i}^{k}\in s_{i}}\left(\delta_{i}^{k}(T_{i})len(s_{i}^{k})\right)+\left(\sum_{\forall\tau_{j}\in\gamma_{i}}\left\lceil \frac{T_{i}}{T_{j}}\right\rceil s_{i_{max}}\right)}{T_{i}}\label{eq:fblt_pnf_comparison_3}\\
\le & \sum_{\forall\tau_{i}}\frac{\left(\sum_{\forall\tau_{j}\in\gamma_{i}}\sum_{\bar{s_{j}^{h}}(\theta),\,\theta\in\theta_{i}}\left(\left\lceil \frac{T_{i}}{T_{j}}\right\rceil len\left(\bar{s_{j}^{h}}(\theta)\right)\right)\right)}{T_{i}}\nonumber 
\end{eqnarray}
One of the $s_{i}^{k}$ is $s_{i_{max}}$, so \ref{eq:fblt_pnf_comparison_3}
becomes 
\begin{eqnarray}
 & \sum_{\forall\tau_{i}}\frac{\sum_{\forall s_{i}^{k}\in s_{i},\, s_{i}^{k}\ne s_{i_{max}}}\left(\delta_{i}^{k}(T_{i})len(s_{i}^{k})\right)+\left(\left(\delta_{i_{max}}+\sum_{\forall\tau_{j}\in\gamma_{i}}\left\lceil \frac{T_{i}}{T_{j}}\right\rceil \right)s_{i_{max}}\right)}{T_{i}}\label{eq:fblt_pnf_comparison_10}\\
\le & \sum_{\forall\tau_{i}}\frac{\left(\sum_{\forall\tau_{j}\in\gamma_{i}}\sum_{\bar{s_{j}^{h}}(\theta),\,\theta\in\theta_{i}}\left(\left\lceil \frac{T_{i}}{T_{j}}\right\rceil len\left(\bar{s_{j}^{h}}(\theta)\right)\right)\right)}{T_{i}}\nonumber 
\end{eqnarray}


For each $s_{i}^{k}\in s_{i}$ including $s_{i_{max}}$, there are
a set of one or more $\bar{s_{j}^{h}}(\theta)\in\tau_{j},\,\forall\tau_{j}\in\gamma_{i}$
that are conflicting with $s_{i}^{k}$. Assuming this set of conflicting
transactions with $s_{i}^{k}$ is denoted as $\eta_{i}^{k}=\left\{ \bar{s_{j}^{h}}(\theta)\in\tau_{j}:\left(\theta\in\theta_{i}\right)\wedge\left(\forall\tau_{j}\in\gamma_{i}\right)\wedge\left(\bar{s_{j}^{h}}(\theta)\not\in\eta_{i}^{l},\, l\ne k\right)\right\} $.
The last condition $\bar{s_{j}^{h}}(\theta)\not\in\eta_{i}^{l},\, l\ne k$
in definition of $\eta_{i}^{k}$ ensures that common transactions
$\bar{s_{j}^{h}}$ that can conflict with more than one transaction
$s_{i}^{k}\in\tau_{i}$ are split among different $\eta_{i}^{k},\, k=1,..,|s_{i}|$.
This condition is necessary because in PNF, no two or more transactions
of $\tau_{i}^{x}$ can be aborted by the same transaction of $\tau_{j}^{h}$
whether $p_{j}^{h}>p_{i}^{x}$ or not. By substitution of $\eta_{i}^{k}$
in \ref{eq:fblt_pnf_comparison_3}, we get 
\begin{eqnarray}
 & \sum_{\forall\tau_{i}}\frac{\left(\sum_{\forall s_{i}^{k}\in s_{i},\, s_{i}^{k}\ne s_{i_{max}}}\left(\delta_{i}^{k}(T_{i})len(s_{i}^{k})\right)\right)+\left(\left(\delta_{i_{max}}(T_{i})+\sum_{\forall\tau_{j}\in\gamma_{i}}\left\lceil \frac{T_{i}}{T_{j}}\right\rceil \right)s_{i_{max}}\right)}{T_{i}}\label{eq:fblt_pnf_comparison_5}\\
\le & \sum_{\forall\tau_{i}}\frac{\left(\sum_{\forall k=1}^{|s_{i}|}\sum_{\bar{s_{j}^{h}}(\theta)\in\eta_{i}^{k}}\left(\left\lceil \frac{T_{i}}{T_{j}}\right\rceil len\left(\bar{s_{j}^{h}}(\theta)\right)\right)\right)}{T_{i}}\nonumber 
\end{eqnarray}
Since $s_{i_{max}}\in s_{i}$, $\therefore$ (\ref{eq:fblt_pnf_comparison_5})
holds if the following two conditions hold for each $\tau_{i}$ :
\begin{enumerate}
\item $\delta_{i_{max}}\le\frac{\left(\sum_{\bar{s_{j}^{h}}(\theta)\in\eta_{i_{max}}}\left\lceil \frac{T_{i}}{T_{j}}\right\rceil len\left(\bar{s_{j}^{h}}(\theta)\right)\right)-\left(\sum_{\forall\tau_{j}\in\gamma_{i}}\left\lceil \frac{T_{i}}{T_{j}}\right\rceil s_{i_{max}}\right)}{s_{i_{max}}}$,
where $\eta_{i_{max}}$ is one of the $\eta_{i}^{k}$s that corresponds
to $s_{i_{max}}$. $\because\,\delta_{i_{max}}\ge0,\,\therefore\,\left(\sum_{\bar{s_{j}^{h}}(\theta)\in\eta_{i_{max}}}\left\lceil \frac{T_{i}}{T_{j}}\right\rceil len\left(\bar{s_{j}^{h}}(\theta)\right)\right)\ge\left(\sum_{\forall\tau_{j}\in\gamma_{i}}\left\lceil \frac{T_{i}}{T_{j}}\right\rceil s_{i_{max}}\right)$.
\item For each $s_{i}^{k}$ other than $s_{i_{max}}$, $\delta_{i}^{k}\le\sum_{\bar{s_{j}^{h}}(\theta)\in\eta_{i}^{k}}\left(\left\lceil \frac{T_{i}}{T_{j}}\right\rceil len\left(\frac{\bar{s_{j}^{h}}(\theta)}{s_{i}^{k}}\right)\right)$.
\end{enumerate}
Claim follows.

\end{proof}

\subsection{FBLT vs. Lock-free}

\begin{clm}\label{clm:fblt_edf_lock-free}

Under G-EDF and G-RMA, schedulability of FBLT is equal or better than
lock-free's if $s_{max}\le r_{max}$. If transactions execute in FIFO
order (i. e., $\delta_{i}^{k}=0,\,\forall s_{i}^{k}$) and contention
is high, $s_{max}$ can be much larger than $r_{max}$.

\end{clm}

\begin{proof}

Lock-free synchronization \cite{key-5,stmconcurrencycontrol:emsoft11}
accesses only one object. Thus, the number of accessed objects per
transaction in FBLT is limited to one. This allows us to compare the
schedulability of FBLT with the lock-free algorithm.

By substituting $RC_{A}(T_{i})$ and $RC_{B}(T_{i})$ in (\ref{eq:utilization comparison})
with (\ref{eq:fblt_rc}) and (6.17) in \cite{shambake_phd_proposal}
respectively. 

\begin{eqnarray}
 & \sum_{\forall\tau_{i}}\frac{\sum_{\forall s_{i}^{k}\in s_{i}}\left(\delta_{i}^{k}(T_{i})len(s_{i}^{k})+\sum_{s_{iz}^{k}\in\chi_{i}^{k}}len(s_{iz}^{k})\right)+RC_{re}(T_{i})}{T_{i}}\label{eq:fblt_lf_comparison_1}\\
\le & \sum_{\forall\tau_{i}}\frac{\left(\sum_{\forall\tau_{j}\in\gamma_{i}}\left(\left(\left\lceil \frac{T_{i}}{T_{j}}\right\rceil +1\right)\beta_{i,j}r_{max}\right)\right)+RC_{re}(T_{i})}{T_{i}}\nonumber 
\end{eqnarray}
where $\beta_{i,j}$ is the number of retry loops of $\tau_{j}$ that
access the same objects as accessed by any retry loop of $\tau_{i}$
\cite{key-5}. $r_{max}$ is the maximum execution cost of a single
iteration of any retry loop of any task \cite{key-5}. For G-EDF(G-RMA),
any job $\tau_{i}^{x}$ under FBLT has the same pattern of interference
from higher priority jobs as ECM(RCM) respectively. $RC_{re}(T_{i})$
for ECM, RCM and lock-free are given by Claims 25, 26 and 27 in \cite{shambake_phd_proposal}
respectively.$\therefore$ $RC_{re}(T_{i})=\left\lceil \frac{T_{i}}{T_{j}}\right\rceil s_{i_{max}},\,\forall\tau_{j}\in\gamma_{i}$
covers $RC_{re}(T_{i})$ for G-EDF/FBLT and G-RMA/FBLT. $RC_{re}(T_{i})=\left\lceil \frac{T_{i}}{T_{j}}\right\rceil r_{i_{max}},\,\forall\tau_{j}\in\gamma_{i}$
covers retry cost for G-EDF/lock-free and G-RMA/lock-free. $\therefore$
(\ref{eq:fblt_lf_comparison_1}) becomes 
\begin{eqnarray}
 & \sum_{\forall\tau_{i}}\frac{\sum_{\forall s_{i}^{k}\in s_{i}}\left(\delta_{i}^{k}(T_{i})len(s_{i}^{k})+\sum_{s_{iz}^{k}\in\chi_{i}^{k}}len(s_{iz}^{k})\right)+\sum_{\tau_{j}\in\gamma_{i}}\left\lceil \frac{T_{i}}{T_{j}}\right\rceil s_{i_{max}}}{T_{i}}\label{eq:fblt_lf_comparison_8}\\
\le & \sum_{\forall\tau_{i}}\frac{\left(\sum_{\forall\tau_{j}\in\gamma_{i}}\left(\left(\left\lceil \frac{T_{i}}{T_{j}}\right\rceil +1\right)\beta_{i,j}r_{max}\right)\right)+\sum_{\tau_{j}\in\gamma_{i}}\left\lceil \frac{T_{i}}{T_{j}}\right\rceil r_{i_{max}}}{T_{i}}\nonumber 
\end{eqnarray}
Since $s_{max}\ge s_{i_{max}},\, len(s_{i}^{k}),\, len(s_{iz}^{k}),\,\forall i,z,k$
and $r_{max}\ge r_{i_{max}}$ $\therefore$ (\ref{eq:fblt_lf_comparison_8})
holds if 
\begin{eqnarray}
 & \sum_{\forall\tau_{i}}\frac{\left(\left(\sum_{\forall s_{i}^{k}\in s_{i}}\left(\delta_{i}^{k}(T_{i})+\sum_{s_{iz}^{k}\in\chi_{i}^{k}}1\right)\right)+\sum_{\tau_{j}\in\gamma_{i}}\left\lceil \frac{T_{i}}{T_{j}}\right\rceil \right)s_{max}}{T_{i}}\label{eq:fblt_lf_comparison_6}\\
\le & \sum_{\forall\tau_{i}}\frac{\left(\left(\sum_{\forall\tau_{j}\in\gamma_{i}}\left(\left(\left\lceil \frac{T_{i}}{T_{j}}\right\rceil +1\right)\beta_{i,j}\right)\right)+\sum_{\tau_{j}\in\gamma_{i}}\left\lceil \frac{T_{i}}{T_{j}}\right\rceil \right)r_{max}}{T_{i}}\nonumber 
\end{eqnarray}
$\therefore$ (\ref{eq:fblt_lf_comparison_6}) holds if for each $\tau_{i}$
\begin{eqnarray}
 & \left(\left(\sum_{\forall s_{i}^{k}\in s_{i}}\left(\delta_{i}^{k}(T_{i})+\sum_{s_{iz}^{k}\in\chi_{i}^{k}}1\right)\right)+\sum_{\tau_{j}\in\gamma_{i}}\left\lceil \frac{T_{i}}{T_{j}}\right\rceil \right)s_{max}\label{eq:fblt_lf_comparison_7}\\
\le & \left(\left(\sum_{\forall\tau_{j}\in\gamma_{i}}\left(\left(\left\lceil \frac{T_{i}}{T_{j}}\right\rceil +1\right)\beta_{i,j}\right)\right)+\sum_{\tau_{j}\in\gamma_{i}}\left\lceil \frac{T_{i}}{T_{j}}\right\rceil \right)r_{max}\nonumber 
\end{eqnarray}
$\therefore$
\begin{eqnarray}
\frac{s_{max}}{r_{max}} & \le & \frac{\left(\sum_{\forall\tau_{j}\in\gamma_{i}}\left(\left(\left\lceil \frac{T_{i}}{T_{j}}\right\rceil +1\right)\beta_{i,j}\right)\right)+\sum_{\tau_{j}\in\gamma_{i}}\left\lceil \frac{T_{i}}{T_{j}}\right\rceil }{\left(\sum_{\forall s_{i}^{k}\in s_{i}}\left(\delta_{i}^{k}(T_{i})+\sum_{s_{iz}^{k}\in\chi_{i}^{k}}1\right)\right)+\sum_{\tau_{j}\in\gamma_{i}}\left\lceil \frac{T_{i}}{T_{j}}\right\rceil }\label{eq:fblt_lf_comparison_9}
\end{eqnarray}
It appears from (\ref{eq:fblt_lf_comparison_9}) that as $\delta_{i}^{k}$,
as well as $|\chi_{i}^{k}|$, increases, then $s_{max}/r_{max}$ decreases.
So, to get the lower bound on $s_{max}/r_{max}$, let $\sum_{\forall s_{i}^{k}\in s_{i}}\left(\delta_{i}^{k}(T_{i})+\sum_{s_{iz}^{k}\in\chi_{i}^{k}}1\right)$
reaches its maximum value. This maximum value is the total number
of interfering transactions belonging to any job $\tau_{j}^{l},\, j\ne i$.
Priority of $\tau_{j}^{l}$ can be higher or lower than current instance
of $\tau_{i}$. Beyond this maximum value, higher values for any $\delta_{i}^{k}$
are ineffective as there will be no more transactions to conflict
with $s_{i}^{k}$. $\therefore\,\sum_{\forall s_{i}^{k}\in s_{i}}\left(\delta_{i}^{k}(T_{i})+\sum_{s_{iz}^{k}\in\chi_{i}^{k}}1\right)\le\sum_{\forall\tau_{j}\in\gamma_{i}}\left(\left\lceil \frac{T_{i}}{T_{j}}\right\rceil +1\right)$.
Consequently, (\ref{eq:fblt_lf_comparison_9}) will be 
\begin{eqnarray}
\frac{s_{max}}{r_{max}} & \le & \frac{\left(\sum_{\forall\tau_{j}\in\gamma_{i}}\left(\left(\left\lceil \frac{T_{i}}{T_{j}}\right\rceil +1\right)\beta_{i,j}\right)\right)+\sum_{\tau_{j}\in\zeta_{i}}\left\lfloor \frac{T_{i}}{T_{j}}\right\rfloor }{\left(\sum_{\forall\tau_{j}\in\gamma_{i}}\left(\left\lceil \frac{T_{i}}{T_{j}}\right\rceil +1\right)\right)+\sum_{\tau_{j}\in\zeta_{i}}\left\lfloor \frac{T_{i}}{T_{j}}\right\rfloor }\label{eq:fblt_lf_comparison_10}
\end{eqnarray}
As we look for the lower bound on $\frac{s_{max}}{r_{max}}$, let
$\beta_{i,j}$ assumes its minimum value. So, $\beta_{i,j}=1$. $\therefore$
(\ref{eq:fblt_lf_comparison_10}) holds if $\frac{s_{max}}{r_{max}}\le1$.

Let $\delta_{i}^{k}(T_{i})\rightarrow0$ in (\ref{eq:fblt_lf_comparison_9}).
This means transactions approximately execute in their arrival order.
Let $\beta_{i,j}\rightarrow\infty,\,\left\lceil \frac{T_{i}}{T_{j}}\right\rceil \rightarrow\infty$
in (\ref{eq:fblt_lf_comparison_9}) . This means contention is high.
Consequently, $\frac{s_{max}}{r_{max}}\rightarrow\infty$. So, if
transactions execute in FIFO order and contention is high, $s_{max}$
can be much larger than $r_{max}$. Claim follows.

\end{proof}

\bibliographystyle{plain}
\bibliography{/e/lectures/real-time/PhD-work/STM/writing/global_bibliography/global_bibliography}

\end{document}