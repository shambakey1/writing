\documentclass[conference]{IEEEtran}
%\documentclass{acm_proc_article-sp}

\makeatletter
\newif\if@restonecol

\let\algorithm\relax
\let\endalgorithm\relax
\usepackage[tight,footnotesize]{subfigure}
\usepackage {paralist}
\usepackage{comment}
\usepackage{array}
\usepackage{amsmath}
\usepackage{graphicx}
\usepackage{amssymb}
\usepackage{cite}
\usepackage{url}
\makeatother
\makeatletter
\providecommand{\tabularnewline}{\\}


\newtheorem{clm}{Claim}
\begin{document}

\title{Real-Time Length-based Contention Management for STM}


%\author{Mohammed Elshambakey
%\affil{ECE Dept., Virginia Tech, Blacksburg, VA 24060, USA}
%Binoy Ravindran
%\affil{ECE Dept., Virginia Tech, Blacksburg, VA 24060, USA}
%}

\begin{comment}
\author{\IEEEauthorblockN{Mohammed Elshambakey}
\IEEEauthorblockA{ECE Dept, Virginia Tech\\
Blacksburg, VA 24061, USA\\
Email: shambake@vt.edu}
\and
\IEEEauthorblockN{Binoy Ravindran}
\IEEEauthorblockA{ECE Dept, Virginia Tech\\
Blacksburg, VA 24061, USA\\
Email: binoy@vt.edu}
}
\end{comment}


\maketitle


\begin{clm}
\label{priority_inversion}
A higher priority job, $\tau_i^z$, suffers from priority inversion for at most the number of atomic sections in $\tau_i^z$.
\end{clm}
\begin{proof}
Consider three atomic sections, $s_i^k(\theta)$, $s_j^l(\theta)$, and $s_a^b(\theta)$, where $p_j > p_i$. Assume that $s_j^l(\theta)$ interferes with $s_i^k(\theta)$ after $\alpha_{ij}^{kl}$. Then $s_j^l(\theta)$ will have to abort and retry. At this time, let $s_a^b(\theta)$ interfere with the other two atomic sections. Now, the LCM will decide on which transaction to commit by comparing the two transactions. 
%%BR: Identify the 2 transactions: ".....by comparing the two transactions X and Y."
We have the following cases:
\begin{itemize}
\item $p_a < p_i < p_j$. For this case, $s_a^b(\theta)$ will not abort any one, because it is still in its beginning and has the lowest priority. So $\tau_j$ is not indirectly blocked by $\tau_a$.
%
\item $p_i<p_a<p_j$. Now, even if $s_a^b(\theta)$ interferes with $s_i^k(\theta)$ before $\alpha_{ia}^{kb}$, $s_a^b(\theta)$ will abort $s_i^k(\theta)$. After comparing $s_j^l(\theta)$ and $s_a^b(\theta)$, the LCM will choose $s_j^l(\theta)$ to commit and abort $s_a^b(\theta)$, because the latter is still at its beginning, and $\tau_j$ is of higher priority. If $s_a^b(\theta)$ is not allowed to abort $s_i^k(\theta)$, the situation is still the same, because $s_j^l(\theta)$ was already retrying until $s_i^k(\theta)$ finishes.
%
\item $p_a>p_j>p_i$. Now, if $s_a^b(\theta)$ is chosen to commit, there is no priority inversion for $\tau_j$,  because $\tau_a$ is of higher priority.
\item if $\tau_a$ preempts $\tau_i$ ($\tau_i$ is of lowest priority), then the LCM will compare only between $s_j^l(\theta)$ and $s_a^b(\theta)$. If $p_a<p_j$, then $s_j^l(\theta)$ will commit, because of its task's higher priority and $s_a^b(\theta)$ is still at its beginning. Otherwise, $s_j^l(\theta)$ will retry. However, this will not be priority inversion, because $\tau_a$ is already of higher priority than $\tau_j$.
\end{itemize}
By generalizing these cases to any number of conflicting jobs, it can be seen that when an atomic section  $s_j^l(\theta)$ of a higher priority job conflicts with a number of atomic sections belonging to lower priority jobs, $s_j^l(\theta)$ will suffer from priority inversion by only one of them. Thus, if each atomic section belonging to a higher priority job suffers from priority inversion, then the claim follows.
%%BR: Must reword the last sentence. Usually, you end a proof by saying ".... Thus, the claim follows." (and not by an if-then-else statement)."
\end{proof}

\begin{clm}
\label{max_pri_inv}
The maximum delay suffered by $s_j^l(\theta)$ due to priority inversion is caused by the minimum length atomic section accessing object $\theta$ that belongs to a lower priority job than $\tau_j^b$ that owns $s_j^l(\theta)$.
%%BR: I reworded. Check for accuracy.
\end{clm}
\begin{proof}
Consider three atomic sections, $s_i^k(\theta)$, $s_j^l(\theta)$, and $s_h^z(\theta)$, where $p_j>p_i$, $p_j>p_h$, and $len(s_i^k(\theta))>len(s_h^z(\theta))$. Now, $\alpha_{ij}^{kl}>\alpha_{hj}^{zl}$ and $c_{ij}^{kl}<c_{hj}^{zl}$. By applying~(\ref{eq48}) to obtain the contribution of $s_i^k(\theta)$ and $s_h^z(\theta)$ to the priority inversion of $s_j^l(\theta)$ and dividing them, we get:
\begin{eqnarray*}
\frac{W_{j}^{l}(s_{i}^{k}(\theta))}{W_{j}^{l}(s_{h}^{z}(\theta))} & = & \frac{\left(1-\alpha_{ij}^{kl}\right)len(s_{i}^{k}(\theta))}{\left(1-\alpha_{hj}^{zl}\right)len(s_{h}^{z}(\theta))}
\end{eqnarray*}
Substituting $\alpha$s from~(\ref{eq49}), we get:
\begin{eqnarray*}
 & = & \frac{(1-\frac{ln\psi}{ln\psi-c_{ij}^{kl}})len(s_{i}^{k}(\theta))}{(1-\frac{ln\psi}{ln\psi-c_{hj}^{zl}})len(s_{h}^{z}(\theta))}
  =  \frac{(\frac{-c_{ij}^{kl}}{ln\psi-c_{ij}^{kl}})len(s_{i}^{k}(\theta))}{(\frac{-c_{hj}^{zl}}{ln\psi-c_{hj}^{zl}})len(s_{h}^{z}(\theta))}\end{eqnarray*}
By substitution from~(\ref{cm_eq}):
\begin{eqnarray*}
 & = & \frac{len(s_{j}^{l}(\theta))/(ln\psi-c_{ij}^{kl})}{len(s_{j}^{l}(\theta))/(ln\psi-c_{hj}^{zl})}
  =  \frac{ln\psi-c_{hj}^{zl}}{ln\psi-c_{ij}^{kl}}<1\end{eqnarray*}
Thus, as the length of the interfered atomic section decreases, the effect of priority inversion on the interfering atomic section increases. 
%%BR: I reworded. Check for accuracy.
The claim follows.
\end{proof}

\bibliographystyle{IEEEtran}
\bibliography{IEEEabrv,global_bibliography}

\end{document}
