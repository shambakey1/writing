%
% PROJECT: <ETD> Electronic Thesis and Dissertation Initiative
%   TITLE: LaTeX report template for ETDs in LaTeX
%  AUTHOR: Neill Kipp, nkipp@vt.edu
%     URL: http://etd.vt.edu/latex/
% SAVE AS: etd.tex
% REVISED: September 6, 1997 [GMc 8/30/10]
% 

% Instructions: Remove the data from this document and replace it with your own,
% keeping the style and formatting information intact.  More instructions
% appear on the Web site listed above.

\documentclass[12pt,english]{report}

\makeatletter
\newif\if@restonecol

\let\algorithm\relax
\let\endalgorithm\relax
%\usepackage[tight,footnotesize]{subfigure}
\usepackage{subfigure}
\usepackage {paralist}
\usepackage{comment}
\usepackage{array}
\usepackage{amsmath}
\usepackage{graphicx}
\usepackage{amssymb}
\usepackage{cite}
\usepackage{url}
\usepackage[ruled]{algorithm2e}
\usepackage{babel}
\makeatother
\makeatletter
\providecommand{\tabularnewline}{\\}


\newtheorem{clm}{Claim}
\newtheorem{proof}{Proof}
\newtheorem{mydef}{Definition}


\makeatletter

%%%%%%%%%%%%%%%%%%%%%%%%%%%%%% LyX specific LaTeX commands.
\pdfpageheight\paperheight
\pdfpagewidth\paperwidth

%% Because html converters don't know tabularnewline
\providecommand{\tabularnewline}{\\}
%% A simple dot to overcome graphicx limitations
\newcommand{\lyxdot}{.}


%%%%%%%%%%%%%%%%%%%%%%%%%%%%%% User specified LaTeX commands.



\makeatother



\setlength{\textwidth}{6.5in}
\setlength{\textheight}{8.5in}
\setlength{\evensidemargin}{0in}
\setlength{\oddsidemargin}{0in}
\setlength{\topmargin}{0in}

\setlength{\parindent}{0pt}
\setlength{\parskip}{0.1in}

% Uncomment for double-spaced document.
% \renewcommand{\baselinestretch}{2}

% \usepackage{epsf}

\begin{document}

\thispagestyle{empty}
\pagenumbering{roman}
%\begin{center}



\pagenumbering{arabic}
\pagestyle{myheadings}

\chapter{\label{conclusions} Conclusions and Proposed Post Preliminary Exam Work}
\markright{Mohammed Elshambakey \hfill Chapter 9. Conclusions and Post  Preliminary Work \hfill}

\section{Conclusions}

In this dissertation, we designed, analyzed, and experimentally evaluated four real-time CMs. Designing real-time CMs is straightforward. The simplest logic is to use the same rationale as that of the underlying real-time scheduler. This was shown in the design of ECM and RCM. ECM allows the transaction with the earliest absolute deadline (i.e., dynamic priority) to commit first. RCM allows the transaction with the smallest period (i.e., fixed priority) to commit first. We established upper bounds for retry costs and response times under ECM and RCM, and identified the conditions under which they have better schedulability than lock-free synchronization. 

Under both ECM and RCM, a task incurs $2.s_{max}$ retry cost for each of its atomic sections due to a conflict with another task's atomic section. Retries under RCM and lock-free synchronization are affected by a larger number of conflicting task instances than under ECM. While task retries under ECM and lock-free are affected by all other tasks, retries under RCM are affected only by higher priority tasks. 

STM and lock-free synchronization have similar parameters that affect their retry costs -- i.e., the number of conflicting jobs and how many times they access shared objects. The $s_{max}/r_{max}$ ratio determines whether STM is better or as good as lock-free. For ECM, this ratio cannot exceed 1, and it can be 1/2 for higher number of conflicting tasks. For RCM, for the common case, $s_{max}$ must be 1/2 of $r_{max}$, and in some cases, $s_{max}$ can be larger than $r_{max}$ by many orders of magnitude.

LCM, which can be used with both G-EDF and G-RMA, tries to compromise between priority of transactions (which is the priority of the underlying tasks), and the remaining execution time of the interfered transaction. As the remaining execution time of the interfered transaction decreases, it is not effective to abort the transaction when it can shortly commit. The parameters $\alpha$ and $\psi$ are used to determine whether or not to abort the interfered transaction. $\alpha$ ranges between 0 and 1. When $\alpha \rightarrow 0$, LCM defaults to FCFS. When $\alpha\rightarrow1$, G-EDF/LCM defaults to ECM, and G-RMA/LCM defaults to RCM. We also derived upper bounds on retry costs and response times under LCM, and compared the schedulability of LCM with ECM, RCM, and lock-free synchronization. Also, we identified the conditions under which LCM performs better than the other synchronization techniques. LCM reduces the retry cost  of each atomic section to $(1+\alpha_{max})s_{max}$, instead of $2.s_{max}$ as in ECM and RCM. In ECM and RCM, tasks do not retry due to lower priority tasks, whereas in LCM, they do so. In G-EDF/LCM, retry due to a lower priority job is encountered only from a task $\tau_{j}$'s last job instance during $\tau_{i}$'s period. This is not the case with G-RMA/LCM, because,  each higher priority task can be aborted and retried by any job instance of lower priority tasks. 

Schedulability of G-EDF/LCM and G-RMA/LCM is better or equal to ECM and RCM, respectively, by proper choices for $\alpha_{min}$ and $\alpha_{max}$. Schedulability of G-EDF/LCM and G-RMA/LCM is better or equal to lock-free synchronization as long as $s_{max}/r_{max}$ does not exceed 0.5. By proper choice of $\alpha$s, $s_{max}/r_{max}$ can be increased to 2 under G-EDF/LCM, and to larger values under G-RMA/LCM.

ECM, RCM, and LCM are affected by transitive retry. Transitive retry occurs when a transaction accesses multiple objects. It causes a transaction to abort and retry  due to another non-conflicting transaction. PNF avoids transitive retry, and and also optimizes the processor usage by reducing the priority of the aborted transaction. This way, other tasks can proceed if they do not conflict with other executing transactions. 

We upper bounded PNF's retry cost and response time. We also compared PNF's schedulability to other synchronization techniques. PNF has better schedulability than lock-free synchronization as long as $s_{max}$ does not exceed $r_{max}$. 

We also implemented the CMs and conducted experimental studies. Our experimental studies revealed that the CMs' have shorter retry costs than lock-free synchronization. In particular, PNF has shorter retry costs than others as long as transitive retry and contention exist. However, PNF's implementation is relatively complex. 



\section{Proposed Post Preliminary Exam Research}

We propose the following research directions:

\subsection{Supporting Nested Transactions} 

Transactions can be nested \textit{linearly}, where each transaction has at most one pending transaction~\cite{Moss2006186}. Nesting can also be done in \textit{parallel}, where transactions execute concurrently within the same parent~\cite{volos2009nepaltm}. Linear nesting can be \textit{1) flat:} If a child transaction aborts, then the parent transaction also aborts. If a child commits, no effect is taken until the parent commits. Modifications made by the child transaction are only visible to the parent until the parent commits, after which they are externally visible. 
%
\textit{2) Closed:} Similar to \textit{flat nesting}, except that if a child transaction conflicts, it is aborted and retried, without aborting the parent, potentially improving concurrency over flat nesting. 
%
\textit{3) Open:} If a child transaction commits, its modifications are immediately externally visible, releasing memory isolation  of objects used by the child, thereby potentially improving concurrency over closed nesting. However, if the parent conflicts after the child commits, then compensating actions are executed to undo the actions of the child, before retrying the parent and the child. 


We propose to develop real-time contention managers that allow these different nesting models and establish their retry and response time upper bounds. Additionally, we propose to formally compare their schedulability with nested critical sections under lock-based synchronization. Note that, nesting is not viable under lock-free synchronization.


The real-time CMs proposed so far can be directly extended for different types of nesting. On conflict, the CM should decide on which transaction must be  aborted. In case of flat nesting, the outer transaction should be aborted and restarted. In case of closed and open nesting, only the interfered transaction should be restarted. In flat nesting, it will be useful to delay a lower priority transaction when it interferes with a higher priority one. Thus, the lower priority transaction need not be restarted from the beginning. This can reduce retry costs, especially when the inner most transaction is the conflicting one.


ECM and RCM can use the same criteria to determine which transaction to abort or wait. LCM may need a redefinition of the $\alpha$ parameter. Each child transaction can have its own $\alpha$. So, for each depth of nesting, there is a corresponding $\alpha$. Alternatively, there can be only one $\alpha$ for the outer transaction. Inner transactions do not have their $\alpha$s. The first choice may be suitable for closed and open nesting, while the second choice seems more suitable for flat nesting. 


PNF can be used directly with all types of nesting. The only requirement is that each transaction checks for conflicts with itself, as well as its inner transactions. Executing transactions under PNF are non-preemptive. Thus, any nested transaction will not be aborted. This requirement for PNF simplifies implementation, but makes no use of the nesting structure (i.e., PNF deals with transactions as if they are not nested). The previous requirement for PNF can be alleviated by checking for conflicts of any inner transaction only when the inner transaction begins. So, when a transaction starts, it checks its own objects -- not inner transactions -- against objects of executing transactions. If no conflict is found, the transaction executes. Consequently, when an inner transaction starts, it can detect a conflict with already executing transactions. Thus, the inner transaction should wait until other conflicting executing transactions commit. This choice for the PNF implementation may add additional waiting time to inner transactions. Different choices thus impose different tradeoffs. 


Worst case response time analysis, similar to the design of CMs, needs modifications  to cope with nested transactions. The simplest method is to combine all accessed objects in all nested transactions into one group, and consider that this group is accessed by only one non-nested transaction. This simplifies calculations but loosens the upper bounds, especially with closed and open nesting. Since closed and open nesting only restarts the inner aborted transaction, it is of no use to include the length of the outermost parent in the response time calculations.


\subsection{Combining and Optimizing LCM and PNF} 

LCM is designed to reduce the retry cost of a transaction when it is interfered close to the end of its execution. In contrast, PNF is designed to avoid transitive retry when transactions access multiple objects. An interesting direction is to combine the two contention managers to obtain the benefits of both algorithms. 

An important problem in developing such a LCM/PNF combination is that LCM allows aborts of running transactions depending on the $\alpha$ value, whereas PNF does not permit aborts of any executing transaction. Thus, the first approach to combine LCM and PNF is by considering the transitive retry level. If the transitive retry level is high, then PNF could be used. Otherwise, LCM could be used. Another way to combine LCM and PNF is to change the $\alpha$ value with time. Thus, at some time point, $\alpha$ can equal $0$. This means that the current transaction cannot be aborted by any other transaction. This is similar to executing transactions in PNF. However, $\alpha$ should not be $0$ all the time. Thus, $\alpha$ should change with time.

For such a combined LCM/PNF contention manager, importantly, we must  understand what are its schedulability advantages over that of LCM and PNF individually, and how such a combined CM behaves in practice. 


The current implementation of PNF is centralized. In this implementation, PNF acts as a centralized contention manager that controls all transactions. Contention managers are usually decentralized, as described in Section~\ref{sec:contention manager}. Each transaction usually maintains its own contention manager; this reduces overhead and transaction blocking. Thus, one optimization direction is to develop a decentralized implementation of PNF, to reduce overhead and blocking. Moreover, the current implementation of PNF uses locks, which increases overhead. Another optimization direction is therefore to develop a lock-free PNF implementation.


Design optimizations of LCM and PNF may also be possible to reduce their retry costs and response times, by considering additional criteria for resolving transactional conflicts. 

Developing a LCM/PNF combination and optimizing PNF and LCM implementations and designs constitute  our second research direction. 

\subsection{Formal and Experimental Comparison with Real-Time Locking} 

Lock-free synchronization offers numerous advantages over locking protocols, but (coarse-grain) locking protocols have had significant traction in real-time systems due to their good programmability (even though their concurrency is low).  Example such real-time locking protocols include PCP and its variants~\cite{chen1990dynamic,6031129,Rajkumar:1991:SRS:532621,sha1990priority}, multicore PCP (MPCP)~\cite{lakshmanan2009coordinated,rajkumar2002real}, SRP~\cite{Buttazzo:2004:HRC:1027504, baker1991stack}, multicore SRP (MSRP)~\cite{gai2003comparison}, PIP~\cite{easwaran2009resource}, FMLP~\cite{key-4,brandenburg2008implementation,holman2006locking}, and OMLP~\cite{Baruah:2007:TMG:1338441.1338647}. FMLP has been established to be superior to other protocols~\cite{brandenburg2008comparison}. How does their schedulability compare with that of the proposed contention managers? How do they compare in practice? These questions constitute our third research direction. 


OMLP~\cite{key-3} is similar to FMLP, and is simpler in implementation. OMLP does not require changes in the underlying scheduler (i.e., G-EDF) as FMLP does. Under OMLP, each resource has a FIFO queue of length at most equals number of processors, $m$, and a priority queue. Requests for each resource are enqueued in the corresponding FIFO queue. If the FIFO queue is full, requests are added to the priority queue according to the requesting job's priority. The head of the FIFO queue is the resource holding task. Other queued requests are suspended until their turn arrives. Under suspension oblivious analysis for OMLP, each job has a maximum blocking time that is proportional to number of processors. In suspension oblivious, the suspension time is added to task execution, in contrast to suspension aware analysis. This is why
OMLP is asymptotically optimal under suspension oblivious analysis. We intend to implement OMLP in the ChronOS real-time OS. We also propose to analytically and experimentally compare OMLP against different CMs, including that for nested and non-nested transactions.


\bibliographystyle{plain}
\bibliography{global_bibliography}




\end{document}