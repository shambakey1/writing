\documentclass[conference]{IEEEtran}

\usepackage{paralist}
\usepackage{comment}
\usepackage{amsthm}
\usepackage{amssymb,amsmath}
\usepackage{subfigure}
\usepackage{url}
\usepackage{graphicx}
\usepackage[ruled]{algorithm2e}

\newtheorem{clm}{Claim}

\usepackage{pslatex}

\begin{document}

\title{Concurrency Control for Embedded Real-Time Software Using STM and Checkpointing}

\begin{comment}
\author{\IEEEauthorblockN{Mohammed Elshambakey}
\IEEEauthorblockA{Electrical and Computer Engineering School, VT\\
Blacksburg, VA, USA, 24060\\
Email: shambake@vt.edu}
\and
\IEEEauthorblockN{Binoy Ravindran}
\IEEEauthorblockA{Electrical and Computer Engineering School, VT\\
Blacksburg, VA, USA, 24060\\
Email: binoy@vt.edu}}
\end{comment}

\maketitle

\begin{abstract}
We consider checkpointing with software transactional memory (STM) concurrency control for embedded multicore real-time software, and present a modified version of FBLT contention manager called \textit{Checkpointing FBLT} (CPFBLT). We upper bound transactional retries and task response times under CPFBLT, and identify when CPFBLT is a more appropriate alternative to FBLT without checkpointing. Our implementation and experimental studies reveal that CPFBLT has shorter response time than FBLT.
\end{abstract}

\section{Introduction}
\label{sec:intro}

Embedded systems sense physical processes and control their behaviour, typically through feedback loops. Since physical processes are concurrent, computations that control them must also be concurrent, enabling them to process multiple streams of sensor input and control multiple actuators, all concurrently while satisfying time constraints. Often, such computations need to concurrently read/write shared data objects. They must also process sensor input and react, while satisfying time constraints. 

The de facto standard for concurrent programming is the threads abstraction, and the 
de facto synchronization abstraction is locks. Lock-based concurrency control has significant programmability, scalability, and composability challenges~\cite{Herlihy:2006:AMP:1146381.1146382}.
 Transactional memory (TM) is an alternative synchronization model for shared memory objects that promises to alleviate these difficulties. With TM, code that read/write shared objects is organized as \textit{memory transactions}, which execute speculatively, while logging changes made to objects. Two transactions conflict if they access the same object and at least one access is a write. When that happens, a contention manager (CM)~\cite{Guerraoui:2005:TTT:1073814.1073863} resolves the conflict by aborting one and allowing the other to commit, yielding (the illusion of) atomicity. Aborted transactions are re-started, after rolling back the changes. In addition to a simple programming model, TM provides performance comparable to lock-free approach, especially for high contention and read-dominated workloads (see an example TM system's performance in~\cite{Saha:2006:MHP:1122971.1123001}), and is composable~\cite{Harris:2005:CMT:1065944.1065952}. TM has been proposed in hardware, called HTM, and in software, called STM, with the usual tradeoffs: HTM has lesser overhead, but needs transactional support in hardware; STM is available on any hardware.

Given STM's programmability, scalability, and composability advantages, it is a compelling concurrency control technique also for multicore embedded real-time software. However, this requires  bounding transactional  retries, as real-time threads which subsume transactions, must satisfy time constraints.  Retry bounds under STM are dependent on the CM policy at hand. 

Past real-time CM research proposed resolving transactional contention using dynamic and fixed priorities of parent threads. \cite{6045438,stmconcurrencycontrol:emsoft11,lcmdac2012} present  Earliest Deadline CM (ECM) and Rate Monotonic CM (RCM)~, which are used with global EDF (G-EDF) and global RMS (G-RMS) multicore real-time schedulers \cite{Davis:2011:SHR:1978802.1978814}.
In particular,~\cite{stmconcurrencycontrol:emsoft11} shows that ECM and RCM achieve higher schedulability -- i.e., greater number of task sets meeting their time constraints -- than lock-free synchronization only under some ranges for the maximum atomic section length. That range is significantly expanded with the Length-based CM (LCM) in~\cite{lcmdac2012}, increasing the coverage of STM's timeliness superiority. ECM, RCM, and LCM suffer from transitive retry -- i.e., a transaction aborts and retries due to another transaction with no shared objects between the two transactions -- and cannot handle multiple objects per transaction efficiently. These limitations are overcome with the Priority with Negative value and First access CM (PNF)~\cite{pnf_dac_asp,shambake_phd_proposal}. However, PNF requires prior knowledge of all objects accessed by each transaction. This significantly limits programmability, and is incompatible with dynamic STM implementations~\cite{Herlihy:2003:STM:872035.872048}. Additionally, PNF is a centralized CM, which increases overheads and retry costs, and has a complex implementation. First Bounded, Last Timestamp CM (or FBLT)~\cite{fblt}, in contrast to PNF, does not require prior knowledge of objects accessed by transactions. Moreover, FBLT allows each transaction to access multiple objects with shorter transitive retry cost than ECM, RCM and LCM. Additionally, FBLT is a decentralized CM and has a simpler implementation than PNF.

Checkpointing~\cite{Koskinen:2008:CCI:1378533.1378563} can be used to further reduce response time of threads with conflicting transactions. Under checkpointing, a transaction retreats to a previous control flow location upon conflict. So, an aborted transaction does not have to retreat to its beginning.

We introduce checkpointing FBLT (CPFBLT) that extends original FBLT with checkpointing (Section~\ref{sec:cpfblt_design}). Original FBLT will be refereed as non-checkpointing FBLT (NCPFBLT). We present the motivation for introducing checkpointing into FBLT (Section~\ref{sec:motivation}). We establish CPFBLT's retry and response time upper bounds under G-EDF and G-RMA schedulers (Section~\ref{sec:cpfblt rc}). We also identify the conditions under which CPFBLT is a better alternative to non-checkpointing FBLT (Section~\ref{sec:schedulabiltiy comparison}). We implement FBLT and CPFBLT in the Rochester STM framework~\cite{marathe2006lowering} and conduct experimental studies (Section~\ref{exp_eval}). Our results reveal that CPFBLT has shorter response time than non-checkpointing FBLT.

Thus, the paper's contribution is the use of checkpointing as a complementary tool to FBLT to enhance response time. CPFBLT, thus allows programmers to reap STM's significant programmability and composability benefits for a broader range of multicore embedded real-time software than what was previously possible.

\section{Related Work}
\label{sec:past}

Transactional-like concurrency control without using locks, for real-time systems, has been previously studied in the context of non-blocking data structures (e.g.,~\cite{anderson95realtime}). Despite their numerous advantages over locks 
(e.g., deadlock-freedom), their programmability has remained a challenge. Past studies show that they are best suited for simple data structures where their retry cost is competitive to the cost of lock-based synchronization~\cite{bc+08}.  In contrast, STM is semantically simpler~\cite{Herlihy:2006:AMP:1146381.1146382}, and is often the only viable lock-free solution for complex data structures (e.g., red/black tree)~\cite{key-1} and nested critical sections~\cite{Saha:2006:MHP:1122971.1123001}. STM concurrency control for real-time systems has been previously studied in~\cite{manson2006preemptible,fahmy2009bounding,sarni2009real,schoeberl2010rttm,key-1,barrosmanaging,stmconcurrencycontrol:emsoft11,lcmdac2012,pnf_dac_asp,fblt}.

\cite{manson2006preemptible} proposes a restricted version of STM. The proposed version in \cite{manson2006preemptible} is for uniprocessor not multi-processors systems. \cite{fahmy2009bounding} bounds response times in distributed  systems with STM synchronization. \cite{fahmy2009bounding} considers Pfair scheduling, limited to small atomic regions with fixed size. \cite{fahmy2009bounding} limits transaction execution to span at most two quanta. \cite{sarni2009real} presents real-time scheduling of transactions and serializes transactions based on deadlines. However, the work does not bound retries and response times. \cite{schoeberl2010rttm} proposes real-time HTM. \cite{schoeberl2010rttm} assumes that the worst case conflict between atomic sections of different tasks occurs when the sections are released at the same time. \cite{key-1} upper bounds retries and response times for ECM with G-EDF, and identify the tradeoffs with locking and lock-free protocols. Similar to~\cite{schoeberl2010rttm},~\cite{key-1} also assumes that the worst case conflict between atomic sections occurs when the sections are released simultaneously. The ideas in~\cite{key-1} are extended in~\cite{barrosmanaging}, which presents three real-time CM designs. \cite{key-1} does not upper bound retry cost nor schedulability. 
 
\cite{stmconcurrencycontrol:emsoft11} presents the ECM and RCM contention managers, and upper bounds transactional retries and task response times under them.~\cite{stmconcurrencycontrol:emsoft11} also identifies the conditions under which ECM and RCM are superior to lock-free techniques. In particular, \cite{stmconcurrencycontrol:emsoft11} shows that, STM's superiority holds only under some ranges for the maximum atomic section length.  Moreover,~\cite{stmconcurrencycontrol:emsoft11} restricts transactions to access only one object. \cite{lcmdac2012} presents the LCM contention manager, and upper bounds transactional retry cost and task response times for G-EDF and G-RMA schedulers. This work also compares (analytically and experimentally) LCM with ECM, RCM, and lock-free synchronization. However, similar to~\cite{stmconcurrencycontrol:emsoft11},~\cite{lcmdac2012} restricts transactions to access only one object. \cite{pnf_dac_asp} presents the PNF contention manager, which allows transactions to access  multiple objects and avoids the consequent transitive retry effect. The work also upper bounds transactional retries and task response times under G-EDF and G-RMA. However, PNF requires a-priori knowledge of the objects accessed by each transaction, which is not always possible, limits programmability, and is incompatible with dynamic STM implementations~\cite{Herlihy:2003:STM:872035.872048}. Additionally, PNF is a centralized CM.

\cite{fblt} presents the FBLT contention manager. In contrast to PNF, FBLT does not require prior knowledge of required objects by each transaction. FBLT premits multiple objects per transaction.
Under FBLT, each transaction can be aborted for a specific number of times. Afterwards, the transaction becomes non-preemptive. Non-preemptive transaction cannot be aborted except by another non-preemptive transaction. Non-preemptive transactions resolve conflicts based on the time they became non-preemptive. 

Previous CMs try to enhance response time of real-time tasks using different policies for conflict resolution. Previous CMs have no concern when an object is first accessed within the transaction. Thus, upon a conflict, an aborted transaction has to retry from its beginning. On the contrary, checkpointing does not require aborted transaction to restart from beginning, as long as the contested object is not initially accessed at the beginning of the transaction. Thus, Checkpointing can be plugged into different CMs to further improve response time. \cite{Koskinen:2008:CCI:1378533.1378563} introduces checkpointing as an alternative to closed nesting transactions\cite{turcu2012closed}. \cite{Koskinen:2008:CCI:1378533.1378563} uses boosted transactions \cite{Herlihy:2008:TBM:1345206.1345237} instead of closed nesting \cite{Peri:2011:CCE:1946143.1946152,turcu2012closed,6267834} to implement checkpointing. Booseted transactions are based on linearizable objects with abstract states and concrete implementation. Methods under boosted transaction have well defined semantics to transform objects from one state to another. Inverse methods are used to restore objects to previous states. Upon a conflict, a transaction does not need to revert to its beginning, but rather to a point where the conflict can be avoided. Thus, checkpointing enables partial aborts. \cite{turcu_phd_proposal} applies check pointing in distributed transactional memory using Hyflow \cite{Saad:2011:HHP:1996130.1996167}. Checkpointing showed performance improvement compared to flat (non-checkpointing) transactions.
%
\section{Preliminary}

We consider a multiprocessor system with $m$ identical processors and $n$ sporadic tasks $\tau_1, \tau_2,\ldots, \tau_n$. The $k^{th}$ instance (or job) of a task $\tau_i$ is denoted $\tau_i^k$. Each task $\tau_i$ is specified by its worst case execution time (WCET) $c_i$, its minimum period $T_i$ between any two consecutive instances, and its relative deadline $D_i$, where $D_i=T_i$. Job $\tau_i^j$ is released at time $r_i^j$ and must finish no later than its absolute deadline $d_i^j=r_i^j+D_i$. Under a fixed priority scheduler such as G-RMA, $p_i$ determines $\tau_i$'s (fixed) priority and it is constant for all instances of $\tau_i$. Under a dynamic priority scheduler such as G-EDF, a job $\tau_i^j$'s priority, $p_i^j$, differs from one instance to another. 
A task $\tau_j$ may interfere with task $\tau_i$ for a number of times during an interval $L$, and this number is denoted as $G_{ij}(L)$. 

\textit{Shared objects.}
 A task may need to read/write shared, in-memory data objects while it is executing any of its atomic sections (transactions), which are synchronized using STM. 
The set of atomic sections of task $\tau_i$ is denoted $s_i$. $s_i^k$ is the $k^{th}$ atomic section of $\tau_i$. Each object, $\theta$, can be accessed by multiple tasks. The set of distinct objects accessed by $\tau_i$ is $\Theta_i$. The set of atomic sections used by $\tau_i$ to access $\theta$ is $s_i(\theta)$, and the sum of the lengths of those atomic sections is $len(s_i(\theta))$. $s_i^k(\theta)$ is the $k^{th}$ atomic section of $\tau_i$ that accesses $\theta$.
%
 $s_i^k$ can access one or more objects in $\Theta_i$. So, $s_i^k$ refers to the transaction itself, regardless of the objects accessed by the transaction. We denote the set of all accessed objects by $s_i^k$ as $\Theta_i^k$. While $s_i^k(\theta)$ implies that $s_i^k$ accesses an object $\theta \in \Theta_i^k$, $s_i^k(\Theta)$ implies that $s_i^k$ accesses a set of objects $\Theta=\{\theta \in \Theta_i^k$ \}. $\bar{s_i^k}=\bar{s_i^k}(\Theta)$ refers only once to $s_i^k$, regardless of the number of objects in $\Theta$. So, $|\bar{s_i^k}(\Theta)|_{\forall \theta \in \Theta}=1$.
%
 $s_i^k(\theta)$  executes for a duration $len(s_i^k(\theta))$. $len(s_i^k)=len(s_i^k(\theta))=len(s_i^k(\Theta))=len(s_i^k(\Theta_i^k))$ The set of tasks sharing $\theta$ with $\tau_i$ is denoted $\gamma_i(\theta)$. 

The maximum-length atomic section in $\tau_i$ that accesses $\theta$ is denoted $s_{i_{max}} (\theta)$, while the maximum one among all tasks is $s_{max} (\theta)$, and the maximum one among tasks with priorities lower than that of $\tau_i$ is $s_{max}^i (\theta)$. $s_{max}^i(\Theta_h^i)=max\{s_{max}^i(\theta):\forall \theta \in \Theta_h^i\}$.

\textit{STM retry cost.} If two or more atomic sections conflict, the CM will commit one section and abort and retry the others, increasing the time to execute the aborted transactions. The increased time that an atomic section $s_i^p (\theta)$ will take to execute due to a conflict with another section $s_j^k (\theta)$, is denoted $W_{i}^{p}(s_{j}^{k}(\theta))$. If an atomic section, $s_i^p$, is already executing, and another atomic section $s_j^k$ tries to access a shared object with $s_i^p$, then $s_j^k$ is said to ``interfere" or ``conflict" with $s_i^p$. The job of $s_j^k$ is the ``interfering job", and the job of $s_i^p$ is the ``interfered job".

\textit{transitive retry}~\cite{pnf_dac_asp,fblt} is defined as the retrial of an atomic section $s_i^k(\Theta_i^k)$ due to another atomic section $s_j^l(\Theta_j^l)$, where $\Theta_i^k \cap \Theta_j^l = \emptyset$. $\Theta_i^*$ denotes the set of objects not accessed directly by atomic sections in $\tau_i$, but can cause transactions in $\tau_i$ to retry due to transitive retry. $\Theta_i^{ex}(=\Theta_i + \Theta_i^*)$ is the set of all objects that can cause transactions in $\tau_i$ to retry directly or through transitive retry. $\Theta_i^{k^{ex}}$ is the subset of objects in $\Theta_i^{ex}$ that can cause direct or transitive conflict to $s_i^k$. $\gamma_i^*$ is the set of tasks that accesses  objects in $\Theta_i^*$. $\gamma_i^{ex}(=\gamma_i + \gamma_i^*)$ is the set of all tasks that can directly or indirectly (through transitive retry) cause transactions in $\tau_i$ to abort and retry. $\gamma_i^k$ is the set of tasks that can directly cause $s_i^k$ to abort and retry. $\gamma_i^{k^{ex}}$ is the set of tasks that can directly or indirectly (through transitive retry) cause $s_i^k$ to abort and retry.

The total time that a task $\tau_i$'s atomic sections have to retry over $T_i$ is denoted $RC(T_i)$. The additional amount of time by which all interfering jobs of $\tau_j$ increases the response time of any job of $\tau_i$ during $L$, without considering retries due to atomic sections, is denoted $W_{ij}(L)$.

\section{Motivation}\label{sec:motivation}

Under checkpointing, a transaction $s_i^k \in \tau_i$ does not need to restart from the beginning upon a conflict on object $\theta$. $s_i^k$ just needs to return to the first point it accessed $\theta$. Thus, response time of $\tau_i$ can be improved by checkpointing unless $s_i^k$ acquires all its objects at its beginning. While CMs try to resolve conflicts using proper strategies, checkpointing enhances response time by reducing aborted part of each transaction. Thus, checkpointing acts as a complementary component to different CMs to further enhance response time.

Behaviour of some CMs, like PNF~\cite{pnf_dac_asp}, make checkpointing useless. PNF requires a priori knowledge of accessed objects within transactions. Only the first $m$ non-conflicting transactions are allowed to execute concurrently and non-preemptively. Thus, PNF makes no use of checkpointing because there is no conflict between non-preemptive transactions.

Other CMs (e.g., FBLT\cite{fblt}) allow conflicting transaction to run concurrently. So, FBLT can benefit from checkpointing. FBLT, by definition, depends on LCM. LCM, in turn, depends
on ECM (RCM) for G-EDF (G-RMA), respectively. Experimental results
show superiority of FBLT over LCM, ECM and RCM\cite{fblt}. Thus, we extend FBLT to checpointing FBLT (CPFBLT) to improve response time.

\section{Checkpointing FBLT (CPFBLT)}\label{sec:cpfblt_design}

CPFBLT depends on FBLT which in turn depends on LCM~\cite{lcmdac2012}. So, we initially illustrate LCM with required modification to implement checkpointing (Section~\ref{sec:cplcm}). Afterwards, we illustrate FBLT with checkpointing extension in (Section~\ref{sec:design_cpfblt}).

\subsection{Checkpointing LCM (CPLCM)}\label{sec:cplcm}

CPLCM is shown in Algorithm~\ref{alg:cplcm}. 
%
\begin{algorithm}[!htpd]
\footnotesize{
\LinesNumbered
\KwData{\\
$s_{i}^k\rightarrow$ interfered transaction.\\
$s_{j}^l\rightarrow$ interfering transaction with $s_i^k$ on object $\theta_{ij}^{kl}$.\\
$\psi\rightarrow$ predefined threshold $\in [0,1]$.\\
$\epsilon_i^k\rightarrow$ remaining execution length of $s_i^k$.\\
$cp_h^u(\theta) \rightarrow$ recorded checkpoint in transaction $s_h^u$ for newly accessed object $\theta$}
\KwResult{which transaction of $s_{i}^k$ or $s_{j}^l$ aborts}
\ForEach{newly accessed $\theta$ requested by any transaction $s_h^u$}{Add a checkpoint $cp_h^u(\theta)$\label{cplcm_step_add_new_cp}}
\eIf{$p_i^k > p_j^l$}
	{$s_{j}^l$ aborts and retreats to $cp_j^l(\theta_{ij}^{kl})$\label{cp-step_sicommits}\;
	Remove all checkpoints in $s_j^l$ recorded after $cp_j^l(\theta_{ij}^{kl})$\label{cp_step_rem_cp_jl_1}}
	{$c_{ij}^{kl}=len(s_j^l)/len(s_i^k)$\label{cp-step_cijkl}\;
	$\alpha_{ij}^{kl}=ln(\psi)/(ln(\psi)-c_{ij}^{kl})$\label{cp-step_alphaijkl}\;
	$\alpha=\left(len(s_i^k)-\epsilon_i^k\right)/len(s_i^k)$\label{cp_step_alpha}\;
	\eIf{$\alpha \le \alpha_{ij}^{kl}$}
	{$s_{i}^k$ aborts and retreats to $cp_i^k(\theta_{ij}^{kl})$\label{cp-step_siaborts}\;
	Remove all checkpoints in $s_i^k$ recorded after $cp_i^k(\theta_{ij}^{kl})$\label{cp_step_rem_cp_ik}}
	{$s_{j}^l$ aborts and retreats to $cp_j^l(\theta_{ij}^{kl})$\label{cp-step_sjaborts}\;
	Remove all checkpoints in $s_j^l$ recorded after $cp_j^l(\theta_{ij}^{kl})$\label{cp_step_rem_cp_jl_2}}
	}
	}
\caption{CPLCM}
\label{alg:cplcm}
\end{algorithm}
%
A new checkpoint is recorded for each newly accessed object $\theta$ by any transaction $s_h^u$ (step~\ref{cplcm_step_add_new_cp}). Checkpoint is recorded when $\theta$ is accessed for the first time because any further changes to $\theta$ will be discarded upon conflict. CPLCM uses the remaining length of $s_{i}^{k}$ when it is interfered, as well as $len(s_{j}^{l})$, to decide which transaction must be aborted. If $p_i^k > p_j^l$, then $s_{j}^l$ would be the transaction to abort because of lower priority of $s_j^l$, and earlier start of $s_{i}^k$ before $s_j^l$ (step~\ref{cp-step_sicommits}). Otherwise, $c_{ij}^{kl}$ is calculated (step~\ref{cp-step_cijkl}) to determine whether it is worth aborting $s_{i}^k$ in favour of $s_{j}^l$. If $len(s_j^l)$ is relatively small compared to $len(s_i^k)$, then $c_{ij}^{kl}$, and $\alpha_{ij}^{kl}$ tend to be small (steps~\ref{cp-step_cijkl},~\ref{cp-step_alphaijkl}). Consequently, $s_i^k$ aborts in favour of $s_j^l$. Also, if the remaining execution length of $s_i^k$ is long, then $\alpha$ tends to be small (step~\ref{cp_step_alpha}). Consequently, $s_i^k$ will abort in favour of $s_j^l$. When $s_i^k$ aborts upon a conflict with $s_j^l$ on object $\theta_{ij}^{kl}$, then checkpoints in $s_i^k$ recorded after $cp_i^k(\theta_{ij}^{kl})$ are removed (step~\ref{cp_step_rem_cp_ik}). Prior checkpoints to $cp_i^k(\theta_{ij}^{kl})$ remain the same. Also, if $s_j^l$ aborts in favour of $s_i^k$, then all checkpoints in $s_j^l$ recorded after $cp_j^l(\theta_{ij}^{kl})$ are removed (steps~\ref{cp_step_rem_cp_jl_1},~\ref{cp_step_rem_cp_jl_2}).

\subsection{Design of CPFBLT}\label{sec:design_cpfblt}

Algorithm~\ref{cpfblt-algorithm} illustrates CPFBLT. 
%
\begin{algorithm}[!htpd]
\footnotesize{
\LinesNumbered
\KwData{\\
$s_{i}^k$: interfered transaction.\\
$s_{j}^l$: interfering transaction.\\
$\delta_i^k$: the maximum number of times $s_i^k$ can be aborted during $T_i$.\\
$\eta_i^k$: number of times $s_i^k$ has already been aborted up to now.\\
$m\_$set: contains at most $m$ non-preemptive transactions. $m$ is number of processors.\\
$m\_prio$: priority of any transaction in $m\_$set. $m\_prio$ is higher than any priority of any real-time task.\\
$r(s_i^k)$: time point at which $s_i^k$ joined $m\_$set.\\
$cp_h^u(\theta) \rightarrow$ recorded checkpoint in transaction $s_h^u$ for newly accessed object $\theta$
}
\KwResult{which transaction, $s_i^k$ or $s_j^l$, aborts}
\ForEach{newly accessed $\theta$ requested by any transaction $s_h^u$}{Add a checkpoint $cp_h^u(\theta)$\label{cpfblt_step_add_new_cp}}
\uIf{\label{cpfblt-both preemptive}$s_i^k,\, s_j^l \not\in m\_set$}
{
%
Apply CPLCM (Algorithm~\ref{alg:cplcm})\label{apply cplcm}\;
%
\eIf{\label{cpfblt-preemptive s_i^k aborted}$s_{i}^k$ is aborted}
{
\eIf{$\eta_i^k<\delta_i^k$}
{
Increment $\eta_i^k$ by 1\label{cpfblt-increment eta 1}\;
}
{
Add $s_i^k$ to $m\_$set\label{cpfblt-add to m_set 1}\;
Record $r(s_i^k)$\label{cpfblt-record 1}\;
Increase priority of $s_i^k$ to $m\_prio$\label{cpfblt-increase priority 1}\;
}
}
{
Swap $s_{i}^k$ and $s_{j}^l$\;
Go to Step~\ref{cpfblt-preemptive s_i^k aborted}\;
}
}
\uElseIf{\label{cpfblt-s_j^l is non preemptive}$s_j^l \in m\_set, s_i^k \not\in m\_set$}
{
$s_{i}^k$ aborts and retreats to $cp_i^k(\theta_{ij}^{kl})$\;
Remove all checkpoints in $s_i^k$ recorded after $cp_i^k(\theta_{ij}^{kl})$\label{cpfblt_step_rem_cp_ik_1}\;
\eIf{$\eta_i^k < \delta_i^k$}
{
Increment $\eta_i^k$ by 1\label{cpfblt-increment eta 2}\;
}
{
Add $s_i^k$ to $m\_$set\label{cpfblt-add to m_set 2}\;
Record $r(s_i^k)$\label{cpfblt-record 2}\;
Increase priority of $s_i^k$ to $m\_prio$\label{cpfblt-increase priority 2}\;
}
}
\uElseIf{\label{cpfblt-s_i^k is non-preemptive}$s_i^k \in m\_set, s_j^l \not\in m\_set$}
{
Swap $s_{i}^k$ and $s_{j}^l$\;
Go to Step~\ref{cpfblt-s_j^l is non preemptive}\label{cpfblt-end preemptive and non preemptive}\;
}
\Else
{
\label{cpfblt-both non preemptive}
\eIf{$r(s_i^k)<r(s_j^l)$}
{	
$s_{j}^l$ aborts and retreats to $cp_j^l(\theta_{ij}^{kl})$\label{cpfblt-s_i^k first in m_set}\;
Remove all checkpoints in $s_j^l$ recorded after $cp_j^l(\theta_{ij}^{kl})$\label{cpfblt_step_rem_cp_jl}\;
}
{
$s_{i}^k$ aborts and retreats to $cp_i^k(\theta_{ij}^{kl})$\label{cpfblt-s_j^l first in m_set}\;
Remove all checkpoints in $s_i^k$ recorded after $cp_i^k(\theta_{ij}^{kl})$\label{cpfblt_step_rem_cp_ik_2}\;
}
}
}
\caption{The CPFBLT Algorithm}\label{cpfblt-algorithm}
\end{algorithm}
%
A new checkpoint is recorded for each newly accessed object $\theta$ by any transaction $s_h^u$ (step~\ref{cpfblt_step_add_new_cp}). Checkpoint is recorded when $\theta$ is accessed for the first time because any further changes to $\theta$ will be discarded upon conflict. Each transaction $s_{i}^{k}$ can be aborted during $T_i$ for at most $\delta_{i}^{k}$ times. $\eta_{i}^{k}$ records  the number of times $s_{i}^{k}$ has already been aborted up to now. If $s_i^k$ and $s_j^l$ have not joined the $m\_$set yet, then they are preemptive transactions. Preemptive transactions resolve conflicts using CPLCM (step~\ref{apply cplcm}). Thus, CPFBLT defaults to CPLCM when the conflicting transactions ($s_i^k$ and $s_j^l$) have not reached their $\delta$s ($\delta_i^k$ and $\delta_j^l$). $\eta_i^k$ is incremented each time $s_{i}^k$ is aborted as long as $\eta_i^k < \delta_i^k$ (steps~\ref{cpfblt-increment eta 1} and~\ref{cpfblt-increment eta 2}). Otherwise, $s_i^k$ is added to the $m\_$ set and priority of $s_{i}^k$ is increased to $m\_prio$ (steps~\ref{cpfblt-add to m_set 1} to~\ref{cpfblt-increase priority 1} and~\ref{cpfblt-add to m_set 2} to~\ref{cpfblt-increase priority 2}). When the priority of $s_i^k$ is increased to $m\_prio$, $s_i^k$ becomes a non-preemptive transaction. Non-preemptive transactions cannot be aborted by other preemptive transactions, nor by any other real-time job (steps~\ref{cpfblt-s_j^l is non preemptive} to~\ref{cpfblt-end preemptive and non preemptive}). The $m\_$set can hold at most $m$ concurrent transactions because there are $m$ processors in the system. $r(s_i^k)$ records the time $s_i^k$ joined the $m\_$set (steps~\ref{cpfblt-record 1} and~\ref{cpfblt-record 2}). When non-preemptive transactions conflict together (step~\ref{cpfblt-both non preemptive}), the transaction that joined $m\_$set first becomes the transaction that commits first (steps~\ref{cpfblt-s_i^k first in m_set} and~\ref{cpfblt-s_j^l first in m_set}). Thus, non-preemptive transactions are executed in FIFO order. When $s_i^k$($s_j^l$) aborts due to a conflict on $\theta_{ij}^{kl}$ with $s_j^l$($s_i^k$), then $s_i^k$($s_j^l$) retreats to $cp_i^k(\theta_{ij}^{kl})$($cp_j^l(\theta_{ij}^{kl})$), respectively. All checkpoints recorded after $cp_i^k(\theta_{ij}^{kl})$($cp_j^l(\theta_{ij}^{kl})$) are removed (steps~\ref{cpfblt_step_rem_cp_ik_1},~\ref{cpfblt_step_rem_cp_jl} and~\ref{cpfblt_step_rem_cp_ik_2}). 

\section{CPFBLT Retry Cost}\label{sec:cpfblt rc}

\begin{clm}\label{clm:basic_rc}
Assume only two transaction $s_i^k$ and $s_j^l$ conflicting together. Let $s_i^k$ begins at time $S\left(s_i^k\right)$ and $s_j^l$ begins at time $S\left(s_j^l\right)$. Let $\triangle=S\left(s_j^l\right)-S\left(s_i^k\right)$. In the absence of checkpointing, retry cost of $s_i^k$ due to $s_j^l$ is given by
%
\begin{equation}
BASE\_RC_{ij}^{kl} \le \begin{cases}
len\left(s_{j}^{l}\right)+\triangle & ,-len\left(s_{j}^{l}\right)\le\triangle\le len\left(s_{i}^{k}\right)\\
0 & ,\, Otherwise
\end{cases}
\label{eq:base_rc}
\end{equation}
%
$BASE\_RC_{ij}^{kl}$ is upper bounded by 
\begin{equation}
len\left(s_j^l\right)+\left(s_i^k\right)\label{eq:upper_base_rc}
\end{equation}
%
which is the same upper bound given by Proofs of Claims 1 and 3 in~\cite{stmconcurrencycontrol:emsoft11}
\end{clm}
%
\begin{proof}
Due to absence of checkpointing, $s_i^k$ aborts and retries from its beginning due to $s_j^l$. So, $s_i^k$ retries for the period starting from $S\left(s_i^k\right)$ to the end of execution of $s_j^l$. $s_j^l$ ends execution at $S\left(s_j^l\right)+len\left(s_j^l\right)$. If $S\left(s_j^l\right)<S\left(s_i^k\right)-len\left(s_j^l\right)$, then $s_j^l$ finishes execution before start of $s_i^k$ and there will be no conflict. Also, if $S\left(s_j^l\right)>S\left(s_i^k\right)+len\left(s_i^k\right)$, then $s_j^l$ starts execution after $s_i^k$ finishes execution and there will be no conflict. Thus,~(\ref{eq:base_rc}) follows. Equation~(\ref{eq:upper_base_rc}) is derived by substitution of $\triangle$ by its maximum value (i.e., $\left(s_{i}^{k}\right)$). Claim follows.
\end{proof}
%
\begin{clm}\label{clm:2_tx_cp_retry_cost}
Assume only two transactions $s_i^k$ and $s_j^l$ conflicting on one object $\theta$. Let $\nabla_{j}^{l}$ be the time interval between the start of $s_{j}^{l}$ and the first access to $\theta$. Similarly, let $\nabla_{i}^{k}$ be the time interval between the start of $s_i^k$ and the first access to $\theta$. Let $\triangle$ be the time difference between start of $s_j^l$ relative to start of $s_i^k$. So, $\triangle < 0$ if $s_j^l$ starts before $s_i^k$. Under checkpointing, $s_{i}^{k}$ aborts and retries due to $s_{j}^{l}$
for 
\begin{equation}
RC0_{ij}^{kl} \le \begin{cases}
len\left(s_{j}^{l}\right)-\nabla_{i}^{k}+\triangle & \mbox{, if }\begin{gathered}\triangle\ge\nabla_{i}^{k}-len\left(s_{j}^{l}\right)\\
\triangle\le len\left(s_{i}^{k}\right)-\nabla_{j}^{l}
\end{gathered}
\\
0 & \mbox{, Otherwise}
\end{cases}\label{eq:2_tx_cp_retry_cost}
\end{equation}
%
$RC0_{ij}^{kl}$ is upper bounded by 
\begin{equation}
len\left(s_{j}^{l}+s_{i}^{k}\right)-\nabla_{j}^{l}-\nabla_{i}^{k}\label{eq:rc0_upper_bound}
\end{equation}

\end{clm}
%
\begin{proof}
%
As $s_i^k$ and $s_j^l$ conflict only on one object $\theta$, there will be no conflict before both $s_i^k$ and $s_j^l$ access $\theta$. Retry cost of $s_i^k$ due to $s_j^l$ is derived by Claim~\ref{clm:basic_rc} excluding parts of $s_i^k$ and $s_j^l$ before both transactions access $\theta$. Thus, $len\left(s_i^k\right)$ in Claim~\ref{clm:basic_rc} is substituted by $len\left(s_i^k\right)-\triangle_i^k$. $len\left(s_j^l\right)$ is substituted by $len\left(s_j^l\right)-\triangle_j^l$. $\triangle$ in Claim~\ref{clm:basic_rc} is substituted by $\triangle+\nabla_j^l-\nabla_i^k$. Claim follows.
%
\end{proof}
%
\begin{clm}\label{clm:rc1_upper_bound}
%
Assume only two transactions $s_i^k$ and $s_j^l$ conflicting on a number of object $\theta_1,\,\theta_2\,...\,\theta_z$. Let $\nabla_{i*}^k$ be the time interval between start of $s_i^k$ and the first object $\theta_i$ accessed by $s_i^k$ and shared with $s_j^l$. Let $\nabla_{j*}^l$ be the time interval between start of $s_j^l$ and the first object $\theta_j$ accessed by $s_j^l$ and shared with $s_i^k$. $\theta_i$ and $\theta_j$ may not be the same. With checkpointing, retry cost of $s_i^k$ due to $s_j^l$ is upper bounded by 
%
\begin{equation}
RC1_{ij}^{kl} \le len\left(s_i^k+s_j^l\right)-\nabla_{i*}^k-\nabla_{j*}^l
\label{eq:rc1_upper_bound}
\end{equation}
%
\end{clm}
%
\begin{proof}
%
Proof follows directly from Claim~\ref{clm:2_tx_cp_retry_cost} by maximizing (\ref{eq:rc0_upper_bound}). $len\left(s_i^k\right)$, as well as, $len\left(s_j^l\right)$ in (\ref{eq:rc0_upper_bound}) cannot be changed. Thus, by choosing minimum values of $\nabla_i^k$ and $\nabla_j^l$ that correspond to shared objects between $s_i^k$ and $s_j^l$, (\ref{eq:rc0_upper_bound}) is maximized. Claim follows.
%
\end{proof}
%
\begin{clm}\label{clm:cp_with_transitive_retry}
If $s_j^l$ is conflicting indirectly (through transitive retry) with $s_i^k$, then it is safe to ignore $\nabla_{i*}^k$ in calculating the upper bound of retry cost of $s_i^k$ due to $s_j^l$.
\end{clm}
%
\begin{proof}
If $s_j^l$ is conflicting indirectly with $s_i^k$, then $s_j^l$ is accessing an object $\theta$ that does not belong to $\Phi_i^k$. In this case, to get an upper bound for the retry cost of $s_i^k$ due to $s_j^l$, $\nabla_{i*}^k$ assumes its minimum value in (\ref{eq:rc1_upper_bound}). Thus, $\nabla_{i*}^k=0$. Claim follows.
\end{proof}
%
\begin{clm}\label{clm:non_preemptive_2tx_cpfblt_rc}
Assume only two non-preemptive transactions $s_i^k$ and $s_j^l$ under CPFBLT. With checkpointing, retry cost of $s_i^k$ due to direct or indirect conflict with $s_j^l$ is upper bounded by 
%
\begin{equation}
RC2_{ij}^{kl} \le len\left(s_{j}^{l}\right)-\nabla_{i*}^{k}\label{eq:rc2_upper_bound}
\end{equation}
%
where $\nabla_{i*}^k=0$ in case of indirect conflict.
%
\end{clm}
%
\begin{proof}
Proof follows directly from Claims~\ref{clm:2_tx_cp_retry_cost},~\ref{clm:rc1_upper_bound} and~\ref{clm:cp_with_transitive_retry} except that $s_j^l$ must have become non-preemptive before $s_i^k$. So, $s_j^l$ starts execution non-preemptively before $s_i^k$. Otherwise, by definition of CPFBLT, $s_j^l$ will not be able to abort $s_i^k$. Thus, $\triangle$ must not exceed 0. Claim follows.
\end{proof}
%
\begin{clm}\label{clm:non_preemptive_all_tx_rc}
Let $s_i^k$ be a non-preemptive transaction under CPFBLT. Let $\chi_i^k$ be the set of transactions conflicting (directly or indirectly) with $s_i^k$. Each transaction $s_j^l \in \chi_i^k$ belongs to a distinct task. Transactions in $\chi_i^k$ are organized in non-increasing order of $RC2_{ij}^{kl}$ for each $s_j^l \in \chi_i^k$. Total retry cost of non-preemptive transaction $s_i^k$ due to other non-preemptive transactions is upper bounded by 
%
\begin{equation}
RC3_i^k \le \sum_{a=1}^{a=min\left(|\chi_i^k|, m-1\right)} RC2_i^k\left(\chi_i^k(a)\right)
\label{eq:non_preemptive_all_tx_rc}
\end{equation}
%
where $\chi_i^k(a)$ is the $a^{th}$ transaction in $\chi_i^k$. If $\chi_i^k(a)=s_j^l$, then $RC2_i^k\left(\chi_i^k(a)\right)=RC2_{ij}^{kl}$.
%
\end{clm}
%
\begin{proof}
By definition of CPFBLT, a transaction $s_i^k$ can be preceded by at most $m-1$ non-preemptive transactions. As non-preemptive transactions are organized in FIFO order, no two non-preemptive transactions can belong to the same task. Maximum retry cost of non-preemptive $s_i^k$ occurs when: 1) $s_i^k$ is preceded by at most $m-1$ transactions conflicting with $s_i^k$. 2) Each conflicting transaction $s_j^l$ to $s_i^k$ must have one of the highest $m-1$ values for $RC2_{ij}^{kl}$. 3) Non-preemptive transactions preceding $s_i^k$ are executing sequentially. Thus, retry cost of non-preemptive $s_i^k$ can be upper bounded by Claim~\ref{clm:non_preemptive_2tx_cpfblt_rc} for at most the first $m-1$ transactions in $\chi_i^k$. If the third condition is not satisfied, then (\ref{eq:non_preemptive_all_tx_rc}) still gives a correct, but not tight, upper bound. Claim follows.
\end{proof}
%
\begin{clm}\label{clm:delta_ik_rc}
%
Under CPFBLT, a preemptive transaction $s_i^k$ aborts and retries for at most
\begin{equation}
RC4_i^k \le \delta_i^k \left(len\left(s_i^k\right)-min\left(\nabla_{i*}^k\right)\right)
\label{eq:delta_ik_rc}
\end{equation}
%
where $min\left(\nabla_{i*}^k\right)$ is the minimum $\nabla_{i*}^k$ for $s_i^k$ and any other conflicting transaction $s_j^l$. If there are indirectly conflicting transactions with $s_i^k$, then $min\left(\nabla_{i*}^k\right)=0$.
%
\end{clm}
%
\begin{proof}
%
No transaction will make preemptive $s_i^k$ aborts and retries before $min\left(\nabla_{i*}^k\right)$. By checkpointing, $s_i^k$ will not retreat earlier than $min\left(\nabla_{i*}^k\right)$. By definition of CPFBLT, preemptive $s_i^k$ aborts for at most $\delta_i^k$ times before it becomes non-preemptive. Claim follows.
\end{proof}
%
\begin{clm}\label{clm:closed_nested_fblt_final}
%
The total retry cost of any job $\tau_i^x$ under CPFBLT due to 1) conflicts with other transactions during an interval $L$. 2) release of higher priority jobs during execution of preemptive transactions is upper bounded by
%
\begin{equation}
RC(L)_{to}^i=\sum_{s_i^k \in s_i} \left(RC4_i^k + RC3_i^k \right) + RC_{re}(L)
\label{eq:cpfblt_final}
\end{equation}
where $RC_{re}(L)$ is the retry cost resulting from the release of higher priority jobs during execution of preemptive transactions. $RC_{re}(L)$ is calculated by (6.8) in [10] for G-EDF, and (6.10) in [10] for G-RMA schedulers.
%
\end{clm}
%
\begin{proof}
%
Following Claims~\ref{clm:cp_with_transitive_retry},~\ref{clm:non_preemptive_all_tx_rc},~\ref{clm:delta_ik_rc}  and Claim 1 in~\cite{fblt}, Claim follows.
%
\end{proof}
%
Any newly released task $\tau_{i}^{x}$ can be blocked by $m$ lower priority non-preemptive nested transactions. Blocking time $D_{i}$ of any job $\tau_i^x$ is independent of checkpointing. Thus, $D_i$ is calculated by Claim 3 in \cite{fblt}. Claim 2 in \cite{fblt} is used to calculate response time under CPFBLT where $RC_{to}(T_{i})$ is calculated by (\ref{eq:cpfblt_final}).
%
\section{CPFBLT vs. NCPFBLT}\label{sec:schedulabiltiy comparison}

\begin{clm}\label{clm:cp_ncp_fblt_schedulabiltiy_comp}
%
Schedulability of CPFBLT is better or equal to schedulability of NCPFBLT if for each transaction $s_i^k$: initial access to shared objects between $s_i^k$ and other transactions occurs closer to the end of execution of $s_i^k$.
%
\end{clm}
%
\begin{proof}
%
Let upper bound on retry cost of any task $\tau_{i}^{x}$ during $T_{i}$ under NCPFBLT be denoted as $RC_{i}^{ncp}$. $RC_{i}^{ncp}$ is calculated by Claim 1 in \cite{fblt}. Let upper bound on retry cost of any task $\tau_{i}^{x}$ during $T_{i}$ under CPFBTL be denoted as $RC_{i}^{cp}$. $RC_{i}^{cp}$ is calculated by (\ref{eq:cpfblt_final}). Let $D_{i}$ be the upper bound on blocking time of any newly released task $\tau_{i}^{x}$ during $T_{i}$ due to lower priority jobs. $D_{i}$ is the same for both CPFBLT and NCPFBLT. $D_{i}$ is calculated by Claim 2 in \cite{fblt}. For CPFBLT schedulability to be better than schedulability of NCPFBLT: 
%
\begin{equation}
\sum_{\forall\tau_{i}}\frac{c_{i}+RC_{i}^{cp}+D_{i}}{T_{i}}\le\sum_{\forall\tau_{i}}\frac{c_{i}+RC_{i}^{ncp}+D_{i}}{T_{i}}\label{eq:schedulability_comparison}
\end{equation}
$\because$ $D_{i}$ and $c_{i}$ are the same for each $\tau_{i}$
under CPFBLT and NCPFBLT, then (\ref{eq:schedulability_comparison})
holds if:
\[
\forall\tau_{i},\, RC_{i}^{cp}\le RC_{i}^{ncp}
\]
%
\begin{eqnarray}
 & \delta_{i}^{k}\left(len\left(s_{i}^{k}\right)-min\left(\nabla_{i*}^{k}\right)\right)+\sum_{a=1}^{min\left(|\chi_{i}^{k}|,m-1\right)}\left(len\left(\chi_{i}^{k}(a)\right)-\nabla_{i*}^{k}\right)\nonumber\\
\le & \delta_{i}^{k}len\left(s_{i}^{k}\right)+\sum_{a=1}^{min\left(|\gamma_{i}^{k}|,m-1\right)}\left(len\left(\gamma_{i}^{k}(a)\right)\right)
\label{eq:cpfblt_vs_ncpfblt_1}
\end{eqnarray}
%
where $\gamma_i^k$ is the set of at most $m-1$ longest transactions conflicting directly or indirectly with $s_i^k$. Thus, $\gamma_i^k(a) \ge \chi_i^k(a), \forall a$. Thus, by increasing $\nabla_{i*}^k$, (\ref{eq:cpfblt_vs_ncpfblt_1}) holds. Claim follows.
%
\end{proof}
%
\section{Experimental Evaluation}\label{exp_eval}
%
We now would like to understand how CPFBLT's retry cost and response time compare with NCPFBLT in practice (i.e., on average). Since this can only be understood experimentally, we implement CPFBLT and NCPFBLT and conduct experiments. 

We used the ChronOS real-time Linux kernel~\cite{dellinger2011chronos} and the Rochester STM (RSTM) library~\cite{marathe2006lowering} in our implementation. We implemented G-EDF and G-RMA schedulers in ChronOS, and modified RSTM to include implementations of CPFBLT and NCPFBLT. We used an 8 core, 2GHz AMD Opteron platform. We used three task sets consisting of 4, 8, and 20 periodic tasks. Each task runs in its own thread and has a set of atomic sections. Atomic section properties are probabilistically controlled using three parameters: the maximum and minimum lengths of any atomic section within a task, and the total length of atomic sections within any task. For each run, $\nabla_{i^k,\,\forall s_i^k}$ lies in $\{0,0.2,0.4,0.6,0.8\}$.

%
\begin{figure}
\centering
\subfigure[4 tasks, G-EDF, Response time]{
\includegraphics[scale=0.7]
{/e/lectures/real-time/PhD-work/STM/Dropbox/Practical/results_uno/figures/4_tasks/Res_Time/fblt/NCP_vs_CP_Res_4t_50obj_100wr_-1eta_gedf.eps}
\label{fig:4t_gedf_res}
}
%~
\subfigure[4 tasks, G-RMA, Response time]{
\includegraphics[scale=0.7]
{/e/lectures/real-time/PhD-work/STM/Dropbox/Practical/results_uno/figures/4_tasks/Res_Time/fblt/NCP_vs_CP_Res_4t_50obj_100wr_-1eta_grma.eps}
\label{fig:4t_grma_res}
}
\caption{Average response time difference between NCPFBLT and CPFBLT for 4 tasks.}
\label{fig:4t_res}
\end{figure}
%
\begin{figure}
\centering
\subfigure[8 tasks, G-EDF, Response time]{
\includegraphics[scale=0.7]
{/e/lectures/real-time/PhD-work/STM/Dropbox/Practical/results_uno/figures/8_tasks/Res_Time/fblt/NCP_vs_CP_Res_8t_90obj_100wr_-1eta_gedf.eps}
\label{fig:8t_gedf_res}
}
%~
\subfigure[8 tasks, G-RMA, Response time]{
\includegraphics[scale=0.7]
{/e/lectures/real-time/PhD-work/STM/Dropbox/Practical/results_uno/figures/8_tasks/Res_Time/fblt/NCP_vs_CP_Res_8t_90obj_100wr_-1eta_grma.eps}
\label{fig:4t_grma_res}
}
\caption{Average response time difference between NCPFBLT and CPFBLT for 8 tasks.}
\label{fig:8t_res}
\end{figure}
%
%
\begin{figure}
\centering
\subfigure[20 tasks, G-EDF, Response time]{
\includegraphics[scale=0.7]
{/e/lectures/real-time/PhD-work/STM/Dropbox/Practical/results_uno/figures/20_tasks/Res_Time/fblt/NCP_vs_CP_Res_20t_210obj_100wr_-1eta_gedf.eps}
\label{fig:20t_gedf_res}
}
%~
\subfigure[20 tasks, G-RMA, Response time]{
\includegraphics[scale=0.7]
{/e/lectures/real-time/PhD-work/STM/Dropbox/Practical/results_uno/figures/20_tasks/Res_Time/fblt/NCP_vs_CP_Res_20t_210obj_100wr_-1eta_grma.eps}
\label{fig:20t_grma_res}
}
\caption{Average response time difference between NCPFBLT and CPFBLT for 20 tasks.}
\label{fig:20t_res}
\end{figure}
%
%
\begin{figure}
\centering
\subfigure[4 tasks, G-EDF, Retry cost]{
\includegraphics[scale=0.7]
{/e/lectures/real-time/PhD-work/STM/Dropbox/Practical/results_uno/figures/4_tasks/Abr_Dur/fblt/NCP_vs_CP_Abr_dur_4t_50obj_100wr_-1eta_gedf.eps}
\label{fig:4t_gedf_rc}
}
%~
\subfigure[4 tasks, G-RMA, Retry cost]{
\includegraphics[scale=0.7]
{/e/lectures/real-time/PhD-work/STM/Dropbox/Practical/results_uno/figures/4_tasks/Abr_Dur/fblt/NCP_vs_CP_Abr_dur_4t_50obj_100wr_-1eta_grma.eps}
\label{fig:4t_grma_rc}
}
\caption{Average retry cost difference between NCPFBLT and CPFBLT for 4 tasks.}
\label{fig:4t_rc}
\end{figure}
%
%
\begin{figure}
\centering
\subfigure[8 tasks, G-EDF, Retry cost]{
\includegraphics[scale=0.7]
{/e/lectures/real-time/PhD-work/STM/Dropbox/Practical/results_uno/figures/8_tasks/Abr_Dur/fblt/NCP_vs_CP_Abr_dur_8t_90obj_100wr_-1eta_gedf.eps}
\label{fig:8t_gedf_rc}
}
%~
\subfigure[8 tasks, G-RMA, Retry cost]{
\includegraphics[scale=0.7]
{/e/lectures/real-time/PhD-work/STM/Dropbox/Practical/results_uno/figures/8_tasks/Abr_Dur/fblt/NCP_vs_CP_Abr_dur_8t_90obj_100wr_-1eta_grma.eps}
\label{fig:8t_grma_rc}
}
\caption{Average retry cost difference between NCPFBLT and CPFBLT for 8 tasks.}
\label{fig:8t_rc}
\end{figure}
%
%
\begin{figure}
\centering
\subfigure[20 tasks, G-EDF, Retry cost]{
\includegraphics[scale=0.7]
{/e/lectures/real-time/PhD-work/STM/Dropbox/Practical/results_uno/figures/20_tasks/Abr_Dur/fblt/NCP_vs_CP_Abr_dur_20t_210obj_100wr_-1eta_gedf.eps}
\label{fig:20t_gedf_rc}
}
%~
\subfigure[20 tasks, G-RMA, Retry cost]{
\includegraphics[scale=0.7]
{/e/lectures/real-time/PhD-work/STM/Dropbox/Practical/results_uno/figures/20_tasks/Abr_Dur/fblt/NCP_vs_CP_Abr_dur_20t_210obj_100wr_-1eta_grma.eps}
\label{fig:20t_grma_rc}
}
\caption{Average retry cost difference between NCPFBLT and CPFBLT for 20 tasks.}
\label{fig:20t_rc}
\end{figure}
%
Average response time difference between NCPFBLT and CPFBLT for the 4, 8 and 12 task sets are shown in figures~\ref{fig:4t_gedf_res} to~\ref{fig:20t_grma_res} for both G-EDF and G-RMA. Results show benefits of checkpointing when combined with FBLT. As level of sharing extends to all objects, then any transaction $s_i^k$ under both CPFBLT and NCPFBLT can retreat to its beginning. Thus, in this case (i.e., objects can be shared from the beginning of each transaction), NCPFBLT can show some better results than CPFBLT as shown in the figures.

Figures~\ref{fig:4t_gedf_rc} to~\ref{fig:20t_grma_rc} show average retry cost difference between NCPFBLT and CPFBLT. Retry cost results can be misleading because of the negative difference between retry cost of NCPFBLT and CPFBLT. But this is natural due to behaviour of CPFBLT. Under NCPFBLT, a transaction $s_i^k$ returns to its beginning upon a conflict with $s_j^l$ on object $\theta$. Whereas, under CPFBLT, $s_i^k$ returns to the first point it accessed $\theta$. Thus, under CPFBLT, $s_i^k$ tries to access $\theta$ directly after returning to the proper checkpoint. But $s_j^l$ is still holding $\theta$. Accordingly, $s_i^k$ will abort and retry again. This retrial (donated as $RC_{cp}s_i^k$) is added to the accumulated retry cost of $s_i^k$ under CPFBLT. Under NCPFBLT, $s_i^k$ returns to its beginning when it conflicts with $s_j^l$. Thus, when $s_i^k$ reaches $\theta$ once again, $s_j^l$ may have finished execution. Let the time between start of $s_i^k$ and first access to $\theta$ be $\nabla_i^k(\theta)$. Accordingly, retry cost of $s_i^k$ under NCPFBLT can be less than retry cost of $s_i^k$ under CPFBLT. Whereas, $RC_{cp}s_i^k$ can be much less than $\nabla_i^k(\theta)$. Thus, $RC_{cp}s_i^k$ contributes by a smaller value to the response time of $\tau_i$, in contrast to $\nabla_i^k(\theta)$. This is why response time for CPFBLT is better than NCPFBLT, whereas, retry cost of NCPFBLT is less than CPFBLT.
\section{Conclusion}\label{sec:conclusion}

Past research on real-time CMs focused on developing different conflict resolution strategies for transactions. Except for LCM~\cite{lcmdac2012}, no policy was made to reduce the length of conflicting transactions. In this paper, we analysed effect of checkpointing over FBLT CM. Analysis shows that response time of CPFBLT can be reduced than NCPFBLT by properely delaying access to shared objects. Experimental evaluation reveals better response time for CPFBLT than NCPFBLT. Despite retry cost of NCPFBLT is lower than retry cost of CPFBLT, but this is natural as explained previously. Some CMs make no use of checkpointing due to behaviour of that CM (e.g, under PNF, all non-preemptive transactions are non-conflicting). Future work includes hyperid CM that combines checkpointing with non-checkpointing according to object sharing and initial access to shared objects.


\bibliographystyle{IEEEtran}
%\bibliographystyle{IEEEabrv}
\bibliography{%IEEEabrv,
/e/lectures/real-time/PhD-work/STM/Dropbox/writing/global_bibliography/global_bibliography}



% that's all folks
\end{document}
