% acmsmall-sample.tex, dated 24th May 2012
% This is a sample file for ACM small trim journals
%
% Compilation using 'acmsmall.cls' - version 1.3, Aptara Inc.
% (c) 2010 Association for Computing Machinery (ACM)
%
% Questions/Suggestions/Feedback should be addressed to => "acmtexsupport@aptaracorp.com".
% Users can also go through the FAQs available on the journal's submission webpage.
%
% Steps to compile: latex, bibtex, latex latex
%
% For tracking purposes => this is v1.3 - May 2012

\documentclass[prodmode,acmtecs]{acmsmall}

% Package to generate and customize Algorithm as per ACM style
\usepackage {paralist}
\usepackage{comment}
\usepackage{cite}
\usepackage{subfigure}
\usepackage{url}
\usepackage{graphicx}
\usepackage[ruled]{algorithm2e}
\renewcommand{\algorithmcfname}{ALGORITHM}
\newtheorem{clm}{Claim}
\newtheorem{mydef}{Definition}
\SetAlFnt{\small}
\SetAlCapFnt{\small}
\SetAlCapNameFnt{\small}
\SetAlCapHSkip{0pt}
\IncMargin{-\parindent}

% Metadata Information
\acmVolume{}
\acmNumber{}
\acmArticle{}
\acmYear{}
\acmMonth{}

% Document starts
\begin{document}

% Page heads
\markboth{M. Elshambakey et al.}{FBLT}

% Title portion
\title{FBLT}
\author{Mohammed Elshambakey
\affil{Virginia Tech University}
Binoy Ravindran
\affil{Virginia Tech University}}

\begin{abstract}
\bf{NEEDS TO BE WRITTEN}
\end{abstract}

\category{C.3}{Special-Purpose and Application-based Systems}{Real-time and embedded systems}

\terms{Design, Experimentation, Measurement}

\keywords{Software transactional memory (STM), real-time contention manager}

\acmformat{Elshambakey, M., Binoy, R. 2012. FBLT.}

\begin{bottomstuff}
New Paper, Not an Extension of a Conference Paper.\\
Author's addresses: M. Elshambakey {and} B. Ravindran, ECE Department,
Virginia Tech University, Blacksburg, VA 24060, USA.
\end{bottomstuff}

\maketitle



\section{FBLT vs. Other Synchronization Techniques}\label{schedulabiltiy comparison}
%%BR: Rename the section title as "Schedulability Comparison"

We now (formally) compare the schedulability of G-EDF (G-RMA) with FBLT against ECM, RCM, LCM, PNF, and lock-free synchronization~\cite{stmconcurrencycontrol:emsoft11,lcmdac2012,key-5,shambake_phd_proposal}. 
Such a comparison will reveal when FBLT outperforms the others. Toward this, we compare the total utilization under G-EDF (G-RMA)/FBLT with that under the other synchronization methods. In this comparison, we use the inflated execution time of each method, 
%%BR: ERROR! "...of each task,.."??
which is the sum of the worst-case execution time of the task and its retry cost, in the utilization calculation of the task.

No processor is available for $\tau_i^x$ during the blocking time. 
%%BR: Note that, you are not using the notation $\tau_i^x$ subsequently in this sentence. So the notation doesn't really serve any purpose, except to set the tone for the next sentences. I propose rewording as "Note that, for a job $\tau_i^x$, no processor is available during its blocking time."
Since each processor is busy with some job other than $\tau_i^x$, $D(\tau_i^x)$ is not added to the inflated execution time of $\tau_i^x$. Hence, $D(\tau_i^x)$ is not added to the utilization calculation of $\tau_i^x$.

Let $RC_{A}(T_{i})$ and $RC_{B}(T_{i})$ denote the retry cost of $\tau_{i}^{x}$ 
%%BR: "...of a job $\tau_{i}^{x}$..."
during $T_{i}$ using the synchronization method $A$ and synchronization
method $B$, respectively. 
%%BR: I reworded. Check. 
Now, schedulability of $A$ is comparable to $B$ if:
\begin{eqnarray}
\sum_{\forall\tau_{i}}\frac{c_{i}+RC_{A}(T_{i})}{T_{i}} & \le & \sum_{\forall\tau_{i}}\frac{c_{i}+RC_{B}(T_{i})}{T_{i}}\nonumber \\
\sum_{\forall\tau_{i}}\frac{RC_{A}(T_{i})}{T_{i}} & \le & \sum_{\forall\tau_{i}}\frac{RC_{B}(T_{i})}{T_{i}}\label{eq:utilization comparison}
\end{eqnarray}



\subsection{FBLT vs. ECM}

\begin{clm}\label{clm:fblt_ecm}
The schedulability of FBLT is equal to or better than ECM's when the maximum abort number of any preemptive transaction $s_i^k$ is less than or equal to the number of transactions directly conflicting with $s_i^k$ in all other jobs with higher priority than $\tau_{i}$'s current job. 
\end{clm}

\begin{proof}
By substituting $RC_{A}(T_{i})$ and $RC_{B}(T_{i})$ in (\ref{eq:utilization comparison})
with (\ref{eq:fblt_rc}) and (6.7) in \cite{shambake_phd_proposal}, 
respectively, we get: 
\begin{eqnarray}
 & \sum_{\forall\tau_{i}}\frac{\sum_{\forall s_{i}^{k}\in s_{i}}\left(\delta_{i}^{k}len(s_{i}^{k})+\sum_{s_{iz}^k\in \chi_i^k} len(s_{iz}^{k})\right)+RC_{re}(T_{i})}{T_{i}}\label{eq:fblt_edf_comparison_1}\\
\le & \sum_{\forall\tau_{i}}\frac{\left(\sum_{\forall\tau_{j}\in\gamma_{i}^{ex}}\sum_{\theta\in\theta_{i}^{ex}}\left(\left\lceil \frac{T_{i}}{T_{j}}\right\rceil \sum_{\forall\bar{s_{j}^{h}}(\theta)}len\left(\bar{s_{j}^{h}}(\theta)+s_{max}^{j}(\theta)\right)\right)\right)+RC_{re}(T_{i})}{T_{i}}\nonumber 
\end{eqnarray}

%%BR: STOPPED HERE. 

Let $\theta_{i}^{ex}=\theta_{i}+\theta_{i}^{*}$, where $\theta_{i}^{*}$
is the set of objects not accessed directly by $\tau_{i}$ but can
cause transactions in $\tau_{i}$ to retry due to transitive retry.
Let $\gamma_{i}^{ex}=\gamma_{i}+\gamma_{i}^{*}$, where $\gamma_{i}^{*}$
is the set of tasks that access objects in $\theta_{i}^{*}$. $\bar{s_{j}^{h}}(\theta)$
can access multiple objects, so $s_{max}^{j}(\theta)$ is the maximum
length transaction conflicting with $\bar{s_{j}^{h}}(\theta)$. $\bar{s_{j}^{h}}(\theta)$ is included only once for all $\theta \in \Theta_j^h$. Each $\theta \in \theta_i^{ex}$ has its own $s_{max}^j(\theta)$. But $s_i^h$ can access multiple objects, denoted as $\Theta_j^h$. So, $s_{max}^j(\theta)$ is replaced by $s_{max}^j(\Theta_j^h)$, where $s_{max}^j(\Theta_j^h)=max\{s_{max}^j(\theta),\forall \theta \in \Theta_j^h\}$.
%%BR: I inserted a period to end the previous sentence. 
 $s_{max}^j(\Theta_j^h)$ is included once for each $\theta \in \theta_i$. 
 
 
Each $\tau_i^x$ 
%%BR: "Each job $\tau_i^x$.."?? (ALWAYS introduce what is $\tau_i^x$ in similar situations. Fix this throughout the paper). 
has the same interference pattern from higher priority jobs, $\tau_j^h$, under FBLT and ECM. Hence, $RC_{re}(T_i)$ for $\tau_i^x$ is the same under FBLT and ECM. Consequently, (\ref{eq:fblt_edf_comparison_1})
becomes:
\begin{eqnarray}
 & \sum_{\forall\tau_{i}}\frac{\sum_{\forall s_{i}^{k}\in s_{i}}\left(\delta_{i}^{k}len(s_{i}^{k})+\sum_{s_{iz}^k\in \chi_i^k} len(s_{iz}^{k})\right)}{T_{i}}\label{eq:fblt_edf_comparison_3}\\
\le & \sum_{\forall\tau_{i}}\frac{\left(\sum_{\forall\tau_{j}\in\gamma_{i}^{ex}}\sum_{\forall \bar{s_{j}^{h}}(\Theta_j^h),\,\Theta_j^h\in\theta_{i}^{ex}}\left(\left\lceil \frac{T_{i}}{T_{j}}\right\rceil len\left(\bar{s_{j}^{h}}(\Theta_j^h)+s_{max}^{j}(\Theta_j^h)\right)\right)\right)}{T_{i}}\nonumber 
\end{eqnarray}


Although different $s_{i}^{k}$ 
%%BR: "...different $s_{i}^{k}$s.."
can have common conflicting transactions
$\bar{s_{j}^{h}}$, no more than one $s_{i}^{k}$ can be preceded
by the same $\bar{s_{j}^{h}}$ in the $m\_$set. This happens because
transactions in the $m\_$set are non-preemptive. The original priority
of transactions preceding $s_{i}^{k}$ in the $m\_$set can be 
lower or higher than the original priority of $s_{i}^{k}$. Since under
G-EDF, $\tau_{j}$ can have at least one job of higher
priority than $\tau_{i}^{x}$, $\left\lceil \frac{T_{i}}{T_{j}}\right\rceil \ge1$.
Thus, each one of the $s_{iz}^{k}$ 
%%BR: "...each $s_{iz}^{k}$ term..."
in the left hand side of (\ref{eq:fblt_edf_comparison_3})
is included in one of the $\bar{s_{j}^{h}}(\theta)$ 
%%BR: "...the $\bar{s_{j}^{h}}(\theta)$  term.."
in the right hand side of (\ref{eq:fblt_edf_comparison_3}). 
%
Now, (\ref{eq:fblt_edf_comparison_3}) holds if:
\begin{eqnarray}
 & \sum_{\forall\tau_{i}}\frac{\sum_{\forall s_{i}^{k}\in s_{i}}\delta_i^klen(s_{i}^{k})}{T_{i}}\label{eq:fblt_edf_comparison_4}\\
\le &
\sum_{\forall\tau_{i}}\frac{\sum_{\forall\tau_{j}\in\gamma_{i}^{ex}}\sum_{\forall \bar{s_{j}^{h}}(\Theta_j^h),\,\Theta_j^h\in\theta_{i}^{ex}}\left(\left\lceil \frac{T_{i}}{T_{j}}\right\rceil len\left(s_{max}^{j}(\Theta_j^h)\right)\right)}{T_{i}}\nonumber 
\end{eqnarray}

Since FBLT is required to bound the effect of transitive retry, only $\theta_i$ (not the whole $\theta_i^{ex}$) will be considered in (\ref{eq:fblt_edf_comparison_4}). Thus, ECM acts as if there were no transitive retry. Consequently, (\ref{eq:fblt_edf_comparison_4}) holds if:
\begin{eqnarray}
 & \sum_{\forall\tau_{i}}\frac{\sum_{\forall s_{i}^{k}\in s_{i}}\delta_i^klen(s_{i}^{k})}{T_{i}}\label{eq:fblt_edf_comparison_4_1}\\
\le &
\sum_{\forall\tau_{i}}\frac{\sum_{\forall\tau_{j}\in\gamma_{i}}\sum_{\forall \bar{s_{j}^{h}}(\Theta),\,\Theta\in(\theta_{i}\cap\Theta_j^h)}\left(\left\lceil \frac{T_{i}}{T_{j}}\right\rceil len\left(s_{max}^{j}(\Theta)\right)\right)}{T_{i}}\nonumber 
\end{eqnarray}
where $s_{max}^j(\Theta) \le s_{max}^j(\Theta_j^h)$. 

For each $s_{i}^{k}\in s_{i}$, there are a set of zero or more $\bar{s_{j}^{h}}(\Theta)\in\tau_{j},\,\forall\tau_{j}\ne\tau_{i}$
that are conflicting with $s_{i}^{k}$. Assuming this set of transactions conflicting with $s_{i}^{k}$ is denoted as $\eta_{i}^{k}=\left\{ \bar{s_{j}^{h}}(\Theta)\in\tau_{j}:\left(\Theta\in\(\theta_{i}\cap\Theta_j^h)\right)\wedge\left(\forall\tau_{j}\ne\tau_{i}\right)\wedge\left(\bar{s_{j}^{h}}(\Theta)\not\in\eta_{i}^{l},\, l\ne k\right)\right\} $.


The last condition $\bar{s_{j}^{h}}(\Theta)\not\in\eta_{i}^{l},\, l\ne k$
in the definition of $\eta_{i}^{k}$ ensures that common transactions $\bar{s_{j}^{h}}$
that can conflict with more than one transaction $s_{i}^{k}\in\tau_{i}$
are split among different $\eta_{i}^{k},\, k=1,..,|s_{i}|$. This
condition is necessary, because in ECM, no two or more transactions
of $\tau_{i}^{x}$ can be aborted by the same transaction of $\tau_{j}^{h}$, 
where $p_{j}^{h}>p_{i}^{x}$. By substitution of $\eta_{i}^{k}$ in
(\ref{eq:fblt_edf_comparison_4_1}), we get:
\begin{eqnarray}
 & \sum_{\forall\tau_{i}}\frac{\sum_{\forall s_{i}^{k}\in s_{i}}\delta_i^klen(s_{i}^{k})}{T_{i}}\label{eq:fblt_edf_comparison_5}\\
\le & \sum_{\forall\tau_{i}}\frac{\left(\sum_{\forall k=1}^{|s_{i}|}\sum_{\bar{s_{j}^{h}}(\Theta)\in\eta_{i}^{k}}\left(\left\lceil \frac{T_{i}}{T_{j}}\right\rceil len\left(s_{max}^{j}(\Theta)\right)\right)\right)}{T_{i}}\nonumber 
\end{eqnarray}

(\ref{eq:fblt_edf_comparison_5}) holds if for each $s_{i}^{k}\in\tau_{i}$:
\begin{equation}
\delta_{i}^{k}\le\frac{\sum_{\bar{s_{j}^{h}}(\Theta)\in\eta_{i}^{k}}\left(\left\lceil \frac{T_{i}}{T_{j}}\right\rceil len\left(s_{max}^{j}(\Theta)\right)\right)}{len(s_{i}^{k})}\label{eq:fblt_edf_comparison_6}
\end{equation}

Since $len\left(s_{max}^{j}(\Theta)\right)\ge len(s_{i}^{k})$, (\ref{eq:fblt_edf_comparison_6}) holds if $\delta_{i}^{k}\le \sum_{\bar{s_{j}^{h}}(\Theta)\in\eta_{i}^{k}}\left\lceil \frac{T_{i}}{T_{j}}\right\rceil$. 
%%BR: Missing period. I added.

$\sum_{\bar{s_{j}^{h}}(\Theta)\in\eta_{i}^{k}}\left\lceil \frac{T_{i}}{T_{j}}\right\rceil$
is the maximum number of transactions directly conflicting with $s_i^k$ in all jobs with higher priority than $\p_{i}^x$. Claim follows.
\end{proof}

\subsection{FBLT vs. RCM}

%%BR: STOPPED. 6:45PM, 6/15/2012. 

\begin{clm}\label{clm:fblt_rcm}
The schedulability of FBLT is equal to or better than RCM's when the maximum abort number
of any preemptive transaction $s_{i}^{k}$ is less than or equal to the number of transactions directly conflicting with $s_{i}^{k}$ in all jobs with higher priority than $\tau_{i}$ minus the sum of the maximum $m-1$ transactional lengths in all tasks.
%%BR: This sentence may be too difficult to understand. How about formulating it using an equation such as: "The schedulability of FBLT is equal to or better than RCM's when \begin{equation}..\leq....\end{equation} where ...."
\end{clm}
%
\begin{proof}
By substituting $RC_{A}(T_{i})$ and $RC_{B}(T_{i})$ in (\ref{eq:utilization comparison})
with (\ref{eq:fblt_rc}) and (6.9) in \cite{shambake_phd_proposal}, respectively, we get:
\begin{eqnarray}
 & \sum_{\forall\tau_{i}}\frac{\sum_{\forall s_{i}^{k}\in s_{i}}\left(\delta_i^klen(s_{i}^{k})+\sum_{s_{iz}^k\in \chi_i^k} len(s_{iz}^{k})\right)+RC_{re}(T_{i})}{T_{i}}\label{eq:fblt_rcm_comparison_1}\\
\le & \sum_{\forall\tau_{i}}\frac{\left(\sum_{\forall\tau_{j}^{*}\in\gamma_{i}^{ex}}\sum_{\forall\theta\in\theta_{i}^{ex}}\left(\left\lceil \frac{T_{i}}{T_{j}}\right\rceil +1\right)\sum_{\forall\bar{s_{j}^{h}(\theta)}}len\left(\bar{s_{j}^{h}(\theta)}+s_{max}^{j}(\theta)\right)\right)+RC_{re}(T_{i})}{T_{i}}\nonumber 
\end{eqnarray}
where $\tau_{j}^{*}=\left\{ \tau_{j}:\left(\tau_{j}\ne\tau_{i}\right)\wedge\left(p_{j}>p_{i}\right)\right\} $.


Let $\theta_{i}^{ex}=\theta_{i}+\theta_{i}^{*}$, where $\theta_{i}^{*}$
is the set of objects not directly accessed by $\tau_{i}$, 
%%BR: What is $\tau_{i}$? Introduce it by saying "...accessed by a job $\tau_{i}$,..." (DO THIS IN ALL  similar sentences throughout the paper. Otherwise, readers will be confused.)
but can cause transactions in $\tau_{i}$ to retry due to transitive retry.
%
Let $\gamma_{i}^{ex}=\gamma_{i}+\gamma_{i}^{*}$, where $\gamma_{i}^{*}$
is the set of tasks that access objects in $\theta_{i}^{*}$. $\bar{s_{j}^{h}}(\theta)$
can access multiple objects, so $s_{max}^{j}(\theta)$ is the maximum
length transaction conflicting with $\bar{s_{j}^{h}}(\theta)$. $\bar{s_{j}^{h}}(\theta)$ is included only once for all $\theta \in \Theta_j^h$. Each $\theta \in \theta_i^{ex}$ has its own $s_{max}^j(\theta)$. But $s_i^h$ can access multiple objects, denoted as $\Theta_j^h$. So, $s_{max}^j(\theta)$ is replaced by $s_{max}^j(\Theta_j^h)$, where $s_{max}^j(\Theta_j^h)=max\{s_{max}^j(\theta),\forall \theta \in \Theta_j^h\}$. $s_{max}^j(\Theta_j^h)$ is included once for each $\theta \in \theta_i$. 


Each $\tau_i^x$ has the same interference pattern from higher priority jobs, $\tau_j^h$, under FBLT and RCM. Hence, $RC_{re}(T_i)$ for $\tau_i^x$ is the same under FBLT and RCM. Consequently, (\ref{eq:fblt_rcm_comparison_1}) becomes:
\begin{eqnarray}
 & \sum_{\forall\tau_{i}}\frac{\sum_{\forall s_{i}^{k}\in s_{i}}\left(\delta_i^klen(s_{i}^{k})+\sum_{s_{iz}^k\in \chi_i^k} len(s_{iz}^{k})\right)}{T_{i}}\label{eq:fblt_rcm_comparison_2}\\
\le & \sum_{\forall\tau_{i}}\frac{\sum_{\forall\tau_{j}^{*}\in\gamma_{i}^{ex}}\sum_{\forall \bar{s_j^h}(\Theta_j^h),\Theta_j^h \in\theta_{i}^{ex}}\left(\left\lceil \frac{T_{i}}{T_{j}}\right\rceil +1\right)len\left(\bar{s_{j}^{h}}(\Theta_j^h)+s_{max}^{j}(\Theta_j^h)\right)}{T_{i}}\nonumber 
\end{eqnarray}


Although different $s_{i}^{k}$ 
%%BR: "...different $s_{i}^{k}$s..."??
can have common conflicting transactions
$\bar{s_{j}^{h}}$, no more than one $s_{i}^{k}$ can be preceded
by the same $\bar{s_{j}^{h}}$ in the $m\_$set. This happens because
transactions in the $m\_$set are non-preemptive. 
%
The original priority of transactions preceding $s_{i}^{k}$ in the $m\_$set can be of
lower or higher priority than the original priority of $s_{i}^{k}$. Under
G-RMA, $p_{j}>p_{i}$, which means that $T_{j}\le T_{i}$. Therefore, $\left\lceil \frac{T_{i}}{T_{j}}\right\rceil \ge1$.
For each $s_{i}^{k}\in s_{i}$, there are a set of zero or more $\bar{s_{j}^{h}}(\Theta_j^h)\in\tau_{j}^{*}$
that are conflicting with $s_{i}^{k}$. Assuming this set of 
transactions conflicting with $s_{i}^{k}$ is denoted as $\eta_{i}^{k}=\left\{ \bar{s_{j}^{h}}(\Theta_j^h)\in\tau_{j}^{*}:\left(\Theta_j^h\in\theta_{i}^{ex}\right)\wedge\left(\bar{s_{j}^{h}}(\Theta_j^h)\not\in\eta_{i}^{l},\, l\ne k\right)\right\} $.


The last condition $\bar{s_{j}^{h}}(\theta)\not\in\eta_{i}^{l},\, l\ne k$
in the definition of $\eta_{i}^{k}$ ensures that common transactions $\bar{s_{j}^{h}}$
that can conflict with more than one transaction $s_{i}^{k}\in\tau_{i}$
are split among different $\eta_{i}^{k},\, k=1,..,|s_{i}|$. This
condition is necessary, because in RCM, no two or more transactions
of $\tau_{i}^{x}$ can be aborted by the same transaction of $\tau_{j}^{h}$, 
where $p_{j}^{h}>p_{i}^{x}$. By substitution of $\eta_{i}^{k}$ in
(\ref{eq:fblt_rcm_comparison_2}), we get: 
\begin{eqnarray}
 & \sum_{\forall\tau_{i}}\frac{\sum_{\forall s_{i}^{k}\in s_{i}}\left(\delta_i^klen(s_{i}^{k})+\sum_{s_{iz}^k\in \chi_i^k} len(s_{iz}^{k})\right)}{T_{i}}\label{eq:fblt_rcm_comparison_4}\\
\le & \sum_{\forall\tau_{i}}\frac{\left(\sum_{\forall k=1}^{|s_{i}|}\sum_{\bar{s_{j}^{h}}(\Theta_j^h)\in\eta_{i}^{k}}\left(\left(\left\lceil \frac{T_{i}}{T_{j}}\right\rceil +1\right)len\left(\bar{s_{j}^{h}}(\Theta_j^h)+s_{max}^{j}(\Theta_j^h)\right)\right)\right)}{T_{i}}\nonumber 
\end{eqnarray}


$\bar{s_{j}^{h}}$ belongs to higher priority jobs than $\tau_{i}$. $s_{max}^{j}$ belongs to higher priority jobs than $\tau_{i}$ or $\tau_{i}$ itself. $s_{max}^{j}$ has a lower priority than $\tau_j$. Transactions in the $m\_$set can belong to jobs
with original priority higher or lower than $\tau_{i}$. Thus, (\ref{eq:fblt_rcm_comparison_4})
holds if for each $s_{i}^{k}\in\tau_{i}$:
\begin{equation}
\delta_i^klen(s_{i}^{k})\le\left(\sum_{\bar{s_{j}^{h}}(\Theta_j^h)\in\eta_{i}^{k}}\left(\left(\left\lceil \frac{T_{i}}{T_{j}}\right\rceil +1\right)len\left(\bar{s_{j}^{h}}(\Theta_j^h)+s_{max}^{j}(\Theta_j^h)\right)\right)\right)-\sum_{s_{iz}^k\in \chi_i^k} len(s_{iz}^{k})\label{eq:fblt_rcm_comparison_5}
\end{equation}
Then,
\begin{equation}
\delta_i^k\le\left(\sum_{\bar{s_{j}^{h}}(\Theta_j^h)\in\eta_{i}^{k}}\left(\left(\left\lceil \frac{T_{i}}{T_{j}}\right\rceil +1\right)len\left(\frac{\bar{s_{j}^{h}}(\Theta_j^h)+s_{max}^{j}(\Theta_j^h)}{s_{i}^{k}}\right)\right)\right)-\sum_{s_{iz}^k \in \chi_i^k} len\left(\frac{s_{iz}^{k}}{s_{i}^{k}}\right)\label{eq:fblt_rcm_comparison_14}
\end{equation}

Let $\epsilon=\left\{s_{u_{max}}:(1\le u \le n)\wedge \left(s_{u1_{max}} \ge s_{u2_{max}},\,u1 < u2 \right)\right\}$, where $n$ is the number of tasks, and $s_{u_{max}}$ is the maximum transactional length in any job of $\tau_u$. Thus, $\epsilon$ is the set of maximum transactional lengths of all tasks in non-increasing order. Each $s_{u_{max}} \in \epsilon$ belongs to a distinct task. Thus, $\sum_{s_{iz}^{k} \in \chi_i^k}len\left(\frac{s_{iz}^{k}}{s_{i}^{k}}\right)\le \sum_{u=1,\,s_{u_{max}}\in \epsilon}^{min(n,m)-1} s_{u_{max}}$. $\sum_{u=1,\,s_{u_{max}}\in \epsilon}^{min(n,m)-1} s_{u_{max}}$ is the sum of at most maximum $m-1$ transactional lengths of all tasks. $|\chi_i^k|\le m-1$ and $len(s_{max}^{j}(\Theta_j^h)) \ge len(s_{i}^{k})$. So, (\ref{eq:fblt_rcm_comparison_14})
holds if: 
\begin{equation}
\delta_i^k\le\left(\sum_{\bar{s_{j}^{h}}(\Theta_j^h)\in\eta_{i}^{k}}\left(\left\lceil \frac{T_{i}}{T_{j}}\right\rceil +1\right)\right)-\sum_{u=1,\,s_{u_{max}}\in \epsilon}^{min(n,m)-1} s_{u_{max}} \right)\label{eq:fblt_rcm_comparison_15}
\end{equation}

To bound the effect of transitive retry, only objects that belong to $\theta_i$ (not whole $\theta_i^{ex}$) will be considered. Thus, RCM acts as if there were no transitive retry.  Thus, $\eta_i^k$ is modified to $\bar{\eta_i^k}=\left\{ \bar{s_{j}^{h}}(\Theta)\in\tau_{j}^{*}:\left(\Theta \in \Theta_j^h \cap \theta_{i}\right)\wedge\left(\bar{s_{j}^{h}}(\Theta)\not\in\eta_{i}^{l},\, l\ne k\right)\right\}$. Since $\bar{\eta_i^k} \subseteq \eta_i^k$, (\ref{eq:fblt_rcm_comparison_15}) still holds if $\eta_i^k$ is replaced with $\bar{\eta_i^k}$.  Consequently, (\ref{eq:fblt_rcm_comparison_15}) holds if: 
%%BR: In situations like these when you are about to introduce an equation, always end the sentence with a semi-colon, like I have done ("...holds if: ")
\begin{equation}
\delta_i^k\le\left(\sum_{\bar{s_{j}^{h}}(\Theta)\in\bar{\eta_{i}^{k}}}\left(\left\lceil \frac{T_{i}}{T_{j}}\right\rceil +1\right)\right)-\sum_{u=1,\,s_{u_{max}}\in \epsilon}^{min(n,m)-1} s_{u_{max}} \right) \right)\label{eq:fblt_rcm_comparison_16}
\end{equation}

$\sum_{\bar{s_{j}^{h}}(\Theta)\in \bar{\eta_{i}^{k}}}\left(\left\lceil \frac{T_{i}}{T_{j}}\right\rceil +1\right)$
represents the number of directly conflicting transactions with $s_{i}^{k}$ 
%%BR: In situations like these, always say "....the number of transactions directly conflicting with $s_{i}^{k}$..." Fix here and in all similar situations throughout the paper (and in the future). 
in all jobs with higher priority than $\tau_{i}$. Claim follows.
\end{proof}

\subsection{FBLT vs. G-EDF/LCM}

\begin{clm}\label{clm:fblt_lcm_edf}
The schedulability of FBLT is equal to or better than G-EDF/LCM's when the maximum
abort number of each preemptive transaction $s_{i}^{k}$ is less than or equal to the sum of the total number each transaction
$s_{j}^{h}$ can directly conflict with $s_{i}^{k}$ multiplied by the maximum
$\alpha$ with which $s_{j}^{h}$ can conflict with the maximum length
transaction sharing objects with $s_{i}^{k}$ and $s_{j}^{h}$.
%%BR: This claim is totally UNREADABLE!! Reformat by introducing an equation: "The schedulability of FBLT is equal to or better than G-EDF/LCM's when \begin{equation}..."
%MUST ABSOLUTELY FIX THIS!!
\end{clm}

\begin{proof}
By substituting $RC_{A}(T_{i})$ and $RC_{B}(T_{i})$ in (\ref{eq:utilization comparison})
with (\ref{eq:fblt_rc}) and (6.7) in \cite{shambake_phd_proposal}, respectively, we get:
\begin{eqnarray}
 & \sum_{\forall\tau_{i}}\frac{\sum_{\forall s_{i}^{k}\in s_{i}}\left(\delta_i^klen(s_{i}^{k})+\sum_{\forall s_{iz}^{k}\in\chi_{i}^{k}}len(s_{iz}^{k})\right)+RC_{re}(T_{i})}{T_{i}}\label{eq:fblt_lcm_edf_comparison_1}\\
\le & \sum_{\forall\tau_{i}}\frac{\left(\sum_{\forall\tau_{j}\in\gamma_{i}^{ex}}\sum_{\theta\in\theta_{i}^{ex}}\left(\left\lceil \frac{T_{i}}{T_{j}}\right\rceil \sum_{\forall\bar{s_{j}^{h}(\theta)}}len\left(\bar{s_{j}^{h}(\theta)}+\bar{\alpha_{max}^{jh}}s_{max}^{j}(\theta)\right)\right)\right)}{T_{i}}\nonumber \\
+ & \sum_{\forall\tau_{i}}\frac{\left(\sum_{\forall s_{i}^{k}}\left(1-\alpha_{max}^{ik}\right)len\left(s_{max}^{i}\right)\right)+RC_{re}(T_{i})}{T_{i}}\nonumber 
\end{eqnarray}
%
Let $\theta_{i}^{ex}=\theta_{i}+\theta_{i}^{*}$, where $\theta_{i}^{*}$
is the set of objects not accessed directly by $\tau_{i}$, but can
cause transactions in $\tau_{i}$ to retry due to transitive retry.
Let $\gamma_{i}^{ex}=\gamma_{i}+\gamma_{i}^{*}$, where $\gamma_{i}^{*}$
is the set of tasks that access objects in $\theta_{i}^{*}$. $\bar{s_{j}^{h}}(\theta)$
can access multiple objects, so $s_{max}^{j}(\theta)$ is the maximum
length transaction conflicting with $\bar{s_{j}^{h}}(\theta)$. $\bar{s_{j}^{h}}(\theta)$ is included only once for all $\theta \in \Theta_j^h$. Each $\theta \in \theta_i^{ex}$ has its own $s_{max}^j(\theta)$. But $s_i^h$ can access multiple objects, denoted as $\Theta_j^h$. So, $s_{max}^j(\theta)$ is replaced by $s_{max}^j(\Theta_j^h)$, where $s_{max}^j(\Theta_j^h)=max\{s_{max}^j(\theta),\forall \theta \in \Theta_j^h\}$. 
 $s_{max}^j(\Theta_j^h)$ is included once for each $\theta \in \theta_i$. 
 
 
 Each $\tau_i^x$ has the same interference pattern from higher priority jobs, $\tau_j^h$, under FBLT and G-EDF/LCM. Hence, $RC_{re}(T_i)$ for $\tau_i^x$ is the same under FBLT and G-EDF/LCM. Consequently, (\ref{eq:fblt_lcm_edf_comparison_1}) holds if:
\begin{eqnarray}
 & \sum_{\forall\tau_{i}}\frac{\sum_{\forall s_{i}^{k}\in s_{i}}\left(\delta_i^klen(s_{i}^{k})+\sum_{\forall s_{iz}^{k}\in\chi_{i}^{k}}len(s_{iz}^{k})\right)}{T_{i}}\label{eq:fblt_lcm_edf_comparison_2}\\
\le & \sum_{\forall\tau_{i}}\frac{\left(\sum_{\forall\tau_{j}\in\gamma_{i}^{ex}}\sum_{\forall s_i^h(\Theta_j^h),\,\Theta_j^h\in
\theta_{i}^{ex}}\left(\left\lceil \frac{T_{i}}{T_{j}}\right\rceil len\left(\bar{s_{j}^{h}}(\Theta_j^h)+\bar{\alpha_{max}^{jh}}s_{max}^{j}(\Theta_j^h)\right)\right)\right)}{T_{i}}\nonumber \\
+ & \sum_{\forall\tau_{i}}\frac{\left(\sum_{\forall s_{i}^{k}}\left(1-\alpha_{max}^{ik}\right)len\left(s_{max}^{i}\right)\right)}{T_{i}}\nonumber 
\end{eqnarray}

Although different $s_{i}^{k}$ can have common conflicting transactions
$\bar{s_{j}^{h}}$, no more than one $s_{i}^{k}$ can be preceded
by the same $\bar{s_{j}^{h}}$ in the $m\_$set. This happens because
transactions in the $m\_$set are non-preemptive. The original priority
of transactions preceding $s_{i}^{k}$ in the $m\_$set can be of
lower or higher priority than the original priority of $s_{i}^{k}$. Under
G-EDF/LCM, $\tau_{j}\ne\tau_{i}$ can have at least one job of higher
priority than the current job of $\tau_{i}$. Hence, $\left\lceil \frac{T_{i}}{T_{j}}\right\rceil \ge1$.
Thus, each one of the $s_{iz}^{k}$
%%BR: "...each of the $s_{iz}^{k}$ terms..."
in the left hand side of (\ref{eq:fblt_lcm_edf_comparison_2})
is included in one of the $\bar{s_{j}^{h}}(\Theta_j^h)$ 
%BR: "...in one of the $\bar{s_{j}^{h}}(\Theta_j^h)$ terms..."
in the right hand side of (\ref{eq:fblt_lcm_edf_comparison_2}). Now, (\ref{eq:fblt_lcm_edf_comparison_2}) holds if: 
\begin{eqnarray}
 & \sum_{\forall\tau_{i}}\frac{\sum_{\forall s_{i}^{k}\in s_{i}}\delta_i^klen(s_{i}^{k})}{T_{i}}\label{eq:fblt_lcm_edf_comparison_4}\\
\le & \sum_{\forall\tau_{i}}\frac{\left(\sum_{\forall\tau_{j}\in\gamma_{i}^{ex}}\sum_{\forall\bar{s_{j}^{h}}(\Theta_j^h),\Theta_j^h\in\theta_{i}^{ex}}\left(\left\lceil \frac{T_{i}}{T_{j}}\right\rceil len\left(\bar{\alpha_{max}^{jh}}s_{max}^{j}(\Theta_j^h)\right)\right)\right)}{T_{i}}\nonumber \\
+ & \sum_{\forall\tau_{i}}\frac{\sum_{\forall s_{i}^{k}}\left(1-\alpha_{max}^{ik}\right)len\left(s_{max}^{i}\right)}{T_{i}}\nonumber 
\end{eqnarray}

To bound the effect of transitive retry, only $\theta_i$ (not the whole $\theta_i^{ex}$) will be considered in (\ref{eq:fblt_lcm_edf_comparison_4}). So, G-EDF/LCM acts as if there is no transitive retry. Consequently, (\ref{eq:fblt_lcm_edf_comparison_4}) holds if: 
\begin{eqnarray}
 & \sum_{\forall\tau_{i}}\frac{\sum_{\forall s_{i}^{k}\in s_{i}}\delta_i^klen(s_{i}^{k})}{T_{i}}\label{eq:fblt_lcm_edf_comparison_4_1}\\
\le & \sum_{\forall\tau_{i}}\frac{\left(\sum_{\forall\tau_{j}\in\gamma_{i}^}\sum_{\forall\bar{s_{j}^{h}}(\Theta),\Theta \in \Theta_j^h \cap \theta_{i}}\left(\left\lceil \frac{T_{i}}{T_{j}}\right\rceil len\left(\bar{\alpha_{max}^{jh}}s_{max}^{j}(\Theta)\right)\right)\right)}{T_{i}}\nonumber \\
+ & \sum_{\forall\tau_{i}}\frac{\sum_{\forall s_{i}^{k}}\left(1-\alpha_{max}^{ik}\right)len\left(s_{max}^{i}\right)}{T_{i}}\nonumber 
\end{eqnarray}
where $s_{max}^j(\Theta) \le s_{max}^j(\Theta_j^h)$. 
For each $s_{i}^{k}\in s_{i}$, there are a set of zero or more $\bar{s_{j}^{h}}(\Theta_j^h)\in\tau_{j},\,\forall\tau_{j}\ne\tau_{i}$
that are conflicting with $s_{i}^{k}$. Assuming this set of transactions conflicting with $s_{i}^{k}$ is denoted as $\eta_{i}^{k}=\left\{ \bar{s_{j}^{h}}(\Theta)\in\tau_{j}:\left(\Theta\in\theta_{i} \cap \Theta_j^h \right)\wedge\left(\forall\tau_{j}\ne\tau_{i}\right)\wedge\left(\bar{s_{j}^{h}}(\theta)\not\in\eta_{i}^{l},\, l\ne k\right)\right\} $.


The last condition $\bar{s_{j}^{h}}(\theta)\not\in\eta_{i}^{l},\, l\ne k$
in the definition of $\eta_{i}^{k}$ ensures that common transactions
$\bar{s_{j}^{h}}$ that can conflict with more than one transaction
$s_{i}^{k}\in\tau_{i}$ are split among different $\eta_{i}^{k},\, k=1,..,|s_{i}|$.
This condition is necessary, because in G-EDF/LCM, no two or more transactions
of $\tau_{i}^{x}$ can be aborted by the same transaction of $\tau_{j}^{h}$, 
where $p_{j}^{h}>p_{i}^{x}$. By substitution of $\eta_{i}^{k}$ in
(\ref{eq:fblt_lcm_edf_comparison_4}), we get:  
\begin{eqnarray}
 & \sum_{\forall\tau_{i}}\frac{\sum_{\forall s_{i}^{k}\in s_{i}}\delta_i^klen(s_{i}^{k})}{T_{i}}\label{eq:fblt_lcm_edf_comparison_5}\\
\le & \sum_{\forall\tau_{i}}\frac{\sum_{k=1}^{|s_{i}|}\sum_{\forall\bar{s_{j}^{h}}(\Theta)\in\eta_{i}^{k}}\left(\left\lceil \frac{T_{i}}{T_{j}}\right\rceil len\left(\bar{\alpha_{max}^{jh}}s_{max}^{j}(\Theta)\right)\right)}{T_{i}}\nonumber \\
+ & \sum_{\forall\tau_{i}}\frac{\left(\sum_{\forall s_{i}^{k}}\left(1-\alpha_{max}^{ik}\right)len\left(s_{max}^{i}\right)\right)}{T_{i}}\nonumber 
\end{eqnarray}

$\bar{s_{j}^{h}}$ belongs to higher priority jobs than $\tau_{i}$
and $s_{max}^{j}$ belongs to higher priority jobs than $\tau_{i}$
or $\tau_{i}$ itself. Transactions in the $m\_$ set can belong to jobs
with original priority higher or lower than $\tau_{i}$. Thus, (\ref{eq:fblt_lcm_edf_comparison_5})
holds if for each $s_{i}^{k}\in\tau_{i}$: 
\[
\delta_i^klen(s_{i}^{k})\le\left(\sum_{\forall\bar{s_{j}^{h}}(\Theta)\in\eta_{i}^{k}}\left(\left\lceil \frac{T_{i}}{T_{j}}\right\rceil \right)len\left(\bar{\alpha_{max}^{jh}}s_{max}^{j}(\Theta)\right)\right)+\left(1-\alpha_{max}^{ik}\right)len\left(s_{max}^{i}\right)
\]

This leads to:
%%BR: "This leads to" is always better than simply saying "Then" in similar situations when you are about to throw an equation. Do this consistently throughout. 
\begin{equation}
\delta_i^k\le\left(\sum_{\forall\bar{s_{j}^{h}}(\Theta)\in\eta_{i}^{k}}\left(\left\lceil \frac{T_{i}}{T_{j}}\right\rceil \right)len\left(\frac{\bar{\alpha_{max}^{jh}}s_{max}^{j}(\Theta)}{s_{i}^{k}}\right)\right)+\left(1-\alpha_{max}^{ik}\right)len\left(\frac{s_{max}^{i}}{s_{i}^{k}}\right)\label{eq:fblt_lcm_edf_comparison_6}
\end{equation}

Since $len\left(\frac{s_{max}^{j}(\Theta)}{s_{i}^{k}}\right)\ge1$, (\ref{eq:fblt_lcm_edf_comparison_6}) holds if: 
\begin{equation*}
\delta_i^k\le\left(\sum_{\bar{s_{j}^{h}}(\Theta)\in\eta_{i}^{k}}\left(\left\lceil \frac{T_{i}}{T_{j}}\right\rceil \bar{\alpha_{max}^{jh}}\right)\right)
\end{equation*}. Claim follows.
\end{proof}

%%BR: STOPPED HERE. 11:35PM. 

\subsection{FBLT vs. G-RMA/LCM}

\begin{clm}\label{clm:fblt_lcm_rma}
The schedulability of FBLT is equal to or better than G-RMA/LCM's when the maximum
abort number of each preemptive transaction $s_{i}^{k}$ is less than or equal to the sum of the maximum $m-1$ transactional lengths in all tasks subtracted from the sum of the total number
each higher priority transaction $s_{j}^{h}$ can directly conflict with $s_{i}^{k}$ times maximum $\alpha$ with which $s_{j}^{h}$ can conflict with maximum length transaction sharing objects with $s_{i}^{k}$ and $s_{j}^{h}$.
%%BR: This claim is totally UNREADABLE!! Reformat by introducing an equation: "The schedulability of FBLT is equal to or better than G-RMA/LCM's when \begin{equation}..."
%MUST ABSOLUTELY FIX THIS!!
\end{clm}
\begin{proof}
By substituing $RC_{A}(T_{i})$ and $RC_{B}(T_{i})$ in (\ref{eq:utilization comparison})
with (\ref{eq:fblt_rc}) and (6.9) in \cite{shambake_phd_proposal}, respectively, we get:
%%%^^^^^^^
\begin{eqnarray}
 & \sum_{\forall\tau_{i}}\frac{\sum_{\forall s_{i}^{k}\in s_{i}}\left(\delta_i^klen(s_{i}^{k})+\sum_{s_{iz}^{k}\in\chi_{i}^{k}}len(s_{iz}^{k})\right)+RC_{re}(T_{i})}{T_{i}}\label{eq:fblt_lcm_rma_comparison_1}\\
\le & \sum_{\forall\tau_{i}}\frac{\sum_{\forall\tau_{j}^{*}\in\gamma_{i}^{ex}}\sum_{\forall\theta\in\theta_{i}^{ex}}\left(\left\lceil \frac{T_{i}}{T_{j}}\right\rceil +1\right)\sum_{\forall\bar{s_{j}^{h}(\theta)}}len\left(\bar{s_{j}^{h}(\theta)}+\bar{\alpha_{max}^{jh}}s_{max}^{j}(\theta)\right)}{T_{i}}\nonumber \\
+ & \sum_{\forall\tau_{i}}\frac{\sum_{\forall s_{i}^{k}}\left(1-\alpha_{max}^{ik}\right)len(s_{max}^{i})+RC_{re}(T_{i})}{T_{i}}\nonumber 
\end{eqnarray}
where $\tau_{j}^{*}=\left\{ \tau_{j}:\left(\tau_{j}\ne\tau_{i}\right)\wedge\left(p_{j}>p_{i}\right)\right\} $.
Let $\theta_{i}^{ex}=\theta_{i}+\theta_{i}^{*}$, where $\theta_{i}^{*}$
is the set of objects not accessed directly by $\tau_{i}$ but can
cause transactions in $\tau_{i}$ to retry due to transitive retry.
Let $\gamma_{i}^{ex}=\gamma_{i}+\gamma_{i}^{*}$, where $\gamma_{i}^{*}$
is the set of tasks that access objects in $\theta_{i}^{*}$. $\bar{s_{j}^{h}}(\theta)$
can access multiple objects, so $s_{max}^{j}(\theta)$ is the maximum
length transaction conflicting with $\bar{s_{j}^{h}}(\theta)$. $\bar{s_{j}^{h}}(\theta)$ is included only once for all $\theta \in \Theta_j^h$. Each $\theta \in \theta_i^{ex}$ has its own $s_{max}^j(\theta)$. But $s_i^h$ can access multiple objects, denoted as $\Theta_j^h$. So, $s_{max}^j(\theta)$ is replaced by $s_{max}^j(\Theta_j^h)$, where $s_{max}^j(\Theta_j^h)=max\{s_{max}^j(\theta),\forall \theta \in \Theta_j^h\}$. $s_{max}^j(\Theta_j^h)$ is included once for each $\theta \in \theta_i$. 


Each $\tau_i^x$ has the same interference pattern from higher priority jobs, $\tau_j^h$, under FBLT and G-RMA/LCM. Hence, $RC_{re}(T_i)$ for $\tau_i^x$ is the same under FBLT and G-RMA/LCM. Consequently, (\ref{eq:fblt_lcm_rma_comparison_1}) becomes: 
\begin{eqnarray}
 & \sum_{\forall\tau_{i}}\frac{\sum_{\forall s_{i}^{k}\in s_{i}}\left(\delta_i^klen(s_{i}^{k})+\sum_{s_{iz}^{k}\in\chi_{i}^{k}}len(s_{iz}^{k})\right)}{T_{i}}\label{eq:fblt_lcm_rma_comparison_2}\\
\le & \sum_{\forall\tau_{i}}\frac{\sum_{\forall\tau_{j}^{*}\in\gamma_{i}^{ex}}\sum_
{\forall s_j^h(\Theta_j^h),\,\Theta_j^h\in\theta_{i}^{ex}}\left(\left\lceil \frac{T_{i}}{T_{j}}\right\rceil +1\right)len\left(\bar{s_{j}^{h}}(\Theta_j^h)+\bar{\alpha_{max}^{jh}}s_{max}^{j}(\Theta_j^h)\right)}{T_{i}}\nonumber \\
+ & \sum_{\forall\tau_{i}}\frac{\sum_{\forall s_{i}^{k}}\left(1-\alpha_{max}^{ik}\right)len(s_{max}^{i})}{T_{i}}\nonumber 
\end{eqnarray}

Although different $s_{i}^{k}$ can have common conflicting transactions
$\bar{s_{j}^{h}}$, no more than one $s_{i}^{k}$ can be preceded
by the same $\bar{s_{j}^{h}}$ in the $m\_$set. This happens because
transactions in the $m\_$set are non-preemptive. The original priority
of transactions preceding $s_{i}^{k}$ in the $m\_$set can be of
lower or higher priority than the original priority of $s_{i}^{k}$. Under
G-RMA, $p_{j}>p_{i}$, which means that $T_{j}\le T_{i}$. Hence, $\left\lceil \frac{T_{i}}{T_{j}}\right\rceil \ge1$.
For each $s_{i}^{k}\in s_{i}$, there are a set of zero or more $\bar{s_{j}^{h}}(\Theta_j^h)\in\tau_{j}^{*}$
that are conflicting with $s_{i}^{k}$. Assuming this set of transactions conflicting with $s_{i}^{k}$ is denoted as $\eta_{i}^{k}=\left\{ \bar{s_{j}^{h}}(\Theta_j^h)\in\tau_{j}^{*}:\left(\Theta_j^h\in\theta_{i}^{ex}\right)\wedge\left(\bar{s_{j}^{h}}(\Theta_j^h)\not\in\eta_{i}^{l},\, l\ne k\right)\right\} $.


The last condition $\bar{s_{j}^{h}}(\Theta_j^h)\not\in\eta_{i}^{l},\, l\ne k$
in the definition of $\eta_{i}^{k}$ ensures that common transactions
$\bar{s_{j}^{h}}$ that can conflict with more than one transaction
$s_{i}^{k}\in\tau_{i}$ are split among different $\eta_{i}^{k},\, k=1,..,|s_{i}|$.
This condition is necessary, because in G-RMA/LCM, no two or more transactions
of $\tau_{i}^{x}$ can be aborted by the same transaction of $\tau_{j}^{h}$, 
where $p_{j}^{h}>p_{i}^{x}$. By substitution of $\eta_{i}^{k}$ in
(\ref{eq:fblt_lcm_rma_comparison_2}), we get: 
\begin{eqnarray}
 & \sum_{\forall\tau_{i}}\frac{\sum_{\forall s_{i}^{k}\in s_{i}}\left(\delta_i^klen(s_{i}^{k})+\sum_{s_{iz}^{k}\in\chi_{i}^{k}}len(s_{iz}^{k})\right)}{T_{i}}\label{eq:fblt_lcm_rma_comparison_4}\\
\le & \sum_{\forall\tau_{i}}\frac{\left(\sum_{\forall k=1}^{|s_{i}|}\sum_{\bar{s_{j}^{h}}(\Theta_j^h)\in\eta_{i}^{k}}\left(\left\lceil \frac{T_{i}}{T_{j}}\right\rceil +1\right)len\left(\bar{s_{j}^{h}}(\Theta_j^h)+\bar{\alpha_{max}^{jh}}s_{max}^{j}(\Theta_j^h)\right)\right)}{T_{i}}\nonumber \\
+ & \sum_{\forall\tau_{i}}\frac{\sum_{\forall s_{i}^{k}}\left(1-\alpha_{max}^{ik}\right)len(s_{max}^{i})}{T_{i}}\nonumber 
\end{eqnarray}

$\bar{s_{j}^{h}}$ belongs to higher priority jobs than $\tau_{i}$. $s_{max}^{j}$ belongs to higher priority jobs than $\tau_{i}$ or $\tau_{i}$ itself. $s_{max}^{j}$ has a lower priority than $\tau_j$. Transactions in the $m\_$set can belong to jobs
with original priority higher or lower than $\tau_{i}$. So, (\ref{eq:fblt_lcm_rma_comparison_4})
holds if for each $s_{i}^{k}\in\tau_{i}$: 
\begin{eqnarray}
\delta_i^klen(s_{i}^{k}) & \le & \left(\sum_{\bar{s_{j}^{h}}(\Theta_j^h)\in\eta_{i}^{k}}\left(\left(\left\lceil \frac{T_{i}}{T_{j}}\right\rceil +1\right)len\left(\bar{s_{j}^{h}}(\Theta_j^h)+\bar{\alpha_{max}^{jh}}s_{max}^{j}(\Theta_j^h)\right)\right)\right)\nonumber\\
 & + & \left(1-\alpha_{max}^{ik}\right)len(s_{max}^{i})-\sum_{s_{iz}^{k}\in\chi_{i}^{k}}len(s_{iz}^{k})
 \nonumber
\end{eqnarray}

This leads to:
\begin{eqnarray}
\delta_i^k & \le & \left(\sum_{\bar{s_{j}^{h}}(\Theta_j^h)\in\eta_{i}^{k}}\left(\left(\left\lceil \frac{T_{i}}{T_{j}}\right\rceil +1\right)len\left(\frac{\bar{s_{j}^{h}}(\Theta_j^h)+\bar{\alpha_{max}^{jh}}s_{max}^{j}(\Theta_j^h)}{s_{i}^{k}}\right)\right)\right)\nonumber \\
 & + & \left(1-\alpha_{max}^{ik}\right)len\left(\frac{s_{max}^{i}}{s_{i}^{k}}\right)-\sum_{s_{iz}^{k}\in\chi_{i}^{k}}len\left(\frac{s_{iz}^{k}}{s_{i}^{k}}\right)\label{eq:fblt_lcm_rma_comparison_5}
\end{eqnarray}

%

Let $\epsilon=\left\{s_{u_{max}}:(1\le u \le n)\wedge \left(s_{u1_{max}} \ge s_{u2_{max}},\,u1 < u2 \right)\right\}$, where $n$ is the number of tasks, and $s_{u_{max}}$ is maximum transactional length in any job of $\tau_u$. Thus, $\epsilon$ is the set of maximum transactional lengths of all tasks in non-increasing order. Each $s_{u_{max}} \in \epsilon$ belongs to a distinct task. Thus, $\sum_{s_{iz}^k \in \chi_i^k}len\left(\frac{s_{iz}^{k}}{s_{i}^{k}}\right)\le \sum_{u=1,\,s_{u_{max}}\in \epsilon}^{min(n,m)-1} s_{u_{max}}$. $\sum_{u=1,\,s_{u_{max}}\in \epsilon}^{min(n,m)-1} s_{u_{max}}$ is the sum of at most maximum $m-1$ transactional lengths of all tasks. $|\chi_i^k|\le m-1$ and $len(s_{max}^{j}(\Theta_j^h)) \ge len(s_{i}^{k})$. So, (\ref{eq:fblt_lcm_rma_comparison_5}) holds if: 
\begin{equation}
\delta_i^k\le\left(\sum_{\bar{s_{j}^{h}}(\Theta_j^h)\in\eta_{i}^{k}}\left(\left\lceil \frac{T_{i}}{T_{j}}\right\rceil +1\right)\bar{\alpha_{max}^{jh}}\right)-\sum_{u=1,\,s_{u_{max}}\in \epsilon}^{min(n,m)-1} s_{u_{max}}\label{eq:fblt_lcm_rma_comparison_6}
\end{equation}

To bound the effect of transitive retry, only objects that belong to $\theta_i$ (not whole $\theta_i^{ex}$) will be considered. So, G-RMA/LCM acts as if there were no transitive retry. Thus, $\eta_i^k$ is modified to $\bar{\eta_i^k}=\left\{ \bar{s_{j}^{h}}(\Theta)\in\tau_{j}^{*}:\left(\Theta \in \Theta_j^h \cap \theta_{i}\right)\wedge\left(\bar{s_{j}^{h}}(\Theta)\not\in\eta_{i}^{l},\, l\ne k\right)\right\}$. Since $\bar{\eta_i^k} \subseteq \eta_i^k$, (\ref{eq:fblt_lcm_rma_comparison_6}) still holds if $\eta_i^k$ is replaced with $\bar{\eta_i^k}$.  Consequently, (\ref{eq:fblt_lcm_rma_comparison_6}) holds if:
\begin{equation}
\delta_i^k\le\left(\sum_{\bar{s_{j}^{h}}(\Theta)\in \bar{\eta_{i}^{k}}}\left(\left\lceil \frac{T_{i}}{T_{j}}\right\rceil +1\right)\bar{\alpha_{max}^{jh}}\right)-\sum_{u=1,\,s_{u_{max}}\in \epsilon}^{min(n,m)-1} s_{u_{max}}\label{eq:fblt_lcm_rma_comparison_7}
\end{equation}

$\left(\sum_{\bar{s_{j}^{h}}(\Theta) \in \bar{\eta_{i}^{k}}}\left(\left\lceil \frac{T_{i}}{T_{j}}\right\rceil +1\right)\bar{\alpha_{max}^{jh}}\right)$ is the sum of the total number of times each transaction $s_j^h$ can directly conflict with $s_i^k$ multiplied by the maximum $\alpha$ with which $s_j^h$ can conflict with the maximum length transaction sharing objects with $s_i^k$ and $s_j^h$. 
%%BR: This sentence is way too long and unreadable! Can you break it down and reogranize? (Same comment for similar sentences previously -- fix them as welll). 
Claim follows.
\end{proof}
%%BR: Since this proof is quite similar to G-EDF/LCM's proof, I suggest that you skip it in the paper, by saying: "\begin{proof} The proof is similar to that of Claim~\ref{..} and is therefore skipped for  brevity.  It can be found in~\cite{TR}."
%%BR: Be sure to put out the TR then. 

%%BR: DONE 12:20AM, 6/16/2012. 

% Bibliography
\bibliographystyle{acmsmall}
\bibliography{global_bibliography}

% History dates
\received{}{}{}

% Electronic Appendix
%\elecappendix

\medskip


\end{document}

