%% LyX 2.0.0 created this file.  For more info, see http://www.lyx.org/.
%% Do not edit unless you really know what you are doing.
\documentclass[english]{article}
\usepackage[T1]{fontenc}
\usepackage[latin9]{inputenc}
\usepackage{babel}
\begin{document}

\subsection{Real-Time Lock-free Synchronization}

Transactional-like concurrency control and lock-free synchronization,
for real-time systems, has been previously studied in (e.g.,~\cite{anderson95realtime,anderson1997lock,684918,1203552,896371,anderson96framework,811240}).
Despite their numerous advantages over locks (e.g., deadlock-freedom),
their programmability has remained a challenge. Past studies show
that they are best suited for simple data structures where their retry
cost is competitive to the cost of lock-based synchronization~\cite{bc+08}.
In contrast, STM is semantically simpler~\cite{Herlihy:2006:AMP:1146381.1146382},
and is often the only viable lock-free solution for complex data structures
(e.g., red/black tree)~\cite{key-1} and nested critical sections~\cite{Saha:2006:MHP:1122971.1123001,Agrawal:2008:SOT:1378533.1378553,Peri:2011:CCE:1946143.1946152}.


\subsection{Transactional Memory: Overview }

TM is motivated by database transactions \textemdash{} unit of work
of a database system, with ACID properties \cite{Gray:1981:TCV:1286831.1286846}.
In TM, concurrent threads synchronize via transactions when they access
shared memory. A TM transaction is an explicitly delimited sequence
of steps executed atomically by a thread \cite{harris.larus.rajwar}.
Atomicity implies all-or-nothing: the sequence of steps (i.e., reads
and writes) logically occur at a single instant in time; intermediate
states are invisible to other transactions. The semantic difficulty
of locks and the resulting high development and maintenance costs
have been the driving motivation for seeking alternate concurrency
control abstractions. The design of lock-free and wait-free data structures
are one such alternative. These approaches are highly performant,
but significantly complex to write and reason about, and therefore,
have largely been limited to a small set of basic data structures
\textemdash{} e.g., linked lists, queues, stacks \cite{1508449,4079519,1656921,Cho:2006:UAP:1141277.1141490}.

The term \textquotedblleft{}transactional memory\textquotedblright{}
was proposed by Herlihy and Moss {[}5{]}, where they presented hardware
support for lock-free data structures. TM has been provided in hardware
(HTM) \cite{698569,Martinez:2002:SSA:605397.605400,Oplinger:2002:ESR:605397.605417,Rajwar:2002:TLE:605397.605399,ham04},
software (STM) \cite{4145102,Dolev:2008:CSC:1400751.1400769,Harris:2003:LSL:949305.949340,Harris:2005:CMT:1065944.1065952,Cho:2006:UAP:1141277.1141490,Marathe:2004:DTM:1066650.1066660,Yoo:2008:ATS:1378533.1378564,Saha:2006:ASS:1194816.1194838,Shpeisman:2007:EIO:1250734.1250744}
and hybrid TM \cite{dam06,Kumar:2006:HTM:1122971.1123003,Minh:2007:EHT:1250662.1250673,Shriraman:2007:IHA:1250662.1250676}.
Hybrid TM allows STM to exploit any available HTM support to improve
performance. One basic problem in STM is how to correctly and efficiently
resolve conflicts when multiple threads access one shared objects
simultaneously and at least one writes inot it. Generally, a contention
manager (CM) is responsible to ensure that all transactions eventually
commit by choosing which transaction to delay or abort and when to
restart the aborted transaction in case of conflicts {[}\textbf{REFERNCE
FROM NEW PAPERS}{]}.


\subsection{Real-Time STM}

STM concurrency control for real-time systems has been previously
studied in~\cite{manson2006preemptible,fahmy2009bounding,sarni2009real,schoeberl2010rttm,key-1,barrosmanaging,6045438}.
\cite{manson2006preemptible} proposes a restricted version of STM
for uniprocessors. \cite{fahmy2009bounding} bounds response times
in distributed multiprocessor systems with STM synchronization. They
consider Pfair scheduling, limit to small atomic regions with fixed
size, and limit transaction execution to span at most two quanta.
\cite{sarni2009real} presents real-time scheduling of transactions
and serializes transactions based on deadlines. However, the work
does not bound retries and response times, nor establishes tradeoffs
against lock-free synchronization. \cite{schoeberl2010rttm} proposes
real-time HTM. The retry bound developed in~\cite{schoeberl2010rttm}
assumes that the worst case conflict between atomic sections of different
tasks occurs when the sections are released at the same time. \cite{6045438}
presents earliest-deadline CM or ECM. ECM resolves conflicts by aborting
the transaction with longer absoluted deadline. \cite{6045438} derives
a number of properties for ECM, upper bounds transactional retrys,
and compares schedulability of ECM to retry-loop lock-free synchronization
\cite{key-1}. \cite{6045438}'s assumptions, although simplifies
analysis, can lead to pessimistic upper bounds.

The past work that is closest to ours is~\cite{key-1}, which upper
bounds retries and response times for EDF CM with G-EDF, and identifies
the tradeoffs against locking and lock-free protocols. Similar to~\cite{schoeberl2010rttm},~\cite{key-1}
also assumes that the worst case conflict between atomic sections
occurs when the sections are released simultaneously. In addition,
we consider RMA CM, besides EDF CM. The ideas in~\cite{key-1} are
extended in~\cite{barrosmanaging}, which presents three real-time
CM designs. However, no retry bounds or schedulability analysis is
presented in~\cite{barrosmanaging}.
\end{document}
